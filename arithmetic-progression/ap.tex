\documentclass[aspectratio=1610,8pt]{beamer}

% Standard packages

\usepackage[english]{babel}
%\usepackage[latin1]{inputenc}
%\usepackage{times}
%\usepackage[T1]{fontenc}


% Setup TikZ

\usepackage{tikz}
\usetikzlibrary{arrows}
\tikzstyle{block}=[draw opacity=0.7,line width=1.4cm]


% Author, Title, etc.

\title{Arithmetic Progression}

\author[Shiv Shankar Dayal]
       {
	       Shiv Shankar Dayal
       }

       % The main document

\begin{document}
\begin{frame}
       \titlepage
\end{frame}

\begin{frame}{Sequence and Series}
  \textbf{Sequence:} A succession of numbers $t_1, t_2, t_3, \ldots ,t_n,
  \ldots$ formed according to some definite rule is called a sequence. $t_1,
  t_2, t_3, \ldots, t_n$ are called first, second, third, \ldots, $n$th term
  respectively.

  Alternatively, a sequence is a function whose domain is the set of natural
  numbers $N$ or a subset of $N$ and range is a set of real numbers.

  \textbf{Finite and Infinite Sequences:} A sequnece is called a finite
  sequence if it has finite number of elements and is called an infinite
  sequence if it has infinite number of elements.

  \textbf{Series:}By adding or subtracting the terms of a sequence, we get an
  expression which is called a series. If $a_1, a_2, a_3, \ldots, a_n$` is a
  sequence then $a_1 + a_2 + a_3 + \ldots +a_n$ is a series.

  \textbf{Progression:} It is not mandatory for terms of a sequence to follow a
  pattern or formula for its $n$th term but when it does it is called a
  progression.

  \textbf{Arithmetic Progression:} It is a progression where consecutive terms
  differ by a constant known as common difference.\\
  Examples: \\$1 + 2 + 3 + 4 + \ldots + 10$\linebreak
  $20 + 18 + 16 + 14 \ldots + 2$
\end{frame}

\begin{frame}{$n$th term of an Arithmetic Progression}
  Let $a$ be the first term and $d$ be the common difference of an A. P., then

  \vspace*{.5cm}
  \begin{tabular}{ll}
  First term &$t_1 = a = a + (1 - 1)d$\\
  Second term &$t_2 = a + d = a + (2 - 1)d$\\
  Third term &$t_3 = a + 2d = a + (3 - 1)d$\\
  ...............................&\\
  $n$th term &$t_n = a + (n - 1)d$
  \end{tabular}
\end{frame}

\begin{frame}{To find the sum of $n$ terms of an A.P.}
  Let $a$ be the first term, $d$ the c.d., $t_n$ the $n$th term and $S_n$ the
  sum of $n$ terms of an A. P.

  $$S_n = a + (a + d) + (a + 2d) + \ldots + (t_n - 2d) + (t_n - d) + t_n$$
  $$S_n = t_n + (t_n - d) + (t_n - 2d)+ \ldots + (a - 2d) + (a - d) + a$$

  Adding these two we get

  $$2S_n = (a + t_n) + (a + t_n) + (a + t_n) + \ldots + (a + t_n) + (a + t_n) +
  (a + t_n)$$

  $$= (a + t_n) + (a + t_n) + \ldots \text{to}~n~\text{terms}$$
  $$= n(a + t_n)$$
  $$\therefore S_n = \frac{n}{2}(a + t_n)$$
  $$= \frac{n}{2}[a + a + (n - 1)d]$$
  $$\therefore S_n = \frac{n}{2}[2a + (n - 1)d]$$
\end{frame}

\begin{frame}{Properties of an A.P.}
  If the same quantity is added to or subtracted from all the terms of an A.P.,
  the resulting progression is also an arithmetic progression.

  \textbf{Proof:}\\
  \begin{tabular}{lll}
    Given A. P. & Sequence after adding $k$ to each &
    Sequence after subtracting $k$ from each \\
    &term of given A.P.&term of given A.P.\\
    $t_1 = a$ & $t_1 = a + k$ & $t_1 = a - k$\\
    $t_2 = a + d$ & $t_1 = a + d + k$ & $t_1 = a + d - k$\\
    $t_3 = a + 3d$ & $t_1 = a + 2d + k$ & $t_1 = a + 2d - k$\\
    \ldots&\ldots & \ldots\\
    $t_n = a + (n - 1)d$ & $t_n = a + (n - d) - k$ & $t_n = a + (n - 1)d - k$
  \end{tabular}
  
  Clearly, each term changes by $k$ but the common difference remains same
  making resulting series arithmetic progression as well.
\end{frame}
\begin{frame}{Properties of an A.P.}
  If the corresponsing terms of two arithmetic progressions be added or
  subtracted, the resulting progression is also an arithmetic progression.

  \textbf{Proof:}\\
  \begin{tabular}{lll}
    Terms of first A. P.&Terms of second A.P.&Terms of A.P. after addition\\
    $a_1$&$a_2$&$a_1 + a_2$\\
    $a_1 + d_1$&$a_2 + d_2$&$a_1 + a_2 + d_1 + d_2$\\
    $a_1 + 2d_1$&$a_2 + 2d_2$&$a_1 + a_2 + 2d_1 + 2d_2$\\
    \ldots & \ldots & \ldots
  \end{tabular}

    Clearly, addition results in a new A.P. with first terms as sum of first
    terms and common difference as sum of common differences.
\end{frame}
\begin{frame}{Properties of an A.P.}
  If all the terms of an A.P. are multiplies or divided by some constant then
  the resulting progression is also an A. P.

  \textbf{Proof:}\\
  \begin{tabular}{lll}
    Original A.P.&A.P. after multuplication& A.P. after division\\
    $a$& $ak$ & $\frac{a}{k}$\\
    $a + d$ & $(a + d)k$ & $\frac{a + d}{k}$\\
    $a + 2d$ & $(a + 2d)k$ & $\frac{a + 2d}{k}$\\
    \ldots & \ldots & \ldots
  \end{tabular}

  Clearly, in case of multiplication new first term is $ak$ and common
  difference is $dk$ while in case of division new first term is $\frac{a}{k}$
  and common difference is $\frac{d}{k}$
\end{frame}
\end{document}
