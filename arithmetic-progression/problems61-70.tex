\documentclass[aspectratio=1610,8pt]{beamer}

% Standard packages

\usepackage[english]{babel}
\usepackage{siunitx}
\usepackage{textcomp}
%\usepackage[latin1]{inputenc}
%\usepackage{times}
%\usepackage[T1]{fontenc}


% Setup TikZ

\usepackage{tikz}
\usetikzlibrary{arrows}
\tikzstyle{block}=[draw opacity=0.7,line width=1.4cm]


% Author, Title, etc.

\title{Arithmetic Progression\\Problems 61 to 70}

\author[Shiv Shankar Dayal]{Shiv Shankar Dayal}

% The main document

\begin{document}
\begin{frame}
       \titlepage
\end{frame}
\begin{frame}{Problem 61}
  \textbf{61.} $25$ trees are planted in a straight line at intervals of $5$
  meters. To water them the gardener must bring water for each tree
  separately from a well $10$ meters from the first tree. How far he will have
  to travel to water all the trees beginning with the first if he starts from
  the well.
\end{frame}
\begin{frame}{Solution of problem 61}
  \textbf{Solution} Since the gardener starts from the well.\\
  $$\therefore \text{Distance covered to water first tree} = 10m$$
  $$\text{Distance covered to water second tree} = 10 + 15 = 25m$$
  $$\text{Distance covered to water third tree} = 15 + 20 = 35m$$
  $$\text{Distance covered to water third tree} = 20 + 25 = 45m$$
  $\therefore$ Total distance covered to water all trees
  $$= 10 + 25 + 35 + 45 + \ldots \text{to}~25~\text{terms}$$
  $$= 10 + (25 + 35 + 45 + \ldots \text{to}~24~\text{terms})$$
  $$= 10 + \frac{24}{2}[2.25 + (24 - 1)10]$$
  $$= 10 + 12.280 = 10 + 3360 = 3370m$$
\end{frame}
\begin{frame}{Problem 62}
  \textbf{62.} If $a$ be the first term of an A.P. and the sum of its first $p$
  terms is equal to zero, show that the sum of the next $q$ terms is
  $-\frac{a(p + q)}{p - 1}.q$
\end{frame}
\begin{frame}{Solution of problem 62}
  \textbf{Solution:} Let $d$ be the common difference and $S_p$ be the sum of
  first $p$ terms. Given,
  $$S_p = \frac{p}{2}[2a + (p - 1)d] = 0$$
  $$\because p \ne 0,~\therefore 2a + (p - 1)d = 0$$
  $$\therefore d = -\frac{2a}{p - 1}$$
  Now,
  $$\text{Sum of next}~q~\text{terms}~=~\text{Sum of first}~(p +
  q)~\text{terms} -~\text{Sum of first}~p~\text{terms}$$
  $$= S_{p + q} - S_p = S_{p + q} - 0$$
  $$= \frac{p+q}{2}[2a + (p + q - 1)d]$$
  $$= \frac{p + q}{2}\left[2a + (p+q-1).\left(-\frac{2a}{p-1}\right)\right]$$
  $$= \frac{p + q}{2}2a\left(1 - \frac{p+1-1}{p -1}\right)$$
  $$= -\frac{a(p+q)}{p -1}.q$$
\end{frame}
\begin{frame}{Problem 63}
  \textbf{63.} The sum of the first $p$ terms of an A.P. is equal to the
  sum of its first $q$ terms, prove that the sum of its first $(p + q)$ terms
  is zero.
\end{frame}
\begin{frame}{Solution of problem 63}
  \textbf{Solution:} Let the first term be $a$ and common difference be $d$. Now
  given,
  $$S_p = S_q$$
  $$\frac{p}{2}[2a + (p - 1)d] = \frac{q}{2}[2a + (q - 1)d]$$
  $$2ap + p(p - 1)d = 2aq + q(q - 1)d$$
  $$2a(p - q) + (p^2 - p -q^2 + q)d = 0$$
  $$2a(p - q) + [(p^2 - q^2) -(p - q)]d = 0$$
  $$2a + (p + q - 1)d = 0$$
  $$S_{p + q} = \frac{p +q}{2}[2a + (p + q - 1)d] = 0$$
\end{frame}
\begin{frame}{Problem 64}
  \textbf{64.} Prove that the sum of latter half of $2n$ terms of a series in
  A.P. is equal to the one third of the sum of first $3n$ terms.
\end{frame}
\begin{frame}{Solution of problem 64}
  \textbf{Solution:} Sum of latter half of $2n$ terms $= S_{2n} - S_n.$ Let $a$
  be the first term and $d$ be the common difference. Then,
  $$S_{2n} - S_n = \frac{2n}{2}[2a + (2n - 1)d] - \frac{n}{2}[2a + (n - 1)d]$$
  $$= \frac{n}{2}[4a +2(2n - 1)d - 2a -(n - 1)d]$$
  $$= \frac{n}{2}[2a + (3n - 1)d]$$
  $$=\frac{1}{3}\frac{3n}{2}[2a + (3n - 1)d]$$
  $$= \frac{1}{3}S_{3n}$$
\end{frame}
\begin{frame}{Problem 65}
  \textbf{65.} If $S_1, S_2, S_3, \ldots, S_p$ be the sum of $n$ terms of
  arithmetic progressions whose first terms are respectively $1, 2, 3, \ldots$
  and common differences are $1, 2, 3, \ldots$ prove that
  $$S_1 + S_2 + S_3 + \ldots + S_p = \frac{np}{4}(n+1)(p+1)$$
\end{frame}
\begin{frame}{Solution of problem 65}
  \textbf{Solution:}
  $$S_1 = \frac{n}{2}[2.1 + (n - 1).1] = \frac{n(n + 1)}{2}.1$$
  $$S_2 = \frac{n}{2}[2.2 + (n - 1).2] = \frac{n(n + 1)}{2}.2$$
  $$S_3 = \frac{n}{2}[2.3 + (n - 1).3] = \frac{n(n + 1)}{2}.3$$
  $$\ldots$$
  $$S_p = \frac{n}{2}[2.p + (n - 1).p] = \frac{n(n+1)}{2}.p$$
  $$S_1 + S_2 + S_3 + \ldots + S_p = \frac{n(n + 1)}{2}[1 + 2 + 3 + \ldots +
    p]$$
  $$S_1 + S_2 + S_3 + \ldots + S_p = \frac{n(n + 1)}{2}\frac{p(p + 1)}{2} =
  \frac{np}{4}(n + 1)(p + 1)$$
\end{frame}
\begin{frame}{Problem 66}
  \textbf{66.} If $a,b$ and $c$ be the sum of $p, q$ and $r$ terms rspectively
  of an A.P., prove that
  $$\frac{a}{p}(q - r) + \frac{b}{q}(r - p) + \frac{c}{r}(p - q) = 0$$
\end{frame}
\begin{frame}{Solution of problem 66}
  \textbf{Solution:} Let $x$ be the first term and $d$ be the common difference
  of the A.P. Given,
  \begin{equation}\label{eq:first}
    a = \frac{p}{2}[2x + (p - 1)d] \Rightarrow \frac{a}{p} = x + \frac{p
      -1}{2}d
  \end{equation}
  \begin{equation}\label{eq:second}
    b = \frac{q}{2}[2x + (q - 1)d] \Rightarrow \frac{b}{q} = x + \frac{q
      -1}{2}d
  \end{equation}
  \begin{equation}\label{eq:third}
    c = \frac{r}{2}[2x + (r - 1)d] \Rightarrow \frac{c}{r} = x + \frac{r -
      1}{2}d
  \end{equation}
  Multiplying (\ref{eq:first}) by $(q - r)$, (\ref{eq:second}) by $(r - p)$ and
  (\ref{eq:third}) by $(p - q)$ and adding,
  $$\frac{a}{p}(q - r) + \frac{b}{q}(r - p) + \frac{c}{r}(p - q) = \left(x +
  \frac{q - 1}{d}\right)(q - r) + \left(x + \frac{q - 1}{2}\right)(r - p) +
  \left(x + \frac{r - 1}{2}\right)(p - q)$$
  $$= x[q - r + r - p + p - q] + \frac{d}{2}[(q - r)(p - 1) + (r - p)(q - q) +
    (p - q)(r - 1)]$$
  $$= 0$$
\end{frame}
\begin{frame}{Problem 67}
  \textbf{67.} If the sum of $m$ terms of an A.P. is equal to half the sum of
  $(m + n)$ terms and is also equal to half the sum of $(m + p)$ terms, prove
  that $(m + n)\left(\frac{1}{m} - \frac{1}{p}\right) = (m +
  p)\left(\frac{1}{m} - \frac{1}{n}\right)$
\end{frame}
\begin{frame}{Solution of problem 67}
  \textbf{Solution:} Let $a$ be the first term and $d$ be the common difference
  of the A.P.\\
  According to question, $S_m = \frac{1}{2}S_{m + n}$
  \setcounter{equation}{0}
  \begin{equation}\label{eq:fourth}
    \frac{m}{2}[2a + (m - 1)d] = \frac{m + n}{2}[2a + (m + n - 1)d]
  \end{equation}
  Let $2a + (m - 1)d = x,$ then from (\ref{eq:fourth}) we have
  $$2mx = (m + n)(x + nd)\Rightarrow 2mx - mx - nx = (m + n)nd$$
  \begin{equation}\label{eq:fifth}
    x(m - n) = (m + n)nd
  \end{equation}
  Also, $$S_m = \frac{1}{2}S_{m + p}$$
  Replacing $n$ by $p$ in (\ref{eq:fifth})
  \begin{equation}\label{eq:sixth}
    x(m - p) = (m + p)pd
  \end{equation}
  Dividing (\ref{eq:fifth}) by (\ref{eq:sixth}), we get
  $$\frac{m - n}{m - p} = \frac{(m + n)n}{(m + p)p}$$
  $$(m + n)(m - p)n = (m + p)(m - n)p$$
  Dividing both sides by $mnp$
  $$(m + n)\left(\frac{1}{p} - \frac{1}{m} = (m + p)\left(\frac{1}{n} -
  \frac{1}{m}\right)\right)$$
  $$(m + n)\left(\frac{1}{m} - \frac{1}{p}\right) = (m +
  p)\left(\frac{1}{m} - \frac{1}{n}\right)$$
\end{frame}
\begin{frame}{Problem 68}
  \textbf{68.} If there are $(2n + 1)$ terms in an A.P., then prove that the
  ratio of sum of odd terms and the sum of even terms is $n + 1: n.$
\end{frame}
\begin{frame}{Solution of problem 68}
  \textbf{Solution:} Let the A.P. be $a, a+d, a+2d, \ldots, a + 2nd$\\
  Sum of its odd terms $= a + (a + 2d) + (a + 4d) + \ldots + (a + 2nd)$
  $$= \frac{n + 1}{2}[2a + (n + 1 - 1).2d] = (n + 1)(a + nd)$$
  Sum of even terms $= (a + d) + (a + 3d) + \ldots + (a + (2n-1)d)$
  $$= \frac{n}{2}[2(a + d) + (n - 1)d] = n(a + nd)$$
  $$\therefore \frac{\text{Sum of odd terms}}{\text{Sum of even terms}} =
  \frac{n + 1}{n}$$
\end{frame}
\begin{frame}{Problem 69}
  \textbf{69.} The sum of $n$ terms of two series in A.P. are in the ration
  $(3n - 13): (5n + 21).$ Find the ratio of their $24$th terms.
\end{frame}
\begin{frame}{Solution of problem 69}
  \textbf{Solution:} Let $a_1$ and $a_2$ be the first terms and $d_1$ and $d_2$
  be the common differences of the two series in A.P.\\
  Now, ratio of $24$th terms
  \setcounter{equation}{0}
  \begin{equation}\label{eq:seventh}
    = \frac{a_1 + 23d_1}{a_2 + 23d_2}
  \end{equation}
  Ratio of sum of $n$ terms
  $$\frac{\frac{n}{2}[2a_1 + (n - 1)d_1]}{\frac{n}{2}[2a_2 + (n - 1)d_2]} =
  \frac{3n - 13}{5n + 21}$$
  $$\frac{2a_1 + (n - 1)d_1}{2a_2 + (n - 1)d_2} = \frac{3n - 13}{5n + 21}$$
  Putting $n=47,$ we get
  $$\frac{2a_1 + 46d_1}{2a_2 + 46d_2} = \frac{3.47 - 13}{5.47 + 21}$$
  $$\frac{a_1 + 23d_1}{a_2 + 23d_2} = \frac{1}{2}$$
  Therefore, from (\ref{eq:seventh}) we get the ratio of $24$th terms as $\frac{1}{2}$
\end{frame}
\begin{frame}{Problem 70}
  \textbf{70.} If the $m$th term of an A.P. is $\frac{1}{n}$ and $n$th term of
  an A.P. is $\frac{1}{m}$ then prove that the sum to $mn$ terms is $\frac{mn +
  1}{2}$
\end{frame}
\begin{frame}{Solution of problem 70}
  \textbf{Solution:} Let $a$ be the first term and $d$ be the common difference
  of the A.P.\\
  \setcounter{equation}{0}
  \begin{equation}\label{eq:8th}
    t_m = a + (m - 1)d = \frac{1}{n}
  \end{equation}
  \begin{equation}\label{eq:9th}
    t_n = a + (n - 1)d = \frac{1}{m}
  \end{equation}
  Subtracting (\ref{eq:8th}) from (\ref{eq:9th}), we get
  $$(n - m)d = \frac{1}{m} - \frac{1}{n} = \frac{n - m}{mn}$$
  $$d = \frac{1}{mn}$$
  Substituting the value of $d$ in (\ref{eq:8th}), we get
  $$a + (m - 1)\frac{1}{mn} = \frac{1}{n}$$
  $$a = \frac{1}{n}(1 - \frac{m - 1}{m}) = \frac{1}{mn}$$
  $$S_{mn} = \frac{mn}{2}[2a + (mn - 1)d]$$
  Substituting the value of $d,$ we get
  $$= \frac{mn}{2}\left(\frac{2}{mn} + (mn - 1)\frac{1}{mn}\right)$$
  $$= \frac{mn + 1}{2}$$
\end{frame}
\end{document}
