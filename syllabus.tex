\documentclass[aspectratio=1610,8pt]{beamer}

% Standard packages

\usepackage[english]{babel}
%\usepackage[latin1]{inputenc}
\usepackage{times}
%\usepackage[T1]{fontenc}


% Setup TikZ

\usepackage{tikz}
\usetikzlibrary{arrows}
\tikzstyle{block}=[draw opacity=0.7,line width=1.4cm]


% Author, Title, etc.

\title{Class 12 Algebra}

\author[Shiv Shankar Dayal]
       {
	       Shiv Shankar Dayal
       }

       % The main document

\begin{document}
\begin{frame}
       \titlepage
\end{frame}

\section{Introduction}
\subsection{Introduction}

\begin{frame}{Introduction}
	These videos will try to cover few Class 12 Algebra course.

	Source of these slides can be found at
	https://github.com/shivshankardayal/Class-12-Algebra

	We will solve lots of problems and solutions will be
	presented as math cannot be learned without solving
	problems.
\end{frame}

\section{Class 12 Algebra}
\subsection{Syllabus}
\begin{frame}{Syllabus}
	\begin{itemize}
		\item Theory of Numbers
		\item General Theory of Continued Fractions
		\item Symmetric and Alternating Functions, Substitution
		\item Theory of Equations
		\item Complex Numbers
		\item Determinants and Matrices
		\item Arithmetic Progression
		\item Geometric Progression
		\item Harmonic Progressions
		\item Theorems About Progressions
		\item Surds and Imaginary Quantities
		\item Polynomials
		\item Quadratic Equations
	\end{itemize}
\end{frame}

\begin{frame}
	\begin{itemize}
		\item Miscellaneous Equations
		\item Mathematical Induction
		\item Binomial Theorem, Positive Integral Index
		\item Binomial Theorem, Any Index
		\item Multinomial Theorem
		\item Logarithm
		\item Exponential and Logarithmic Seroes
		\item Inequalities
		\item Limiting Values and Vanishing Fractions
		\item Convergency and Divergency of Series
		\item Undetermined Coefficients
		\item Partial Fractions
		\item Recurring Series
		\item Continued Fractions
	\end{itemize}
\end{frame}
\begin{frame}
	\begin{itemize}
		\item Indeterminate Equations of the First Degree
		\item Recurring Continued Fractions
		\item Indeterminate Equations of the Second Degree
		\item Summation of Series
		\item Infinite Products
		\item Permutation and Combination
		\item Probability
	\end{itemize}
\end{frame}

\end{document}


