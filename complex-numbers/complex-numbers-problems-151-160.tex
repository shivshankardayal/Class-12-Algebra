\documentclass[aspectratio=169,8pt]{beamer}

% Standard packages

\usepackage[english]{babel}
%\usepackage[latin1]{inputenc}
%\usepackage{times}
%\usepackage[T1]{fontenc}
\usepackage{fontspec}
\usepackage[]{unicode-math}
\setmathfont{Inconsolata}
\setsansfont{Roboto}

% Setup asymptote
\usepackage[inline]{asymptote}

\newcounter{counter}
% Author, Title, etc.

\title{Complex Numbers Problems\\ 151-160}

\author[Shiv Shankar Dayal]{Shiv Shankar Dayal}

\begin{document}
\begin{frame}
  \titlepage
\end{frame}
\begin{frame}{Problem 151}
  \textbf{151.} Consider an equilateral triangle $A\left(\frac{2}{\sqrt{3}}e^{i\pi/2}\right),
  B\left(\frac{2}{\sqrt{3}}e^{-i\pi/6}\right)$ and $C\left(\frac{2}{\sqrt{3}}e^{-i5\pi/6}\right)$. If $P(z)$ is any point on the
  incircle then find the value of $AP^2 + BP^2 + CP^2$.
\end{frame}
\begin{frame}{Solution of Problem 151}
  \textbf{Solution:} $A(z_1) = \frac{2i}{\sqrt{3}},B(z_2) = \frac{2}{\sqrt{3}}\left(\frac{\sqrt{3}}{2} - i\frac{1}{2}\right) = 1 -
  \frac{i}{\sqrt{3}}, C(z_3) = \frac{2}{\sqrt{3}}\left(-\frac{\sqrt{3}}{2} - \frac{i}{2}\right) = -1-\frac{i}{\sqrt{3}}$\\
  \vspace*{0.2cm}
  Clearly, the points lie on the circle $z=2/\sqrt{3}$ and $\triangle ABC$ is equilateral and its centroid coincides with
  circumcentre. Hence,\\
  \vspace*{0.2cm}
  $z_1 + z_2 + z_3 = 0$ and $\overline{z_1} + \overline{z_2} + \overline{z_3} = 0$
  Clearly, radius of incircle $= \frac{1}{\sqrt{3}}$ hence any point on circle is $\frac{1}{\sqrt{3}}(\cos\alpha + i\sin\alpha)$.\\
  \vspace*{0.2cm}
  $AP^2 = |z - z_1|^2 = |z|^2 + |z_1|^2 - (z\overline{z_1} + \overline{z}z_1)$
  \vspace*{0.2cm}
  $\Rightarrow AP^2 + BP^2 + CP^2 = 3|z|^2 + |z_1|^2 + |z_2|^2 + |z_3|^2 - z(\overline{z_1} + \overline{z_2} + \overline{z_3}) -
  \overline{z}(z_1 + z_2 + z_3)$\\
  \vspace*{0.2cm}
  $= 3\times\frac{1}{3} + \frac{4}{3} + \frac{4}{3} + \frac{4}{3} - 0 - 0 = 5$
\end{frame}
\begin{frame}{Problem 152}
  \textbf{152.} If $A_1, A_2, \ldots, A_n$ be the vertices of a regular polygon of $n$ sides in a circle of unit radius and $a =
  |A_1A_2|^2 + |A_1A_3|^2 + \ldots + |A_1A_n|^2, b = |A_1A_2||A_1A_3|\ldots |A_1A_n|$, then find $\frac{a}{b}$.
\end{frame}
\begin{frame}{Solution of Problem 152}
  \textbf{Solution:} Let $O$ be the center of the polygon and $z_0, z_1, \ldots, z_{n - 1}$ represent the vertices $A_1, A_2,
  \ldots, A_n$.\\
  \vspace*{0.2cm}
  $\therefore z_0 = 1, z_1 = \alpha, z_2 = \alpha^2, \ldots, z_{n - 1} = \alpha^{n - 1}$ where $\alpha = e^{i2\pi/n}$\\
  \vspace*{0.2cm}
  $|A_1A_2|^2 =|\alpha^r - 1|^2 = |1 - \alpha^r|^2 = \left|1 - \cos\frac{2r\pi}{n} + i\sin\frac{2r\pi}{n}\right|^2$\\
  \vspace*{0.2cm}
  $= \left(1 - \cos\frac{2r\pi}{n}\right)^2 + \sin^2\frac{2r\pi}{n} = 2 - 2\cos\frac{2r\pi}{n}$\\
  \vspace*{0.2cm}
  $\sum_{r=1}^n |A_1A_2|^2 = 2(n - 1) - 2\left[\cos\frac{2\pi}{n} + \cos\frac{4\pi}{3} + \ldots + \cos\frac{2(n -
      1)\pi}{n}\right]$\\
  \vspace*{0.2cm}
  $= 2(n - 1) -2.$ real part of $(\alpha + \alpha^2 + \ldots + \alpha^{n - 1}) = 2n [\because 1 + \alpha + \alpha^2 + \ldots +
    \alpha^{n - 1} = 0]$\\
  \vspace*{0.2cm}
  $|A_1A_2||A_1A_3|\ldots |A_1A_n| = |1 - \alpha||1 - \alpha^2|\ldots|1 - \alpha^{n - 1}|$\\
  \vspace*{0.2cm}
  $= |(1 - \alpha)(1 - \alpha^2)\ldots(1 - \alpha^{n - 1})|$\\
  Since $1, \alpha, \alpha^2, \ldots, \alpha^{n - 1}$ are roots of $z^n - 1 = 0$\\
  \vspace*{0.2cm}
  $(z - 1)(z - \alpha)(z - \alpha^2)\ldots(z - \alpha^{n - 1}) = z^n - 1$\\
  \vspace*{0.2cm}
  $\Rightarrow (z - \alpha)(z - \alpha^2)\ldots(z - \alpha^{n - 1}) = \frac{z^n - 1}{z - 1} = 1 + z + z^2 + \ldots + z^{n - 1}$\\
  \vspace*{0.2cm}
  Putting $z = 1$, we get\\
  \vspace*{0.2cm}
  $|(1 - \alpha)(1 - \alpha^2)\ldots(1 - \alpha^{n - 1})| = n \Rightarrow \frac{a}{b} = 2$
\end{frame}
\begin{frame}{Problem 153}
  \textbf{153.} If $\left(1 + i\frac{x}{a}\right)\left(1+ i\frac{x}{b}\right)\left(1 + i\frac{x}{c}\right)\ldots = A + iB$, then
  prove that $\left(1 + \frac{x^2}{a^2}\right)\left(1 + \frac{x^2}{b^2}\right)\left(1 + \frac{x^2}{c^2}\right)\ldots = A^2 + B^2$.
\end{frame}
\begin{frame}{Solution of Problem 153}
  \textbf{Solution:} Let $L.H.S. = z_1$ and $R.H.S. = z_2$ then $\overline{z_1} = \overline{z_2}$\\
  \vspace*{0.2cm}
  $\Rightarrow z_1\overline{z_1} = z_2\overline{z_2} \Rightarrow z_1^2 = z_2^2$\\
  \vspace*{0.2cm}
  $\Rightarrow \left(1 + \frac{x^2}{a^2}\right)\left(1 + \frac{x^2}{b^2}\right)\left(1 + \frac{x^2}{c^2}\right)\ldots = A^2 + B^2$.
\end{frame}
\begin{frame}{Problem 154}
  \textbf{154.} Find the range of real number $\alpha$ for which the equations $z+\alpha|z−1|+2i=0;z=x+iy$ has a solution. Also,
  find the solution.
\end{frame}
\begin{frame}{Solution of Problem 154}
  \textbf{Solution:} Given, $x + iy + \alpha\sqrt{(x - 1)^2 + y^2} + 2i = 0$\\
  \vspace*{0.2cm}
  Equating real and imaginary parts, we get\\
  \vspace*{0.2cm}
  $y + 2 = 0 \Rightarrow y = -2$ and $x + \alpha\sqrt{(x - 1)^2 + y^2} = 0$\\
  \vspace*{0.2cm}
  Substituting the value of $y$, we get
  $\alpha\sqrt{x^2 - 2x + 5} = -x \Rightarrow (\alpha^2 - 1)x^2 - 2\alpha^2x +5\alpha^2 = 0$\\
  \vspace*{0.2cm}
  Because $x$ is real, the discriminant has to be greater than zero.
  \vspace*{0.2cm}
  $\Rightarrow 4\alpha^4 - 20\alpha^2(\alpha^2 - 1) \geq 0$\\
  \vspace*{0.2cm}
  $\Rightarrow \alpha^2 - 5\alpha^2 + 5 \geq 0 \Rightarrow -\frac{\sqrt{5}}{2}\leq\alpha\leq \frac{\sqrt{5}}{2}$
\end{frame}
\begin{frame}{Problem 155}
  \textbf{155.} For every real number $a\geq0$, find all the complex numbers satisfying the equation $2|z|−4az+1+ia=0$.
\end{frame}
\begin{frame}{Solution of Problem 155}
  \textbf{Solution:} Let $z = x + iy \Rightarrow 2\sqrt{x^2 + y^2} - 4a(x + iy) + 1 + ia = 0$\\
  \vspace*{0.2cm}
  Equating real and imaginary parts, we get\\
  \vspace*{0.2cm}
  $2\sqrt{x^2 + y^2} - 4ax + 1 = 0$ and $-4ay + a = 0 \Rightarrow y = \frac{1}{4}$\\
  \vspace*{0.2cm}
  $2\sqrt{x^2 + \frac{1}{16}} - 4ax + 1 = 0 \Rightarrow 4\left(x^2 + \frac{1}{16}\right) = 16a^2x^2 - 8ax + 1$\\
  \vspace*{0.2cm}
  $x^2(4 - 16a^2) + 8ax - \frac{3}{4} = 0 \Rightarrow x = \frac{-a}{1 - 4a^2} \pm \frac{1}{4}\frac{\sqrt{4a^2 + 3}}{1 - 4a^2}$
\end{frame}
\begin{frame}{Problem 156}
  \textbf{156.} Show that $(x^2 + y^2)^5 = (x^5 - 10x^3y^2 + 5xy^4) + (5x^4y - 10x^2y^3 + y^5)^2$
\end{frame}
\begin{frame}{Solution of Problem 156}
  \textbf{Solution:} $(x + iy)^5 = (x^5 - 10x^3y^2 + 5xy^4) + i(5x^4y - 10x^2y^3 + y^5)$\\
  \vspace*{0.2cm}
  Taking modulus and squaring, we get\\
  \vspace*{0.2cm}
  $(x^2 + y^2)^5 = (x^5 - 10x^3y^2 + 5xy^4) + (5x^4y - 10x^2y^3 + y^5)^2$
\end{frame}
\begin{frame}{Problem 157}
  \textbf{157.} Express $(x^2+a^2)(x^2+b^2)(x^2+c^2)$ as sum of two squares.
\end{frame}
\begin{frame}{Solution of Problem 157}
  \textbf{Solution:} $(x + ia)(x + ib)(x + ic) = [(x^2 - ab) + i(a + b)x](x + ic) = (x^3 - abx - acx - bcx) + i(cx^2 - abc + ax^2 +
  bx^2)$\\
  \vspace*{0.2cm}
  Taking modulus and squaring, we get\\
  \vspace*{0.2cm}
  $(x^2 + a^2)(x^2 + b^2)(x^2 + c^2) = [x^3 -(ab + bc + ca)x] + [(a + b + c)x^2 - abc]^2$
\end{frame}
\begin{frame}{Problem 158}
  \textbf{158.} If $(1 + x)^n = a_0 + a_1x + a_2x^2 + \ldots + a_nx^n$, then prove that $2^n = (a_0 - a_2 + a_4 - \ldots)^2 + (a_1
  - a_3 + a_5 - \ldots)^2$.
\end{frame}
\begin{frame}{Solution of Problem 158}
  \textbf{Solution:} Given, $(1 + x)^n = a_0 + a_1x + a_2x^2 + \ldots + a_nx^n$\\
  \vspace*{0.2cm}
  Substituting $x = i$,we get\\
  \vspace*{0.2cm}
  $(1 + i)^n = a_o + ia_1 - a_2 - ia_3 + a_4 + \ldots = (a_0 - a_2 + a_4 - \ldots) + i(a_1 - a_3 + a_5 - \ldots)$\\
  \vspace*{0.2cm}
  Taking modulus and squaring, we get\\
  \vspace*{0.2cm}
  $2^n = (a_0 - a_2 + a_4 - \ldots)^2 + (a_1 - a_3 + a_5 - \ldots)^2$
\end{frame}
\begin{frame}{Problem 159}
  \textbf{159.} Dividing $f(z)$ by $z−i$, we get $i$ as remainder and if we divide by $z+i$, we get $1+i$ as remainder. Find the
  remainder upon division of $f(z)$ by $z^2+1$.
\end{frame}
\begin{frame}{Solution of Problem 159}
  \textbf{Solution:} Let $f(z) = m(z - i) + i$ and $f(z) = n(z + i) + 1 + i$ where $m$ and $n$ are quotients upon division.\\
  \vspace*{0.2cm}
  Substituting $z = i$ in the first equation and $z = -i$ in the second we obtain $f(i) = i$ and $f(-i) = 1 + i$.\\
  \vspace*{0.2cm}
  Let $g(z)$ be the quotient and $az + b$ be the remainder upong division of $f(z)$ by $z^2 + 1.$ Hence we have\\
  \vspace*{0.2cm}
  $f(z) = g(z)(z^2 + 1) + az + b.$ Substituting $z = i$ and $z = -i,$ we get\\
  \vspace*{0.2cm}
  $f(i) = i = ai + b$ and $f(-i) = 1 + i = -ai + b$\\
  \vspace*{0.2cm}
  Adding, we get $2b = 1 + 2i \Rightarrow b = \frac{1 + 2i}{2} \Rightarrow ai = i - \frac{1 + 2i}{2}$
\end{frame}
\begin{frame}{Problem 160}
  \textbf{160.} If $|z|\leq 1, |w|\leq 1,$ show that $|z - w|^2 \leq (|z| - |w|)^2 + [\arg(z) - \arg(w)]^2$.
\end{frame}
\begin{frame}{Solution of Problem 160}
  \textbf{Solution:} Let $z = r_1e^{i\theta_1}, w = r_2e^{i\theta_2}$. $\because |z|\leq 1$ and $|w|\leq 1 \Rightarrow r_1\leq 1$
  and $r_2\leq 1$\\
  \vspace*{0.2cm}
  $|z - w|^2 = (r_1\cos\theta_1 - r_2\cos\theta_2)^2 + (r_1\sin\theta_1 - r_2\sin\theta_2)^2$\\
  \vspace*{0.2cm}
  $= r_1^2 + r_2^2 - 2r_2r_2\cos(\theta_1 - \theta_2) = (r_1 - r_2)^2 + 2r_2r_2 - 2r_2r_2\cos(\theta_1 - \theta_2)$\\
  \vspace*{0.2cm}
  $= (r_1 - r_2)^2 + 4r_1r_2\sin\left(\frac{\theta_1 - \theta_2}{2}\right)^2\leq (r_1 - r_2)^2 + (\theta_1 - \theta_2)^2[\because
    r_1, r_2\leq 1$ and $\sin\theta \leq \theta]$\\
    \vspace*{0.2cm}
    $= (|z| - |w|)^2 + [\arg(z) - \arg(w)]^2$.
\end{frame}
\end{document}
