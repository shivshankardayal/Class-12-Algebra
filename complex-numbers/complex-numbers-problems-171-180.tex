\documentclass[aspectratio=169,8pt]{beamer}

% Standard packages

\usepackage[english]{babel}
%\usepackage[latin1]{inputenc}
%\usepackage{times}
%\usepackage[T1]{fontenc}
\usepackage{fontspec}
\usepackage[]{unicode-math}
\setsansfont{Roboto}

% Setup asymptote
\usepackage[inline]{asymptote}

\newcounter{counter}
% Author, Title, etc.

\title{Complex Numbers Problems\\ 171-180}

\author[Shiv Shankar Dayal]{Shiv Shankar Dayal}

\begin{document}
\begin{frame}
  \titlepage
\end{frame}
\begin{frame}{Problem 171}
  \textbf{171.} If $n > 1$, show that the roots of the equation $z^n = (1 + z)^n$ are collinear.
\end{frame}
\begin{frame}{Solution of Problem 171}
  \textbf{Solution:} Given $z^n = (z + 1)^n \Rightarrow |z|^n = |z + 1|^n$\\
  \vspace*{0.2cm}
  $\Rightarrow |z| = |z + 1|\Rightarrow x^2 = (x^2 + 2x + 1) \Rightarrow 2x + 1 = 0$\\
  \vspace*{0.2cm}
  which is the equation of a straight line on which roots of the given equation will lie.
\end{frame}
\begin{frame}{Problem 172}
  \textbf{172.} If $A,B,C$ and $D$ are four complex number then show that $AD.BC\leq BD.CA + CD.AB$.
\end{frame}
\begin{frame}{Solution of Problem 172}
  \textbf{Solution:} Let $z_1, z_2, z_3, z_4$ be represented by the points $A, B, C, D$ respectively.\\
  \vspace*{0.2cm}
  $\therefore AD = |z_1 - z_4|$ and $BC = |z_2 - z_3|$\\
  \vspace*{0.2cm}
  Let $a = (z_1 - z_4)(z_2 - z_3), b = (z_2 - z_4)(z_3 - z_1)$ and $c = (z_3 - z_4)(z_1 - z_2)$\\
  \vspace*{0.2cm}
  $b + c = (z_2 - z_4)(z_3 - z_1) + (z_3 - z_4)(z_1 - z_2) = -(z_1 - z_4)(z_2 - z_3) = -a$\\
  \vspace*{0.2cm}
  $|a| = |b + c| \leq |b| + |c| \Rightarrow |-(z_1 - z_4)(z_2 - z_3)| = |(z_2 - z_4)(z_3 - z_1)| + |(z_3 - z_4)(z_1 - z_2)|$\\
  \vspace*{0.2cm}
  $\Rightarrow AD.BC\leq BD.CA + CD.AB$.
\end{frame}
\begin{frame}{Problem 173}
  \textbf{173.} If $a, b\in R$ and $a, b\neq 0$, then show that the equation of line joining $a$ and $ib$ is $\left(\frac{1}{2a} -
  \frac{i}{2b}\right)z + \left(\frac{1}{2a} + \frac{i}{2b}\right)\overline{z} = 1$.
\end{frame}
\begin{frame}{Solution of Problem 173}
  \textbf{Solution:} Euqation of a line joining points $a$ and $ib$ is\\
  \vspace*{0.2cm}
  $\begin{vmatrix}z & \overline{z} & 1\\ a & \overline{a} & 1 \\ ib & =i\overline{b} & 1\end{vmatrix} = 0$ or $(\overline{a} +
    i\overline{b}) - (a - ib)\overline{z} - i(a\overline{b} + \overline{a}b) = 0$\\
    \vspace*{0.2cm}
    $\Rightarrow (a + ib)z - (a -ib)\overline{z} - 2abi = 0[\because a, b\in R \therefore a = \overline{a}, b = \overline{b}]$\\
    \vspace*{0.2cm}
    $\Rightarrow (a + ib)z - (a -ib)\overline{z} = 2abi$\\
    \vspace*{0.2cm}
    $\Rightarrow \left(\frac{1}{2a} - \frac{i}{2b}\right)z + \left(\frac{1}{2a} + \frac{i}{2b}\right)\overline{z} = 1$
\end{frame}
\begin{frame}{Problem 174}
  \textbf{174.} If $z_1$ and $z_2$ are two compelx numbers such that $|z_1| - |z_2| = |z_1 - z_2|$, then show that $\arg(z_1) -
  \arg(z_2) = 2n\pi$ where $n\in I$.
\end{frame}
\begin{frame}{Solution of Problem 174}
  \textbf{Solution:} Let $z_1 = r_1e^{i\theta_1}$ and $z_2 = r_2e^{i\theta_2}$\\
  \vspace*{0.2cm}
  Then $r_1 - r_2 = \sqrt{(r_1\cos\theta_1 - r_2\cos\theta_2)^2 + (r_1\sin\theta_1 - r_2\sin\theta_2)^2}$\\
  \vspace*{0.2cm}
  $\Rightarrow 2r_1r_2 = 2r_1r_2\cos(\theta_1 - \theta_2)\Rightarrow \cos(\theta_1- \theta_2) = \cos 2n\pi$\\
  \vspace*{0.2cm}
  $\Rightarrow \arg(z_1) - \arg(z_2) = 2n\pi$
\end{frame}
\begin{frame}{Problem 175}
  \textbf{175.} Let $A, B, C, D, E$ be points in the complex plane representing complex numbers $z_1, z_2, z_3, z_4, z_5$
  respectvely. If $(z_3 - z_2)z_4 = (z_1 - z_2)z_5,$ prove that $\triangle ABC$ and $\triangle DOE$ are similar.
\end{frame}
\begin{frame}{Solution of Problem 175}
  \textbf{Solution:} $\triangle ABC$ and $\triangle DOE$ will be similar if\\
  \vspace*{0.2cm}
  $\frac{AC}{AB} = \frac{DE}{DO}$ and $\angle BAC = \angle ODE$\\
  \vspace*{0.2cm}
  $\Rightarrow \left|\frac{z_3 - z_1}{z_2 - z_1}\right| = \left|\frac{z_5 - z_4}{0 - z_4}\right|$ and $\arg\left(\frac{z_3 - z_1}{z_2
    - z_1}\right) = \arg\left(\frac{z_5 - z_4}{0 - z_4}\right)$\\
  \vspace*{0.2cm}
  $\Rightarrow \frac{z_3 - z_1}{z_2 - z_1} = \frac{z_5 - z_4}{0 - z_4}$\\
  \vspace*{0.2cm}
  Solving this yields $(z_3 - z_2)z_4 = (z_1 - z_2)z_5$ and hence triangles are similar.
\end{frame}
\begin{frame}{Problem 176}
  \textbf{176.} Let $z$ and $z_0$ are two complex numbers and $z, z_0, z\overline{z_0}, 1$ are represented by points $P, P_0, Q, A$
  respectively. If $|z| = 1$, show that the triangle $POP_0$ and $AOQ$ are congruent and hence $|z - z_0| = |z\overline{z_0} - 1|$,
  where $O$ represents the origin.
\end{frame}
\begin{frame}{Solution of Problem 176}
  \textbf{Solution:} Given $OA = 1$ and $|z| = 1 = OP \Rightarrow OA = OP$\\
  \vspace*{0.2cm}
  $OP_0 = |z_0|$ and $OQ = |z\overline{z_0}| = |z||\overline{z_0}| = |z_0|$\\
  \vspace*{0.2cm}
  $\Rightarrow OP_0 = OQ$. Also give that $\angle P_0OP = \arg\frac{z_0}{z}$\\
  \vspace*{0.2cm}
  $\angle AOQ = \arg\left(\frac{1}{z\overline{z_0}}\right) = \arg\left(\frac{\overline{z}}{\overline{z_0}}\right)[\because
    z\overline{z} = 1]$\\
  \vspace*{0.2cm}
  $= -\arg\left(\frac{\overline{z_0}}{\overline{z}}\right)= -\arg\overline{\left(\frac{z_0}{z}\right)} =
  \arg\left(\frac{z_0}{z}\right) = \angle P_0OP$ and thus the triangles are congruent.
\end{frame}
\begin{frame}{Problem 177}
  \textbf{177.} If the line segment joining $z_1$ and $z_2$ is divided by $P$ and $Q$ in the ratio $a:n$ internally and externally,
  then find $OP^2 + OQ^2$ where $O$ is origin.
\end{frame}
\begin{frame}{Solution of Problme 177}
  \textbf{Solution:} $P = \frac{az_2 + bz_1}{a + b}, Q = \frac{az_2 - bz_1}{a - b}$\\
  \vspace*{0.2cm}
  $OP^2 = \left|\frac{az_2 + bz_1}{a + b}\right|^2 = \left(\frac{az_2 + bz_1}{a + b}\right)\left(\frac{a\overline{z_2} +
    b\overline{z_1}}{a + b}\right)$\\
  \vspace*{0.2cm}
  $= \frac{1}{a^2 + b^2}[a^2|z_2|^2 + b^2|z_1|^2 + ab(z_1\overline{z_2} + \overline{z_1}z_2)]$\\
  \vspace*{0.2cm}
  Similalry $OQ^2$ can be computed and sum found.
\end{frame}
\begin{frame}{Problem 178}
  \textbf{178.} Let $z_1, z_2, z_3$ be three complex numbers and $a, b, c$ be real numbers not all zero such that $a + b + c = 0$
  and $az_1 + bz_2 + cz_3 = 0$, then show that $z_1, z_2, z_3$ are collinear.
\end{frame}
\begin{frame}{Solution of Problem 178}
  \textbf{Solution:} Let $c\neq 0$, then $c = -(a + b)$ so we can write\\
  \vspace*{0.2cm}
  $az_1 + bz_2 - (a + b)z_3 = 0 \Rightarrow z_3 = \frac{az_1 + bz_2}{a + b}$\\
  \vspace*{0.2cm}
  Thus, we see that $z_3$ divides line segment $z_1z_2$ in the ratio of $a:b$ making all three of them collinear.
\end{frame}
\begin{frame}{Problem 179}
  \textbf{179.} If $z_1 + z_2 + \ldots + z_n = 0$, prove that if a line passes through origin then all these do not lie of the same
  side of the line provided they do not lie on the line.
\end{frame}
\begin{frame}{Solution of Problem 179}
  \textbf{Solution:} Equation of a line passing through origin is $a\overline{z} + \overline{a}z = 0$. Let us assume that all the
  points lie on the same side of the above line, so we have\\
  \vspace*{0.2cm}
  $a\overline{z_i} + \overline{a}z_i > 0$ or $< 0$ for $i = 1, 2, 3, \ldots, n$\\
  \vspace*{0.2cm}
  Thus, $a\sum_{i = 1}^n\overline{z_i} + \overline{a}\sum_{i = 1}^nz_i > 0$ or $< 0$\\
  \vspace*{0.2cm}
  But it is given that $\sum_{i = 1}^n z_i = 0 \Rightarrow \sum_{i = 1}^n \overline{z_i} = 0$\\
      \vspace*{0.2cm}
      $\therefore a\sum_{i =1}^n\overline{z_i} + \overline{a}\sum_{i = 1}^nz_i = 0$\\
      \vspace*{0.2cm}
      which in contradiction with equation above. So all points cannot lie on the same side of line.
\end{frame}
\begin{frame}{Problem 180}
  \textbf{180.} The points $z_1 = 9 + 12i$ and $z_2 = 6 - 8i$ are given on a complex plane. Find the equation of the angle formed
  by the vector representing $z_1$ and $z_2$.
\end{frame}
\begin{frame}{Solution of Problem 180}
  \textbf{Solution:} Let $OA$ and $OB$ be the unit vectors representing $z_1$ and $z_2$, then we have\\
  \vspace*{0.2cm}
  $\vec{OA} = \frac{z_1}{|z_1|}, \vec{OB} = \frac{z_2}{|z_2|}$\\
  \vspace*{0.2cm}
  Therefore equation of bisector will be $z = t\left(\frac{z_1}{|z_1|} + \frac{z_2}{|z_2|}\right) = \frac{6}{5}t,$ where is an
  arbitrary positive integer.
\end{frame}
\end{document}
