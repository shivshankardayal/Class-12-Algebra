\documentclass[aspectratio=169,8pt]{beamer}

% Standard packages

\usepackage[english]{babel}
%\usepackage[latin1]{inputenc}
%\usepackage{times}
%\usepackage[T1]{fontenc}
\usepackage{fontspec}
\usepackage[]{unicode-math}
\setmathfont{Inconsolata}
\setsansfont{Roboto}

% Setup TikZ

\usepackage{tikz}
\usetikzlibrary{arrows}
\tikzstyle{block}=[draw opacity=0.7,line width=1.4cm]

\newcounter{counter}
% Author, Title, etc.

\title{Complex Numbers}

\author[Shiv Shankar Dayal]{Shiv Shankar Dayal}

\begin{document}
\begin{frame}
  \titlepage
\end{frame}
\begin{frame}{Theory contd}
  A complex number $z$ which we have considered to be equal to $x + iy$ can be
  represented by a point $P$ whose caretesian coordinates are $(x, y)$ referred
  to rectangular axes $Ox$ and $Oy$ where $O$ is origin i.e. $(0, 0)$ and are
  called \textit{real} and \textit{imaginary} axis respectively. The $xy$ two
  dimensional plane is also called \textit{Argand plane, complex plane or
    Gaussian plane}. The point $P$ is also called the \textit{image} of the
  complex number and $z$ is also called the \textit{affix} or \textit{complex
    coordinate} of point $P.$

  The modulus is given by the length of segment $OP$ which is equal to $OP =
  \sqrt{x^2 + y^2} = |z|.$ Thus, $|z|$ is the length of $OP$.
\end{frame}
\begin{frame}
  \begin{center}
    \begin{tikzpicture}
      \draw[->] (-.5,0) -- (3,0);
      \draw[->] (0,-.5) -- (0,3);
      \draw (0, 3.5) node {$Y$};
      \draw (3.5, 0) node {$X$};
      \draw (2.5,0) -- (2.5,2.5);
      \draw (0,0) -- (2.5, 2.5);
      \draw (.5,0) arc(0:45:.5);
      \draw (.7,.3) node{$\theta$};
      \draw (1.5, 0.2) node {$x$};
      \draw (2.7, 1.5) node {$y$};
      \draw (3.4, 1.1) node {$OP=|z|$};
      \draw (3.5, 0.7) node {$arg(z)=\theta$};
      \draw (2.5, 2.7) node{$P=x+iy$};
      \node [label = below left:{$O$}] (o) at (0, 0) {};
      \draw (o);
    \end{tikzpicture}
  \end{center}
  In the diagram $\theta$ is known as the argument of $z.$ it is the angle made
  with positive direction(i.e. counter-clockwise) of real axis. This arhument
  is not unique. If $\theta$ is an argument of a complex number $z$ then $2n\pi
  + \theta$ where $n\in I$ where $I$ is the set of integers will be arguments
  as well. The value of argument for which $-\pi<\theta\leq \pi$ is called the
  \textit{principal argument.}
\end{frame}
\begin{frame}{Different Arguments of a Complex Number}
  In the digram given in previous slide the argument is given as
  $$arg(z) = \tan^{-1}\left(\frac{y}{x}\right)$$
  this value is for when $z$ is in first quadrant. When $z$ will lie in second,
  third and fourth quadrants then arguments will be
  $$arg(z) = \pi - \tan^{-1}\left(\frac{y}{|z|}\right), arg(z) = -\pi +
  \tan^{-1}\left(\frac{|y|}{|x|}\right), arg(z) =
  -\tan^{-1}\left(\frac{|y|}{x}\right)$$
  \vspace*{0.2cm}
  \textbf{\large{Polar Form of a Complex Number}}\\
  \vspace*{0.2cm}
  If $z$ is a non-zero complex number, then we can write $z = r(\cos\theta +
  i\sin\theta)$ where $r = |z|$ and $\theta = arg(z)$\\
  \vspace*{0.2cm}
  In this case $z$ is also given by $z = r[\cos(2n\pi + \theta) + i\sin(2n\pi +
    \theta)]$ where $n\in I.$\\
  \vspace*{0.2cm}
  \textbf{\large{Euler's Formula}}\\
  The complex number $\cos\theta + i\sin\theta$ is denoted by $e^{i\theta}$.
\end{frame}
\begin{frame}{Properties of Arguments}
  If $z, z_1$ and $z_2$ are complex numbers then
  \begin{enumerate}
    \item $arg(\overline{z}) = -arg(z)$. This can be easily proven as $z = x +
      iy$ and $\overline{z} = x - iy$ so sign of argument will get a -ve sign
      as $y$ gets one.
    \item $arg(z_1z_2) = arg(z_1) + arg(z_2) + 2k\pi$ where
      $$k = \begin{cases}0 & -\pi <arg(z_1) + arg(z_2) \leq \pi\\
      1 & -2\pi < arg(z_1) + arg(z_2)\leq -\pi\\
      -1 & -\pi < arg(z_1) + arg(z_2)\leq 2\pi\end{cases}$$
    \item $arg(z_1\overline{z_2}) = arg(z_1) - arg(z_2)$
    \item $\arg\left(\frac{z_1}{z_2}\right) = arg(z_1) + arg(z_2) + 2k\pi$
      where $k$ is same as item 2 with $+$ sign between $z_1$ and $z_2$ are
      replaced with $-$ sign.
    \item $|z_1 + z_2| = |z_1 - z_2|\Leftrightarrow arg(z_1) - arg(z_2) =
      \pi/2$
    \item $|z_1 + z_2| = |z_1| + |z_2|\Leftrightarrow arg(z_1) = arg(z_2)$
    \item $|z_1 + z_2|^2 = r_1^2 + r_2^2 + 2r_1r_2\cos(\theta_1 - \theta_2)$
    \item $|z_1 - z_2|^2 = r_1^2 + r_2^2 + 2r_1r_2\cos(\theta_1 + \theta_2)$
  \end{enumerate}
\end{frame}
\begin{frame}{Vector Representation}
  Complex numbers can also be represented as vectors. Length of the vector is
  nothing but modulus of complex number and argument is the angle which the
  vector makes with the real axis. It is denoted as $\overrightarrow{OP}$ where
  $OP$ represents the vector of the complex number $z.$
\end{frame}
\end{document}
