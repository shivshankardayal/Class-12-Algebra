\documentclass[aspectratio=1610,8pt]{beamer}

% Standard packages

\usepackage[english]{babel}
%\usepackage[latin1]{inputenc}
%\usepackage{times}
%\usepackage[T1]{fontenc}
\usepackage{fontspec}
\usepackage[]{unicode-math}
\setmathfont{Inconsolata}
\setsansfont{Roboto}


% Setup TikZ

\usepackage{tikz}
\usetikzlibrary{arrows}
\tikzstyle{block}=[draw opacity=0.7,line width=1.4cm]


% Author, Title, etc.

\title{Complex Numbers}

\author[Shiv Shankar Dayal]{Shiv Shankar Dayal}

\begin{document}
\begin{frame}
       \titlepage
\end{frame}
\begin{frame}{Theory}
  A complex number comprises of two numbers: a real number and an imaginary number. An imaginary number is square root of a
  negative number, for example, $\sqrt{-1}, \sqrt{-2}, \sqrt{-3}.$ These are called imaginary numbers because they do not exist in
  real life in the sense that like ordinary numbers they cannot be used for counting.

  A real number like $1$ can also be represented as a complex number having a $0$ imaginary part.
\end{frame}
\end{document}
