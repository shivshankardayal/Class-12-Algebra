\documentclass[aspectratio=169,8pt]{beamer}

% Standard packages

\usepackage[english]{babel}
%\usepackage[latin1]{inputenc}
%\usepackage{times}
%\usepackage[T1]{fontenc}
\usepackage{fontspec}
\usepackage[]{unicode-math}
\setmathfont{Inconsolata}
\setsansfont{Roboto}


% Setup TikZ

\usepackage{tikz}
\usetikzlibrary{arrows}
\tikzstyle{block}=[draw opacity=0.7,line width=1.4cm]

\newcounter{counter}
% Author, Title, etc.

\title{Complex Numbers}

\author[Shiv Shankar Dayal]{Shiv Shankar Dayal}

\begin{document}
\begin{frame}
       \titlepage
\end{frame}
\begin{frame}{Theory}
  A complex number comprises of two numbers: a real number and an imaginary number. An imaginary number is square root of a
  negative number, for example, $\sqrt{-1}, \sqrt{-2}, \sqrt{-3}.$ These are called imaginary numbers because they do not exist in
  real life in the sense that like ordinary numbers they cannot be used for counting.\\
  \vspace*{0.2cm}
  A real number like $1$ can also be represented as a complex number having a $0$ imaginary part. The value $\sqrt{-1}$ is denoted
  by the Greek letter $\iota,$ which stands for \textit{iota.} Typically, we use either $i$ or $j$ to denote this.\\
  \vspace*{0.2cm}
  Clearly we have following:
  $$i^2 = -1, i^3 = -i, i^4 = 1, i^5 = i, i^6 = -1, i^7 = -i, i^8 = 1, \ldots$$
  If you examine carefully you will find that following holds true
  $$i^{4m} = 1, i^{4m + 1} = i, i^{4m + 2} = -1\text{~and~}i^{4m + 3}= -i~\forall~m\in P$$
  $P$ is the set of positive integers including zero.

  \textbf{Note:} $1 = \sqrt{1} = \sqrt{-1*-1} = i * i = -1$

  However, the above result is wrong because for any two real numbers $a$ and $b$ the result $\sqrt{a}*\sqrt{b} = \sqrt{ab}$ holds
  good if and only if the two numbers are zero or positive. Thus $1 = \sqrt{-1*-1}$ is wrong because power of $-$ is $-1$ which
  makes the set of equalities go wrong.
\end{frame}
\begin{frame}{Definitions}
  A complex number is commonly written as $a + ib$ or $x + iy.$ Here $a, b, x$ and $y$ are all real numbers. The complex number
  itself is denoted by $z,$ like $z = x + iy.$ Here $x$ is called the \textit{real} part and is also denote by $Re(z)$ and $y$ is
  called the imaginary part and is also denoted by $Im(z).$\\
  \vspace*{0.2cm}
  A complex number is purely real if its imaginary part or $y$ or $Im(z)$ is zero. Similarly, a complex number is purely imaginary
  if its real part or $x$ or $Re(z)$ is zero. Clearly, as you can fathom that there can exist only one number which has both the
  parts as zero and certainly that is $0.$ That is, $0=0+i0.$\\
  \vspace*{0.2cm}
  The set of all complex number is typically denoted by $C$. Two complex numbers $z_1$ and $z_2$ are said to be true if there real
  parts are equal and imaginary parts are equal. That is if $z_1 = x_1 + iy_1$ and $z_2 = x_2 + iy_2$ then for $z_11$ to be equal
  to $z_2,$ $x_1$ must be equal to $x_2$ and $y_1$ must be equal to $y_2.$
\end{frame}
\begin{frame}{Simple Operations}
  \begin{enumerate}
  \item \textbf{Addition:} $(a + ib) + (c + id) = (a + c) + i(b + d)$
  \item \textbf{Subtraction:} $(a + ib) - (c + id) = (a - c) + i(b - d)$
  \item \textbf{Multiplication:} $(a + ib) * (c + id) = ac + ibc + iad + bdi^2 = (ac - bd) + i(bc + ad)$
    \item \textbf{Division:} $\frac{a + ib}{c + id} = \frac{a + ib}{c + id}.\frac{c - id}{c - id} = \frac{ac + bd + i(bc + ad)}{c^2
    + d^2}$
  \end{enumerate}
  \vspace*{.2cm}
  \textbf{\large{Conjugate of a Complex Number}}\\
  \vspace*{.2cm}
  Let $z = x + iy$ be a complex number then its complex conjugate is a number with imaginary part made negative and it is written
  as $\overline{z} = x - iy.$ $\overline{z}$ is the typical representation for a conjugate of a complex number $z$.\\
  \vspace*{0.2cm}
  \textbf{Properties of Conjugates}\\
  \vspace*{0.2cm}
  \begin{enumerate}
  \item $z_1 = z_2 \Leftrightarrow \overline{z_1} = \overline{z_2}$\\
    Clearly as we know for two complex numbers to be equal both parts must be equal so this is very easy to understand that if $x_1
    = x_2$ and $y_1 = y_2$ then this bidirectional condition is always satisfied.
  \item $\overline(\overline{z}) = z.$\\
    $z = x + iy,$ hence, $\overline{z} = x - iy,$ hence $\overline(\overline{z}) = x - (-iy) = x + iy = z$
  \item $z + \overline{z} = 2Re(x)$\\
    Clearly, $z + \overline{z} = x + iy + x - iy = 2x = 2Re(x)$
  \item $z - \overline{z} = 2iIm(x)$\\
    Clearly, $z - \overline{z} = x + iy - (x - iy) = 2iy = 2iIm(x)$
    \setcounter{counter}{\value{enumi}}
  \end{enumerate}
\end{frame}
\begin{frame}{Conjugate contd.}
  \begin{enumerate}
    \setcounter{enumi}{\value{counter}}
  \item $z + \overline{z} = 0 \Leftrightarrow z$ is purely imaginary.\\
    $z + \overline{z} = x + iy + x - iy = 2x = 0$ which means rela part is zero and hence $z$ is purely imaginary.
  \item $z = \overline{z} \Leftrightarrow z$ is purely real.\\
    $x + iy = x -iy \Rightarrow 2iy = 0$ and thus $z$ is purely real.
  \item $z\overline{z} = [x^2 + y^2]$\\
    Clearly, $z\overline{z} = (x + iy)(x - iy) = x^2 + y^2$
  \item $\overline{z_1 + z_2} = \overline{z_1} + \overline{z_1}\overline{z_1 + z_2} = \overline{(x_1 + iy_1) + (x_2 + iy_2)} = \overline{(x_1 + x_2) + i(y_1 + y_2)}$\\
    $= (x_1 + x_2) -i(y_1 + y_2) = x_1 - iy_1 + x_2 - iy_2 = \overline{z_1} + \overline{z_2}$
  \item $\overline{z_1 - z_2} = \overline{z_1} - \overline{z_2}$\\
    It can be proven like item 8.
  \item $\overline{z_1z_2} = \overline{z_1} - \overline{z_2}$\\
    It can be proven like item 8.
  \item $\overline{\left(\frac{z_1}{z_2}\right)} = \frac{\overline{z_1}}{\overline{z_2}}$ if $z_2\neq 0$
    You can rationalize the base by multiplying it from its conjugate and apply division formula given above to prove it.
  \item If $P(z) = a_0 + a_1z + a_2z^2 + \ldots + a_nz^n.$ where $s_0, a_1, \ldots, a_n$ and $z$ are complex numbers, then\\
    $\overline{P(z)} = \overline{a_0} + \overline{a_1}\overline{z} + \overline{a_2}(\overline{z})^2 + \ldots + \overline{a_n}(\overline{z)^n} = \overline{P}(\overline{z})$ where\\
    $\overline{P}(z) = \overline{a_0} + \overline{a_1}z + \overline{a_2}z^2 + \ldots + \overline{a_n}z^n$
    \setcounter{counter}{\value{enumi}}
  \end{enumerate}
\end{frame}
\begin{frame}{Conjugate contd.}
  \begin{enumerate}
    \setcounter{enumi}{\value{counter}}
  \item If $R(z) = \frac{P(z)}{Q(z)}$ where $P(z)$ and $Q(z)$ are polynomilas in $z,$ and $Q(z)\neq 0,$ then\\
    $\overline{R(z)} = \frac{\overline{P}(\overline{z})}{\overline{Q}(\overline{z})}$
  \item If $z = \begin{vmatrix}a_1 & a_2 & a_3\\b_1 & b_2 & b_3\\c_1 & c_2 & c_3\end{vmatrix}$, then $\overline{z} = \begin{vmatrix}\overline{a_1} & \overline{a_2} & \overline{a_3}\\\overline{b_1} & \overline{b_2} & \overline{b_3}\\\overline{c_1} & \overline{c_2} & \overline{c_3}\end{vmatrix}$ where $a_i, b_i, c_i(i = 1,2,3)$ are complex numbers.
  \end{enumerate}
  \vspace*{0.2cm}
  \textbf{\large{Modulus of a Complex Number}}\\
  \vspace*{0.2cm}
  Modulus of a complex numbe $z$ is denoted by $|z|$ and is equalt to the real number $\sqrt{x^2 + y^2}$. Note that $|z|\geq 0~\forall~z\in C$\\
  \vspace*{0.2cm}
  \textbf{Properties of Modulus}\\
  \vspace*{0.2cm}
  \begin{enumerate}
  \item $|z| = 0 \Leftrightarrow z = 0.$\\
    $x^2 + y^2 = 0 \Leftrightarrow x = 0, y = 0 \Rightarrow z = 0$
  \item $|z| = |\overline{z}| = |-z| = |-\overline{z}| = x^2 + y^2$
  \item $-|z|\leq Re(x)\leq |z|$ Clearly, $-(x^2 + y^2) \leq x^2 \leq (x^2 + y^2)$
  \item $-|z|\leq Im(x)\leq |z|$ Clearly, $-(x^2 + y^2) \leq y^2 \leq (x^2 + y^2)$
  \item $z\overline{z} = |z|^2$ Clearly, $(x + iy)(x - iy) = (x^2 + y^2) = |z|^2$
  \item $|z_1z_2| = |z_1||z_2|$ Clearly, $|z_1z_2| = |x_1x_2 - y_1y_2 + i(x_1y_2 + x_2y_1))|$\\
    $= \sqrt{(x_1x_2 - y_1y_2)^2 + (x_1y_2 + x_2y_1)^2} = \sqrt{(x_1^2 + y_1^2)(x_2^2 + y_2^2)} = |z_1||z_2|$
    \setcounter{enumi}{\value{counter}}
  \end{enumerate}
\end{frame}
\begin{frame}{Modulus contd.}
  \begin{enumerate}
    \setcounter{enumi}{\value{counter}}
  \item $\left|\frac{z_1}{z_2}\right| = \frac{|z_1|}{z_2},$ if $z_2\neq 0$
  \item $|z_1 + z_2|^2 = |z_1|^2 + |z_2|^2 + \overline{z_1}z_2 + z_1\overline{z_2} = |z_1|^2 + |z_2|^2 + 2Re(z_1\overline{z_2})$
  \item $|z_1 - z_2|^2 = |z_1|^2 + |z_2|^2 - \overline{z_1}z_2 - z_1\overline{z_2} = |z_1|^2 + |z_2|^2 - 2Re(z_1\overline{z_2})$
  \item $|z_1 + z_2|^2 + |z_1 - z_2|^2 = 2(|z_1|^2 + |z_2|^2)$
  \item If $a$ amd $b$ are real numbers and $z_1$ and $z_2$ are complex numbers, then\\
    $|az_1 + bz_2|^2 + |bz_1 - az_2|^2 = (a^2 + b^2)(|z_1|^2 + |z_2|^2)$
  \item If $z_1, z_2\neq 0,$ then $|z_1 + z_2|^2 = |z_1|^2 + |z_2|^2 \Leftrightarrow \frac{z_1}{z_2}$ is purely imaginary.
  \item If $z_1$ and $z_2$ are complex numbers then $|z_1 + z_2|\leq |z_2| + |z_2|.$ This expression can be generalized to $n$ terms as well.
  \item Simialrly, these can be proven that $|z_1 - z_2|\leq |z_1| + |z_2|, |z_1| - |z_2|\leq |z_1| + |z_2|$ and $|z_1 - z_2|\geq ||z_1| - |z_2||$
  \end{enumerate}
\end{frame}
\end{document}
