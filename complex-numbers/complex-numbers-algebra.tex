\documentclass[aspectratio=169,8pt]{beamer}

% Standard packages

\usepackage[english]{babel}
%\usepackage[latin1]{inputenc}
%\usepackage{times}
%\usepackage[T1]{fontenc}
\usepackage{fontspec}
\usepackage[]{unicode-math}
\setmathfont{Inconsolata}
\setsansfont{Roboto}

% Setup asymptote
\usepackage[inline]{asymptote}

% Setup TikZ

\usepackage{tikz}
\usetikzlibrary{arrows}
\tikzstyle{block}=[draw opacity=0.7,line width=1.4cm]

\newcounter{counter}
% Author, Title, etc.

\title{Complex Numbers}

\author[Shiv Shankar Dayal]{Shiv Shankar Dayal}

\begin{document}
\begin{frame}
  \titlepage
\end{frame}
\begin{frame}{Geometrical Representation}
  Lte $z_1 = x_1 + iy_1$ and $z_2 = x_2 + iy_2$ be two complex numbers which
  are represented by two points $P_1$ and $P_2$ in the following diagrams.\\
  \vspace*{0.2cm}
  \textbf{\large{Addiiton}}
  \begin{center}
    \begin{tikzpicture}
      \draw[->, >=stealth] (-.5,0) -- (5.5,0);
      \draw[->, >=stealth] (0,-.5) -- (0,5.5);
      \draw (5.7, 0) node {$X$};
      \draw (0,5.7) node {$Y$};
      \draw (0,0) -- (4,1);
      \draw (0,0) -- (1,4);
      \draw (1,4) -- (5,5);
      \draw (4,1) -- (5,5);
      \draw[dashed] (4,1) -- (4,0);
      \draw[dashed] (1,4) -- (1,0);
      \draw[dashed] (5,5) -- (5,0);
      \draw[dashed] (4,1) -- (5,1);
      \draw (-.5,-.5) node {$O$};
      \draw (1,-.5) node {$L$};
      \draw (4,-.5) node {$M$};
      \draw (5,-.5) node {$N$};
      \draw (5.2,1) node {$K$};
      \draw (3, 1.2) node {$P1(x_1+iy_1)$};
      \draw (1.3, 4.3) node {$P2(x_2+iy_2)$};
      \draw (5, 5.2) node {$P(x+iy)$};
    \end{tikzpicture}
  \end{center}
\end{frame}
\begin{frame}{Addition of Two Complex Numbers}
  Clearly, $z = z_1 + z_2 = x_1 + x_2 + i(y_1 + y_2)$.\\
  \vspace*{0.2cm}
  Let $P_1M, P_2L$ and $PN$ be parallel to the $y$-axis; $P_1K$ be parallet to
  the $x$-axis. This implied that triangle $OP_2L$ and $PP_1K$ are congruent.\\
  \vspace*{0.2cm}
  We have $P_1K = OL = x_1$ and $P_2L = PK = y_1$\\
  \vspace*{0.2cm}
  Thus, $ON = OM + MN = OL + P_1K = x_1 + x_2$\\
  \vspace*{0.2cm}
  and $PN = PK + KN = P_2L + P_1M = y_2 + y_1$\\
  \vspace*{0.2cm}
  So we can say that coordinates of $P$ are $(x_1 + x_2, y_1 + y_2)$ which
  represents the complex number $z.$\\
  \vspace*{0.2cm}
  We also see that this obeys vector addition i.e. $OP_1 + OP_2 = OP_1 + P_1P = OP$
\end{frame}
\begin{frame}[fragile]{Subtraction}
  \begin{center}
    \begin{asy}
      import fontsize;
      unitsize(1cm);
      defaultpen(fontsize(9pt));
      draw((-1,0) --(3, 0), arrow=Arrow);
      draw((0,-1) --(0, 3), arrow=Arrow);
      draw((0,0) -- (2, 1));
      draw((0,0) -- (1, 2));
      draw((0, 0) -- (-1, -2));
      draw((0,0) -- (1, -1));
      draw((1,2) -- (2, 1));
      draw((2,1) -- (1, -1));
      draw((-1,-2) -- (1, -1));
      label("$x$", (3,0), align=E);
      label("$y$", (0,3), align=N);
      label("$P_1(x_1, y_1)$", (2, 1), align=NE);
      label("$P_2(x_2, y_2)$", (1, 2), align=N);
      label("$P(x_1 - x_2, y_1 - y_2)$", (1, -1), align=SE);
      label("$P_2'(-x_2,- y_2)$", (-1, -2), align=S);
    \end{asy}
  \end{center}
\end{frame}
\begin{frame}{Subtraction}
  We first represent $-z_2$ by $P_2'$ so that  $P_2P_2'$ is bisected at $O.$
  Complete the parallelogram $OP_1PP_2'.$ Then it can be easily seen that $P$
  representd the difference $z_1 - z_2.$\\
  \vspace*{0.2cm}
  As $OP_1PP_2'$ is a parallelogram so $P_1P = OP_2'.$ Using vetor notation, we
  have,\\
  \vspace*{0.2cm}
  $z_1 - z_2 = OP_1 - OP_2 = OP_1 + OP_2' = OP_1 + P_1P = P_2P1$\\
  \vspace*{0.2cm}
  It follows that the complex number $z_1 - z_2$ is represented by the vector
  $P_1P_2,$ where points $P_1$ and $P_2$ represent the complex numbers $z_1$
  and $z_2$ respectively.\\
  \vspace*{0.2cm}
  It should be noted that $arg(z_1 - z_2)$ is the angle through which $OX$ must
  be rotated in the anticlockwise direction to make it parallel with $P_1P_2.$
\end{frame}
\begin{frame}[fragile]{Multiplication}
  \begin{center}
    \begin{asy}
      import geometry;
      import fontsize;
      unitsize(1cm);
      defaultpen(fontsize(6pt));
      markangle("$\theta_1$", radius=10, (3, 0), (0, 0), (1.732, 1), 0.5*green);
      markangle("$\theta_2$", radius=20, (3, 0), (0, 0), (1, 1), 0.5*blue);
      markangle("$(\theta_1 + \theta_2)$", radius=30, (3, 0), (0, 0), (.732, 2.732), 0.5*red);
      draw((-1,0) --(3, 0), arrow=Arrow);
      draw((0,-1) --(0, 3), arrow=Arrow);
      draw((0, 0) -- (1.732, 1));
      draw((0, 0) -- (1, 1));
      draw((0, 0) -- (.732, 2.732));
      label("$x$", (3,0), align=E);
      label("$y$", (0,3), align=N);
      label("$P_1(r_1.e^{i\theta_1})$", (1.732,1), align=E);
      label("$P_2(r_2.e^{i\theta_2})$", (1,1), align=N);
      label("$P(r_1r_2.e^{i(\theta_1 + \theta_2)})$", (.732,2.732), align=NE);
    \end{asy}
  \end{center}
  For multiplication it is convenient to use Euler's formula of complex numbers.\\
  \vspace*{0.2cm}
  Let $z_1 = r_1e^{i\theta_1}$ and $z_2 = r_2e^{i\theta_2},$ then clealry, $z_1z_2 = r_1r_2e^{i(\theta_1 + \theta_2)}$
\end{frame}
\begin{frame}[fragile]{Division}
  \begin{center}
    \begin{asy}
      import geometry;
      import fontsize;
      unitsize(1cm);
      defaultpen(fontsize(6pt));
      markangle("$\theta_1$", radius=10, (3, 0), (0, 0), (1.732, 1), 0.5*green);
      markangle("$\theta_2$", radius=20, (3, 0), (0, 0), (1, 1), 0.5*blue);
      markangle("$(\theta_1 - \theta_2)$", radius=10, (1.37, -.37), (0, 0), (3, 0), 0.5*red);
      draw((-1,0) --(3, 0), arrow=Arrow);
      draw((0,-1) --(0, 3), arrow=Arrow);
      draw((0, 0) -- (1.732, 1));
      draw((0, 0) -- (1, 1));
      draw((0, 0) -- (1.37, -.37));
      label("$x$", (3,0), align=E);
      label("$y$", (0,3), align=N);
      label("$P_1(r_1.e^{i\theta_1})$", (1.732,1), align=E);
      label("$P_2(r_2.e^{i\theta_2})$", (1,1), align=N);
      label("$P(r_1/r_2.e^{i(\theta_1 - \theta_2)})$", (1.37, -.37), align=SE);
    \end{asy}
  \end{center}
  For division also it is convenient to use Euler's formula of complex numbers.\\
  \vspace*{0.2cm}
  Let $z_1 = r_1e^{i\theta_1}$ and $z_2 = r_2e^{i\theta_2},$ then clealry, $z_1/z_2 = r_1/r_2e^{i(\theta_1 - \theta_2)}$
\end{frame}
\begin{frame}[fragile]{Three Important Results}
  \begin{center}
    \begin{asy}
      import geometry;
      import fontsize;
      unitsize(1cm);
      defaultpen(fontsize(6pt));
      draw((-.3,0) --(3, 0), arrow=Arrow);
      draw((0,-.3) --(0, 2.5), arrow=Arrow);
      draw((0,0) -- (1,2));
      draw((0,0) -- (1.75,0.5));
      draw((1,2) -- (2,0));
      label("$x$", (3,0), align=E);
      label("$y$", (0,2.5), align=N);
      label("$O$", (0,0), align=SW);
      label("$Q(z_2)$", (1, 2), align =N);
      label("$P(z_1)$", (1.75, 0.5), align=NE);
      markangle("$\theta$", radius=5, (2.5, 0), (2, 0), (1.75, 0.5));
    \end{asy}
  \end{center}
  $z_1 - z_2 = \overrightarrow{OP} - \overrightarrow{OQ} = \overrightarrow{QP}$\\
  \vspace*{0.2cm}
  $\therefore |z_1 - z_2| = |\overrightarrow{QP}| = QP$ which is nothing but distance between $P$ and $Q.$\\
  \vspace*{0.2cm}
  $arg(z_1 - z_2)$ is the angle made by $\overrightarrow{QP}$ with $x$-axis whis is nothing but $\theta.$
\end{frame}
\begin{frame}[fragile]
  \begin{center}
    \begin{asy}
      import geometry;
      import fontsize;
      unitsize(1cm);
      draw((1,0) -- (2,2), dashed);
      draw((-1,0) -- (2,2), dashed);
      draw((2,2) -- (3,4), arrow=Arrow);
      draw((2,2) -- (3.5,3), arrow=Arrow);
      draw((-1.5,0) --(4, 0), arrow=Arrow);
      draw((0,-.3) --(0, 2.5), arrow=Arrow);
      label("$x$", (4,0), align=E);
      label("$y$", (0,2.5), align=N);
      label("$O$", (0,0), align=SW);
      markangle("$\theta$", radius=10, (3.5,3), (2,2), (3,4));
      markangle("$\alpha$", radius=10, (3,0), (1,0), (2,2));
      markangle("$\beta$", radius=10, (3,0), (-1,0), (2,2));
      label("$P(z_1)$", (2,2), align=E);
      label("$Q(z_2)$", (3,4), align=NE);
      label("$R(z_3)$", (3.5,3), align=NE);
    \end{asy}
  \end{center}
  $\theta = \alpha - \beta = arg(z_3 - z_1) - arg(z_2 - z_1) \Rightarrow \theta = arg\frac{z_3 - z_1}{z_2 - z_1}$\\
  \vspace*{0.2cm}
  Similarly if three complex numbers are vertices of a triangle then angles of those vertices can also be computed using previous results.\\
  \vspace*{0.2cm}
  Similarly, for four points to be concyclic where those points are represented by $z_1, z_2, z_3$ and $z_4$ if
  $$arg\left(\frac{z_2 - z_4}{z_1 - z_4}.\frac{z_1 - z_3}{z_2 - z_4}\right) = 0$$
\end{frame}
\begin{frame}{Any Root of an Imaginary Number is an Imaginary Number}
  Let $iy$ be an imaginary number such that $y\neq = 0$\\
  \vspace*{0.2cm}
  Let $\sqrt[n]{iy} = a, \therefore iy = a^n$\\
  \vspace*{0.2cm}
  If $a$ is real then $a^n$ will also be real which is not possib;e as $iy$ is an imaginary number so $a$ will also be imaginary.\\
  \vspace*{0.2cm}
  {\large Square Root of a Complex Number}\\
  \vspace*{0.2cm}
  Consider a complex number $z = x + iy$ and let us say that $\sqrt{x + iy} = a + ib \Rightarrow x + iy = (a^2 - b^2) + 2abi$\\
  $$\Rightarrow x = a^2 - b^2, y = 2ab$$
  then we can write
  $$a^2 + b^2 = \sqrt{(a^2 - b^2)^2 + 4a^2b^2}$$
  Thus, from these two equations we can write
  $$a = \pm\sqrt{\frac{\sqrt{x^2 + y^2} + x}{2}}, b = \pm\sqrt{\frac{\sqrt{x^2 + y^2} - x}{2}}$$
\end{frame}
\begin{frame}{Cube Roots of Unity}
  Let $x = 1^{1/3} \Rightarrow x^3 = 1 \Rightarrow x^3 - 1=0 \Rightarrow (x - 1)(x^2 + x + 1) = 0$\\
  \vspace*{0.2cm}
  $x = -1, \frac{-1\pm\sqrt{-3}}{2}$\\
  \vspace*{0.2cm}
  It can be easily verified that if $\omega = \frac{-1 + \sqrt{3}i}{2}$ then $\omega^2 = \frac{-1 - \sqrt{3}i}{2}$\\
  \vspace*{0.2cm}
  Thus, three roots of cube root of unity are $1, \omega$ and $\omega^2.$\\
  \vspace*{0.2cm}
  It can be easily verified that $1 + \omega + \omega^2 = 0$(because $\omega$ is one of the roots) and $\omega^3 = 1$.
\end{frame}
\end{document}
