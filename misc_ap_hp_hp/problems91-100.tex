\documentclass[aspectratio=1610,8pt]{beamer}

% Standard packages

\usepackage[english]{babel}
%\usepackage[latin1]{inputenc}
%\usepackage{times}
%\usepackage[T1]{fontenc}
\usepackage{fontspec}
\usepackage[]{unicode-math}
\setmathfont{Inconsolata}
\setsansfont{Roboto}

% Setup TikZ

\usepackage{tikz}
\usetikzlibrary{arrows}
\tikzstyle{block}=[draw opacity=0.7,line width=1.4cm]


% Author, Title, etc.

\title{Miscellaneous Problems on A.P., G.P. and H.P.\\Problems 91-100}

\author[Shiv Shankar Dayal]{Shiv Shankar Dayal}

% The main document

\begin{document}
\begin{frame}
  \titlepage
\end{frame}
\begin{frame}{Problem 91}
  \textbf{91.} One side of an equilateral triangle is $24$ cm. The mid-point of its sides are joined to form another triangle,
  whose mid-points are in turn joined for form still another triangle. The process continues indefinitely. Find the sum of
  perimeter of all the triangles.
\end{frame}
\begin{frame}{Solution of Problem 91}
  \textbf{Solution:} Perimeter of first triangle will be $24*3$ i.e. $72$ cm. The triangle which will be formed by joining
  mid-points of this triangle will have sides of $12$ cm each.
  \linebreak\linebreak
  Thus, perimeter of second trianble will be $12*3$ i.e. $36$ cm. The triangle which will be formed by joining mid-points of this
  triangle will have sides of $6$ cm each and so on.
  \linebreak\linebreak
  Thus sum of perimeter of all such triangles will be $$72 + 36 + 18 + 9 + \ldots \infty$$
  $$= \frac{72}{1 - \frac{1}{2}} = 144\text{~cm}$$
\end{frame}
\begin{frame}{Problem 92}
  \textbf{92.} A ball is dropped from a height of $900$ cm. Each time is rebounces, it rises to $2/3$ of the height it has fallen
  through. Find the total distance travelled by the ball before it comes to rest.
\end{frame}
\begin{frame}{Solution of Problem 92}
  \textbf{Solution:} First the ball will fall a distance for $900$ cm. Then it will rise and fall for $2/3$rd of that distance
  travelling a total of $2*900*2/3$ cm. Then it will rise and fall for $2/3$rd of that distance travelling a total of
  $2*900*(2/3)^2$ cm. This process will go on till infinity. Thus we can write following equation for total distance travelled,
  $S$(say)
  $$S = 900 + 2*900*\frac{2}{3} + 2*900*\frac{2^2}{3^2} + 2*900*\frac{2^3}{3^3} + \ldots \infty$$
  $$= 900 + \frac{1200}{1 - \frac{2}{3}} = 4500 \text{~cm}$$
\end{frame}
\begin{frame}{Problem 93}
  \textbf{93.} A square is drawn by joining the mid-points of the sides of a given square. A third square is drawn inside the
  second square in the same way and this process continutes indeinitely. If the sides of the first square is $4$ cm, determine the
  sum of the areas of all the squares.
\end{frame}
\begin{frame}{Solution of Problem 93}
  \textbf{Solution:}
  \begin{center}
    \begin{tikzpicture}
      \draw (0, 0) -- (2, 0) -- (2, 2) -- (0, 2) -- cycle;
      \draw (1, 0) -- (2, 1) -- (1, 2) -- (0, 1) -- cycle;
      \draw (0, 0) -- (2, 2);
      \draw (0, 0) node[anchor=north east] {$A$};
      \draw (2, 0) node[anchor=north west] {$B$};
      \draw (2, 2) node[anchor=south west] {$C$};
      \draw (0, 2) node[anchor=south east] {$D$};
      \draw (1, 0) node[anchor=north] {$P$};
      \draw (2, 1) node[anchor=west] {$Q$};
      \draw (1, 2) node[anchor=south] {$R$};
      \draw (0, 1) node[anchor=east] {$S$};
    \end{tikzpicture}
  \end{center}
  Let $ABCD$ be the first sqaure. Let $AB = a \Rightarrow AC = \sqrt{2}a~\therefore PQ = \frac{AC}{2} = \frac{a}{\sqrt{2}}$
  \linebreak\linebreak
  $\therefore $ Area of first square $= a^2$
  \linebreak\linebreak
  Area of second sqaure $= \frac{a^2}{2}$
  \linebreak\linebreak
  Area of third square $= \frac{a^2}{4}$
  \linebreak\linebreak
  Sum of areas of all squares $= a^2 + \frac{a^2}{2} + \frac{a^2}{4} + \ldots = \frac{a^2}{1 - \frac{1}{2}} = 2a^2 = 32$ sq.cm.
\end{frame}
\begin{frame}{Problem 94}
  \textbf{94.} In an increasing G.P., the sum of the first and the last term is $66,$ the product of the second and the last term
  but one term is $128,$ and the sum of all the terms is $126.$ How many terms are there in the progression?
\end{frame}
\begin{frame}{Solution of Problem 94}
  \textbf{Solution:} Let $a$ be the first term and $r$ be the common ratio. Also, let that there are $n$ terms in the G.P. Then
  according to the question
  $$a + ar^{n - 1} = 66, ar.ar^{n - 2} = 128, \frac{a(1 - r^n)}{1 - r} = 126$$
  $$\Rightarrow a^2r^{n - 1} = 128 \Rightarrow a + \frac{128}{a} = 66 \Rightarrow a^2 - 66a + 128 = 0 \Rightarrow a = 2, 64$$
  $$\text{For~} a = 2, a - ar^n = 126(1 - r)\Rightarrow 2 - \frac{128}{2}.r = 126(1 - r) \Rightarrow r = 2$$
  $$\text{For~} a = 64, a - ar^n = 126(1 - r) \Rightarrow 64 - \frac{128}{64}r = 126(1 - r) \Rightarrow r = \frac{1}{2}$$
  However, the G.P. is increasing so $a = 2, r = 2$
  $$\Rightarrow ar.ar^{n - 2} = 128 \Rightarrow 2^2.2^{n - 1} = 128 \Rightarrow 2^{n - 1} = 32 \Rightarrow n = 6$$
\end{frame}
\end{document}
