\documentclass[aspectratio=1610,8pt]{beamer}

% Standard packages

\usepackage[english]{babel}
%\usepackage[latin1]{inputenc}
%\usepackage{times}
%\usepackage[T1]{fontenc}
\usepackage{fontspec}
\usepackage[]{unicode-math}
\setmathfont{Inconsolata}
\setsansfont{Roboto}

% Setup TikZ

\usepackage{tikz}
\usetikzlibrary{arrows}
\tikzstyle{block}=[draw opacity=0.7,line width=1.4cm]


% Author, Title, etc.

\title{Miscellaneous Problems on A.P., G.P. and H.P.\\Problems 11-20}

\author[Shiv Shankar Dayal]{Shiv Shankar Dayal}

% The main document

\begin{document}
\begin{frame}
  \titlepage
\end{frame}
\begin{frame}{Problem 11}
  \textbf{11.} In an A.P. of $2n$ terms the middle pair of terms are $p + q$ and $p - q.$ Show that the sum of cubes of the terms
  in A.P. are $2np[p^2 + (4n^2 - 1)q^2]$
\end{frame}
\begin{frame}{Solution of Problem 11}
  \textbf{Solution:} Let $t_r$ denote the $r$th term of the A.P.
  \linebreak\linebreak
  Given, $t_n = p + q$ and $t_{n + 1} = p - q \therefore d = -2q$
  \linebreak\linebreak
  Also, $t_1 + t_{2n} = t_2 + t_{2n - 1} = \ldots = t_n + t_{n + 1} = 2p$
  \linebreak\linebreak
  Let $S$ be the sum of cubes of the terms of A.P., then $S = (t_1^3 + t_{2n}^3) + (t_2^3 + t_{2n - 1}^3) + \ldots + (t_n^3 + t_{n
    - 1}^3)$
  $$t_1^3 + t_{2n}^3 = (t_1 + t_{2n})^3 - 3t_1t_{2n}(t_1 + t_{2n}) = 8p^3 - 6pt_1t_{2n} = 8p^3 - \frac{6p}{4}[(t_1 + t_{2n})^2 -
    (t_1 - t_{2n})^2]$$
  $$= 8p^3 - \frac{3p}{2}[4p^2 - (2n - 1)^2.4q^2][\because t_{2n} = t_1 + (2n - 1)d~\text{and}~d = -2q]$$
  $$= 2p^3 + 6pq^2(2n - 1)^2$$
  Similarly $$t_2^3 + t_{2n - 1}^3 = 2p^3 + 6pq^2(2n - 3)^2, t_3^3 + t_{2n - 2}^3 = 2p^3 + 6pq^2(2n - 5)^2, \ldots, t_n^3 + t_{n +
    1}^3 = 2p^3 + 6pq^2.1^2$$
  Adding all these, we get
  $$S = 2np^3 + 6pq^2[1^2 + 3^2 + 5^2 + \ldots~\text{to}~n~\text{terms}]$$
  $$= 2np[p^2 + (4n^2 - 1)q^2]$$
\end{frame}
\begin{frame}{Problem 12}
  \textbf{12.} Find the sum $S_n$ of the cubes of the first $n$ terms of an A.P. and show that the sum of the first $n$ terms of
  the A.P. is a factor of $S_n.$
\end{frame}
\begin{frame}{Solution of Problem 12}
  \textbf{Solution:} Let $S$ be the sum of first $n$ terms of the A.P. $a, a + d, a + 2d, \ldots$ then $S = \frac{n}{2}[2a + (n -
    1)d]$
  $$S_n = a^3 + (a + d)^3 + (a + 2d)^3 + \ldots + [a + (n - 1)d]^3$$
  $$= na^3 + 3a^2d[1 + 2 + 3 + \ldots + (n - 1)] + 3ad^2[1^2 + 2^2 + \ldots + (n - 1)^2] + d^3[1^3 + 2^3 + \ldots + (n - 1)^3]$$
  $$= na^3 + 3a^2d.\frac{n(n - 1)}{2} + 3ad^2.\frac{(n - 1).n.(2n - 1)}{6} + d^3\frac{n^2(n - 1)^2}{4}$$
  $$= \frac{n}{2}\left(2a^3 + 3(n - 1)a^2d + (n - 1)(2n - 1)ad^2 + \frac{1}{2}n(n - 1)^2d^3\right)$$
  $$= \frac{n}{2}\left[a^2(2a + (n - 1)d) + (n - 1)ad(2a + (n - 1)d) + \frac{n(n - 1)}{2}d^2(2a + (n - 1)d)\right]$$
  $$= \frac{n}{2}[2a + (n - 1)d]\left[a^2 + (n - 1)ad + \frac{n()n - 1}{2}d^2\right]$$
  $$= S\left[a^2 + (n - 1)ad + \frac{n()n - 1}{2}d^2\right]$$

  Hence, $S$ is a factor of $S_n$
\end{frame}
\begin{frame}{Problem 13}
  \textbf{13.} Show that any positive integral power (greater than $1$) of a positive integer $m,$ is the sum of $m$ consecutive
  odd positive integers. Find the first odd integer for $m^r(r > 1)$
\end{frame}
\begin{frame}{Solution of Problem 13}
  \textbf{Solution:} Let $r$ be a positive integer and $r > 1.$
  \linebreak\linebreak
  Let $m^r = (2k + 1)+ (2k + 3) + \ldots + (2k + 2m - 1)$
  $$m^r = \frac{m}{2}[4k + 2 + (m - 1)2] \Rightarrow m^{r - 1} = 2k + m \Rightarrow k = \frac{m^{r - 1} - m}{2}$$
  Clearly for $r > 1, m^{r - 1}$ and $m$ are both odd or both even. $\therefore m^{r - 1} - m$ is an even number. Thus, such
  integer $k$ exists.
  \linebreak\linebreak
  First off interger $= 2k + 1 = m^{r - 1} - m + 1$
\end{frame}
\begin{frame}{Problem 14}
  \textbf{14.} If $a$ be the sum of $n$ terms and $b^2$ the sum of the square of $n$ terms of an A.P., find the first term and
  common difference of the A.P.
\end{frame}
\begin{frame}{Solution of Problem 14}
  \textbf{14.} Let $x$ be the first term and $d$ be the common difference. Then,
  $$x + (x + d) + \ldots + [x + (n - 1)d] = a \Rightarrow nx + \frac{d.(n - 1)n}{2} = a$$
  Squaring both sides of the above equation
  $$nx^2 + \frac{d^2(n - 1)^2n}{4} + n(n - 1)xd = \frac{a^2}{n}$$
  Also,
  $$x^2 + (x + d)^2 + \ldots + [x + (n - 1)d]^2 = b^2$$
  $$\Rightarrow nx^2 + d^2[1^2 + 2^2 + \ldots + (n - 1)^2] + 2xd[1 + 2 + 3 + \ldots + (n - 1)]= b^2$$
  $$\Rightarrow nx^2 + d^2\frac{(n - 1)n(2n - 1)}{6} + 2xd\frac{n(n - 1)}{2} = b^2$$
  Subtracting the two obtained equations we get
  $$d^2\frac{n(n - 1)(n + 1)}{12} = \frac{nb^2 - a^2}{n}\Rightarrow d = \pm \frac{2\sqrt{3(nb^2 - a^2)}}{n\sqrt{n^2 - 1}}$$
  $$\Rightarrow x = \frac{1}{n}\left[a\mp \frac{-(n - 1)\sqrt{3(nb^2 - a^2)}}{\sqrt{n^2 - 1}}\right]$$
\end{frame}
\begin{frame}{Problem 15}
  \textbf{15.} If $a_1, a_2, \ldots, a_n$ are in A.P., whose common diference is $d$, then find the sum of the series $$\sin
  d[\csc a_1\csc a_2 + \csc a_2\csc a_3 + \ldots + \csc a_{n - 1}\csc a_n]$$
\end{frame}
\begin{frame}{Solution of Problem 15}
  \textbf{Solution:}$$t_1 = \sin d(\csc a_1\csc a_2) = \frac{\sin(a_2 - a_1)}{\sin a_1\sin a_2} = \cot a_1 - \cot a_2$$
  $$t_2 = \cot a_2 - \cot a_3$$
  $$\ldots$$
  $$t_{n - 1} = \cot a_{n - 1} - \cot a_n$$

  Adding, we get $\sin d[\csc a_1\csc a_2 + \csc a_2\csc a_3 + \ldots + \csc a_{n - 1}\csc a_n] = \cot a_1 - \cot a_n$
\end{frame}
\begin{frame}{Problem 16}
  \textbf{16.} If $a_1, a_2, \ldots, a_n$ are in A.P. where $a_i > 0~\forall i,$ show that $$\frac{1}{\sqrt{a_1} + \sqrt{a_2}} +
  \frac{1}{\sqrt{a_2} + \sqrt{a_3}} + \ldots + \frac{1}{\sqrt{a_{n - 1}} + \sqrt{a_n}} = \frac{n - 1}{\sqrt{a_1} + \sqrt{a_n}}$$
\end{frame}
\begin{frame}{Solution of Problem 16}
  \textbf{Solution:} $$t_1 = \frac{1}{\sqrt{a_1} + \sqrt{a_2}} = \frac{\sqrt{a_2} - \sqrt{a_1}}{a_2 - a_1} =
  \frac{1}{d}(\sqrt{a_2} - \sqrt{a_1})$$
  $$t_2 = \frac{1}{d}(\sqrt{a_3} - \sqrt{a_2})$$
  \ldots
  $$t_{n - 1} = \frac{1}{d}(\sqrt{a_n} - a_{n - 1})$$

  Adding, we get $$S = \frac{1}{d}(\sqrt{a_n} - \sqrt{a_1}) = \frac{1}{d}\frac{a_n - a_n}{\sqrt{a_1} + \sqrt{a_n}} = \frac{n -
    1}{\sqrt{a_1} + \sqrt{a_n}}$$
\end{frame}
\begin{frame}{Problem 17}
  \textbf{17.} If $a_1, a_2, \ldots, a_n$ are in A.P., whose common differemce is $d$ show that $\sum_{2}^n\tan^{-1}\frac{d}{1 +
    a_{n - 1}a_n} = \tan^{-1}\frac{a_n - a_n}{1 + a_na_1}$
\end{frame}
\begin{frame}{Solution of Problem 17}
  \textbf{Solution:} We have to prove that $\tan^{-1}\frac{d}{1 + a_1a_2} + \tan^{-1}\frac{d}{1 + a_2a_3} + \ldots +
  \tan^{-1}\frac{d}{1 + a_{n - 1}a_n} = \tan^{-1}\frac{a_n - a_n}{1 + a_na_1}$
  $$t_1 = \tan^{-1}\frac{d}{1 + a_1a_2} = \tan^{-1}\frac{a_2 - a_1}{1 + a_1a_2} = \tan^{-1}a_2 - \tan^{-1}a_1$$
  $$t_2 = \tan^{-1}\frac{d}{1 + a_2a_3} = \tan^{-1}a_3 - \tan^{-1}a_2$$
  $$\ldots$$
  $$t_{n - 1} = \tan^{-1}\frac{d}{1 + a_{n - 1}a_n} = \tan^{-1}a_{n} - \tan^2{-1}a_{n - 1}$$
  Adding, we get
  $$\tan^{-1}\frac{d}{1 + a_1a_2} + \tan^{-1}\frac{d}{1 + a_2a_3} + \ldots + \tan^{-1}\frac{d}{1 + a_{n - 1}a_n} = \tan^{-1}a_n -
  \tan^{-1}a_1 = \tan^{-1}\frac{a_n - a_1}{1 + a_1a_n}$$
\end{frame}
\begin{frame}{Problem 18}
  \textbf{18.} If $a_1, a_2, \ldots, a_n$ are the first $n$ items of an A.P. with first term $a$ and common difference $d$ such
  that $ad > 0.$ Let $S_n = \frac{1}{a_1a_2} + \frac{1}{a_2a_3} - \ldots + \frac{1}{a_{n - 1}a_n}$ Prove that the product
  $a_1a_nS_n$ does not depend on $a$ or $d.$
\end{frame}
\begin{frame}{Solution of Problem 18}
  \textbf{Solution:} $$\frac{1}{a_1a_2} = \frac{a_2 - a_1}{da_1a_2} = \frac{1}{d}\left(\frac{1}{a_1} - \frac{1}{a_2}\right)$$
  $$\frac{1}{a_2a_3} = \frac{1}{d}\left(\frac{1}{a_2} - \frac{1}{a_3}\right)$$
  $$\ldots$$
  $$\frac{1}{a_{n - 1}a_n} = \frac{1}{d}\left(\frac{1}{a_{n - 1}} - \frac{1}{a_n}\right)$$
  Adding, we get
  $$S_n = \frac{1}{d}\left(\frac{1}{a_1} - \frac{1}{a_n}\right) = \frac{n - 1}{a_1a_n}$$
  $\therefore a_1a_nS_n = n - 1$ which is independent of $a$ and $d.$
\end{frame}
\begin{frame}{Problem 19}
  \textbf{19.} If $a_1, a_2, \ldots, a_n, a_{n + 1}, \ldots$ be in A.P., whose common difference is $d$ and $S_1 = a_1 + a_2 +
  \ldots + a_n,$ $S_2 = a_{n + 1} + \ldots + a_{2n}, S_3 = a_{2n + 1} + \ldots + a_{3n}$ Show that $S_1, S_2, S_3, \ldots$ are in
  A.P. whose common difference is $n^2d$
\end{frame}
\begin{frame}{Solution of Problem 19}
  \textbf{Solution:}$$S_2 - S_1 = \frac{n}{2}[2a_{n + 1} + (n - 1)d - 2a_1 - (n - 1)d] = \frac{n}{2}[2(a_1 + nd) - 2a_1] = n^2d$$
  Similarly, $$S_3 - S_2 = S_4 - S_3 = \ldots = n^2d$$
\end{frame}
\begin{frame}{Problem 20}
  \textbf{20.} If $a, b, c$ are three terms of an A.P. such that $a\neq b,$ show that $(b - c)/(a - b)$ is a rational number.
\end{frame}
\begin{frame}{Solution of Problem 20}
  \textbf{Solution:} Let $x$ be the first term and $y$ be the common difference. Also, let $a, b, c$ to be the $p$th, $q$th, $r$th
  term of the A.P.
  \linebreak\linebreak
  $$a = x + (p - 1)y, b = x + (q - 1)y, c = x + (r - 1)y$$
  $$(b - c)/(a - b) = \frac{q - r}{p - q}$$
  Since $p, q, r$ are integers $\frac{q - r}{p - q}$ will be a rational number.
\end{frame}
\end{document}
