\documentclass[aspectratio=1610,8pt]{beamer}

% Standard packages

\usepackage[english]{babel}
%\usepackage[latin1]{inputenc}
%\usepackage{times}
%\usepackage[T1]{fontenc}
\usepackage{fontspec}
\usepackage[]{unicode-math}
\setmathfont{Inconsolata}
\setsansfont{Roboto}

% Setup TikZ

\usepackage{tikz}
\usetikzlibrary{arrows}
\tikzstyle{block}=[draw opacity=0.7,line width=1.4cm]


% Author, Title, etc.

\title{Miscellaneous Problems on A.P., G.P. and H.P.\\Problems 121-130}

\author[Shiv Shankar Dayal]{Shiv Shankar Dayal}

% The main document

\begin{document}
\begin{frame}
  \titlepage
\end{frame}
\begin{frame}{Problem 121}
  \textbf{121.} The sum of first ten terms of an A.P. is equal to $155,$ and the sum of first two terms of a G.P. is $9.$ Find
  these progressions if the first term of the A.P. euqals the common ratio of the G.P. and the first term of G.P. equals the common
  difference of A.P.
\end{frame}
\begin{frame}{Solution of Problem 121}
  \textbf{Solution:} Let $a$ be the first term and $d$ be the common difference of A.P. and thus $d$ will be the first term and $a$
  be the common ratio of the G.P. Given,
  $$155 = \frac{10}{2}[2a + (10 - 1)d] \Rightarrow 2a + 9d = 31$$
  $$d + ad = 9$$
  $$\Rightarrow a = \frac{25}{2}, 2 \Rightarrow d = \frac{2}{3}, 3$$
  Thus, A.P. is $2, 5, 8, \ldots$ or $\frac{25}{2}, \frac{79}{6}, \frac{83}{6}, \ldots$ and the G.P. is $3, 6, 12, \ldots$ or
  $\frac{2}{3}, \frac{25}{3}, \frac{625}{6}, \ldots$
\end{frame}
\begin{frame}{Problem 122}
  \textbf{122.} If $a, b, c$ be in H.P., prove that $\left(\frac{1}{a} + \frac{1}{b} - \frac{1}{c}\right)\left(\frac{1}{b} +
  \frac{1}{c} - \frac{1}{a}\right) = \frac{4}{ac} - \frac{3}{b^2}$
\end{frame}
\begin{frame}{Solution of Problem 122}
  \textbf{Solution:} Since $a, b, c$ are in H.P. therefore $\frac{1}{a}, \frac{1}{b}, \frac{1}{c}$ are in A.P.
  \linebreak\linebreak
  $\Rightarrow \frac{2}{b} = \frac{1}{a} + \frac{1}{c} \Rightarrow b = \frac{2ac}{a + c} \Rightarrow \frac{3}{b} - \frac{2}{c} =
  \frac{1}{a} + \frac{1}{b} - \frac{1}{c} \text{~and~}\frac{3}{b} - \frac{2}{a} = \frac{1}{b} + \frac{1}{c} - \frac{1}{a}$
  $$\left(\frac{1}{a} + \frac{1}{b} - \frac{1}{c}\right)\left(\frac{1}{b} + \frac{1}{c} - \frac{1}{a}\right) = \left(\frac{3}{b} -
  \frac{2}{c}\right)\left(\frac{3}{b} - \frac{2}{a}\right)$$
  $$= \frac{9ac - 6ab - 6bc + 4b^2}{acb^2} = \frac{4}{ac} + \frac{9}{b^2} - \frac{6b(a + c)}{acb^2}$$
  $$= \frac{4}{ac} + \frac{9}{b^2} - \frac{6b}{acb^2}.\frac{2}{b}$$
  $$= \frac{4}{ac} - \frac{3}{b^2}$$
\end{frame}
\begin{frame}{Problem 123}
  \textbf{123.} If $a, b, c$ are positive real numbers which are in H.P. show that $\frac{a + b}{2a - b} + \frac{b + c}{2c - b}\geq
  4$
\end{frame}
\begin{frame}{Solution of Problem 123}
  \textbf{Solution:} Because $a, b, c$ are in H.P. therefore $\frac{2}{b} = \frac{1}{a} + \frac{1}{c}$
  $$\frac{a + b}{2a - b} + \frac{b + c}{2c - b} = \frac{\frac{1}{b} + \frac{1}{a}}{\frac{2}{b} - \frac{1}{a}} + \frac{\frac{1}{b} +
    \frac{1}{c}}{\frac{2}{b} - \frac{1}{c}}$$
  $$= \frac{c}{a} + \frac{c}{b} + \frac{a}{b} + \frac{a}{c} = \frac{c^2 + a^2}{ac} + \frac{a + c}{b}$$
  $$= \frac{c^2 + a^2}{ac} + \frac{(a + c)^2}{2ac} = \frac{c^2 + a^2}{ac} - 2 + \frac{(a + c)^2}{2ac} - 2 + 4$$
  $$= \frac{(c - a)^2}{ac} + \frac{(a - c)^2}{2ac} + 4 \geq 4$$
\end{frame}
\begin{frame}{Problem 124}
  \textbf{124.} If $(a + b)/(1 - ab), b, (b + c)/(1 - bc)$ are in A.P., then prove that $a, b^{-1}, c$ are in H.P.
\end{frame}
\begin{frame}{Solution of Problem 124}
  \textbf{Solution:}$$b - \frac{a + b}{1 - ab} = \frac{b + c}{1 - bc} - b$$
  $$\Rightarrow \frac{b - ab^2 - a - b}{1 - ab} = \frac{b + c - b + b^2c}{1 - bc}$$
  $$\Rightarrow \frac{-a(1 + b^2)}{1 - ab} = \frac{c(1 + b^2)}{1 - bc}\Rightarrow -a(1 - bc) = c(1 - ab)$$
  $$\Rightarrow a + c = 2abc \Rightarrow 2b = \frac{a + c}{ac}$$
  $\therefore a, b^{-1}, c$ are in H.P.
\end{frame}
\begin{frame}{Problem 125}
  \textbf{125.} Suppose $a, b, c$ are in A.P. and $|a|, |b|, |c| < 1$ if $x = 1 + a + a^2 + \ldots \infty, y = 1 + b + b^2 + \ldots
  \infty,$ $z = 1 + c + c^2 + \ldots \infty$ then prove that $x, y, z$ are in H.P.
\end{frame}
\begin{frame}{Solution of Problem 125}
  \textbf{Solution:} $$x = \frac{1}{1 - a}, y = \frac{1}{1 - b}, z = \frac{1}{1 - c}$$
  $$a, b, c\text{~are in A.P.}$$
  $$\Rightarrow 1 - a, 1 - b, 1 - c\text{~are in A.P.}$$
  $$\Rightarrow \frac{1}{1 - a}, \frac{1}{1 - b}, \frac{1}{1 - c}\text{~are in H.P.}$$
  $$\Rightarrow x, y, z\text{~are in H.P.}$$
\end{frame}
\begin{frame}{Problem 126}
  \textbf{126.} If $a^{\frac{1}{x}} = b^{\frac{1}{y}} = c^{\frac{1}{z}}$ and $a, b, c$ are in G.P. prove that $x, y, z$ are in A.P.
\end{frame}
\begin{frame}{Solution of Problem 126}
  \textbf{Solution:} Let $$a^{\frac{1}{x}} = b^{\frac{1}{y}} = c^{\frac{1}{z}} = k$$
  $$\Rightarrow a = k^x, b = k^y, c = k^z$$
  $$\because a, b, c \text{~are in G.P.} \Rightarrow b^2 = ac \Rightarrow k^{2y} = k^{x + z} \Rightarrow 2y = x + z$$
  $\therefore x, y, z$ are in A.P.
\end{frame}
\begin{frame}{Problem 127}
  \textbf{127.} If $a, b, c$ be in A.P., $l, m, n$ be in H.P. and $al, bm, cn$ be in G.P. with common ratio not equal to $1$ and
  $a, b, c, l, m, n$ are positive show that $a:b:c = \frac{1}{n}:\frac{1}{m}:\frac{1}{l}$
\end{frame}
\begin{frame}{Solution of Problem 127}
  \textbf{Solution:} $$2b = a + c, m = \frac{2ln}{l + n}, b^2m^2 = acln$$
  $$\Rightarrow \left(\frac{a + c}{2}.\frac{2ln}{l + n}\right)^2 = acln$$
  $$\Rightarrow \frac{ln}{(l + n)^2} = \frac{ac}{(a + c)^2}$$
  $$\Rightarrow \frac{(a + c)^2}{ac} = \frac{(l + n)^2}{ln}$$
  $$\Rightarrow \frac{a}{c} + \frac{c}{a} = \frac{l}{n} + \frac{n}{l}$$
  $$\Rightarrow a:c = \frac{1}{n}:\frac{1}{l}$$
  Now it can be proven that $a:b:c = \frac{1}{n}:\frac{1}{m}:\frac{1}{l}$
\end{frame}
\begin{frame}{Problem 128}
  \textbf{128:} Find three numbers $a, b, c$ between $2$ and $18$ such that their sum is $25$, the numbers $2, a, b$ are
  consecutive terms of an A.P. and the numbers $b, c, 18$ are consecutive terms of a G.P.
\end{frame}
\begin{frame}{Solution of Problem 128}
  \textbf{Solution:} $$a + b + c = 25, 2a = 2 + b, c^2 = 18b$$
  $$\Rightarrow b = 2(a - 1), c = \sqrt{18b} = \sqrt{36(a - 1)} = 6\sqrt{a - 1}$$
  $$\Rightarrow a + 2(a - 1) + 6\sqrt{a - 1} = 25\Rightarrow a + 2\sqrt{a - 1} = 9$$
  $$\Rightarrow a = 17, 5 \text{~however if~}a = 17\text{~then~}b = 32 > 18$$
  $$\therefore a = 5, b = 8, c = 12$$
\end{frame}
\begin{frame}{Problem 129}
  \textbf{129.} If $a, b, c$ are in A.P. and $a, mb, c$ are in G.P.; prove that $a, m^2b, c$ are in H.P.
\end{frame}
\begin{frame}{Solution of Problem 129}
  \textbf{Solution:} $$2b = a + c, m^2b^2 = ac$$
  $$m^2b = \frac{ac}{b} = \frac{2ac}{a + c}$$
  $$\Rightarrow a, m^2b, c\text{~are in H.P.}$$
\end{frame}
\begin{frame}{Problem 130}
  \textbf{130.} An A.P., a G.P. and an H.P. have the same first term $a$ abd same second term $b,$ show that $n + 2$th terms will
  be in G.P. is $\frac{b^{2n + 2} - a^{2n + 2}}{ab(b^{2n} - a^{2n})} = \frac{n + 1}{n}$
\end{frame}
\begin{frame}{Solution of Problem 130}
  \textbf{Solution:} The common difference of A.P. $= b - a,$ common ratio of G.P. is $b/a$ and common difference for corresponding
  A.P. of H.P. is $(a - b)/ab$
  \linebreak\linebreak
  $n + 2$th term of A.P. $= a + (n + 1)(b - a) = (n + 1)b - na$
  \linebreak\linebreak
  $n + 2$th term of G.P. $= ar^{n + 1} = \frac{b^{n + 1}}{a^n}$
  \linebreak\linebreak
  $n + 2$th term of H.P. $= \frac{1}{\frac{1}{a} + \frac{(n + 1)(a - b)}{ab}} = \frac{ab}{(n + 1)a - nb}$
  \linebreak\linebreak
  These will be in G.P. if
  $$\frac{[(n + 1)b - na]ab}{(n + 1)a - nb} = \frac{b^{2n + 2}}{a^{2n}}$$
  $$(n + 1)a^{2n + 1}b^2 - na^{2n + 2}b = (n + 1)ab^{2n + 2} - nb.b^{2n + 2}$$
  $$(n + 1)ab^2[a^{2n} - b^{2n}] = nb[a^{2n + 2} - b^{2n + 2}]$$
  $$\frac{b^{2n + 2} - a^{2n + 2}}{ab(b^{2n} - a^{2n})} = \frac{n + 1}{n}$$
\end{frame}
\end{document}
