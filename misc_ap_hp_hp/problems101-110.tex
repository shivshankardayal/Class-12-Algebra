\documentclass[aspectratio=1610,8pt]{beamer}

% Standard packages

\usepackage[english]{babel}
%\usepackage[latin1]{inputenc}
%\usepackage{times}
%\usepackage[T1]{fontenc}
\usepackage{fontspec}
\usepackage[]{unicode-math}
\setmathfont{Inconsolata}
\setsansfont{Roboto}

% Setup TikZ

\usepackage{tikz}
\usetikzlibrary{arrows}
\tikzstyle{block}=[draw opacity=0.7,line width=1.4cm]


% Author, Title, etc.

\title{Miscellaneous Problems on A.P., G.P. and H.P.\\Problems 101-110}

\author[Shiv Shankar Dayal]{Shiv Shankar Dayal}

% The main document

\begin{document}
\begin{frame}
  \titlepage
\end{frame}
\begin{frame}{Problem 101}
  \textbf{101.} If $S_n$ be the sum of infinite G.P.'s whose first term is $n$ and the common ratio is $\frac{1}{n + 1},$ find
  $\lim_{n\to \infty} \frac{S_1S_n + S_2S_{n - 1} + \ldots + S_nS_1}{S_1^2 + S_2^2 + \ldots + S_n^2}$
\end{frame}
\begin{frame}{Solution of Problem 101}
  \textbf{Solution:} $$S_1 = \frac{1}{1 - \frac{1}{2}} = 2, S_2 = \frac{2}{1 - \frac{1}{3}} = 3,\ldots S_n = \frac{n}{1 -
    \frac{1}{n + 1}} = n + 1$$
  $$\text{General term of numerator~}t_i = S_iS_{n - i + 1} = (i + 1)(n - i + 2) = (n + 1)i - i^2 + (n + 1)$$
  $$\therefore \text{Sum for numerator~} = \sum_{i=1}^nt_i = \sum[(n + 1)i - i^2 + (n + 1)] = \frac{n(n + 1)^2}{2} - \frac{n(n + 1)(2n
    + 1)}{6} + n(n + 1)$$
  $$\text{Sum for denominator~}= 1^2 + 2^2 + \ldots + (n + 1)^2 - 1 = \frac{(n + 1)(n + 2)(2n + 3)}{6} - 1$$
  Upon simplification $$\lim_{n\to \infty} \frac{S_1S_n + S_2S_{n - 1} + \ldots + S_nS_1}{S_1^2 + S_2^2 + \ldots + S_n^2} = \frac{1}{2}$$
\end{frame}
\begin{frame}{Problem 102}
  \textbf{102.} The sum of the terms of an infinitely decreasing G.P. is equal to the greatest value of the function $f(x) = x^3 +
  3x - 9$ on the interval $[-5, 3],$ and the difference between the first and second terms is $f'(0).$ Prove that the common ratio
  of the progression is $\frac{2}{3}.$
\end{frame}
\begin{frame}{Solution of Problem 102}
  \textbf{Solution:} $f'(x) = 3x^2 + 3$ which yields imaginary roots implying that there is no local maxima. However, $3x^2 + 3$ is
  positive for all values of $x$ which means that $f(x)$ is monotonically increasing in $[-5, 3]$ implying that maximum value will
  be at $x = 3$
  \linebreak\linebreak
  $f(3) = 27,$ also let $a$ to be the first term and $r$ to be the common ratio then given, $a - ar = f'(0) = 3.$ The sum is given
  as $\frac{a}{1 - r} = 27$ solving these yields $r = \frac{2}{3}, -\frac{4}{3}$ but the series is decreasing so $r = \frac{2}{3}$
\end{frame}
\begin{frame}{Problem 103}
  \textbf{103.} Find the sum of the series $\frac{5}{13} + \frac{55}{13^2} + \frac{555}{13^3} + \ldots \infty$
\end{frame}
\begin{frame}{Solution of Problem 103}
  \textbf{Solution:} $$\text{Let~}S = \frac{5}{13} + \frac{55}{13^2} + \frac{555}{13^3} + \ldots \infty$$
  $$= \frac{5}{9}\left[\frac{10 - 1}{13} + \frac{100 - 1}{13^2} + \frac{1000 - 1}{13^3} + \ldots \infty\right]$$
  $$= \frac{5}{9}\left[\frac{10}{13} + \frac{10^2}{13^2} + \frac{10^3}{13^3} + \ldots \infty - \frac{1}{13} - \frac{1}{13^2} -
    \frac{1}{13^3} - \ldots \infty\right]$$
  $$= \frac{5}{9}\left[\frac{\frac{10}{13}}{1 - \frac{10}{13}} - \frac{\frac{1}{13}}{1 - \frac{1}{13}}\right]$$
  $$= \frac{5}{9}\left[\frac{10}{13}.\frac{13}{3} - \frac{1}{13}.\frac{13}{12}\right]$$
  $$= \frac{65}{36}$$
\end{frame}
\begin{frame}{Problem 104}
  \textbf{104.} If $e^{{\sin^2x + \sin^4x + \sin^6x + \ldots \infty}\log_e2}$ satisfies the equation $x^2 - 9x + 8 = 0,$ find the
  value of $\frac{\cos x}{\cos x + \sin x}, 0 < x < \frac{\pi}{2}$
\end{frame}
\begin{frame}{Solution of Problem 104}
  \textbf{Solution:}$$e^{{\sin^2x + \sin^4x + \sin^6x + \ldots \infty}\log_e2} = 2^{\frac{\sin^2x}{1 - \sin^2x}\log_e e} =
  2^{\tan^2x}$$
  $$x^2 - 9x + 8 = 0 \Rightarrow (x - 1)(x - 8) = 0\Rightarrow x = 1 = 2^0, x = 8 = 2^3$$
  $$\therefore 2^{\tan^2x} = 2^0, 2^3~\because 0 < x < \frac{\pi}{2}\Rightarrow  \tan x= \sqrt{3} \Rightarrow x = \frac{\pi}{3}$$
  $$\frac{\cos x}{\cos x + \sin x} = \frac{\frac{1}{2}}{\frac{1}{2} + \frac{\sqrt{3}}{2}} = \frac{1}{1 + \sqrt{3}}$$
\end{frame}
\begin{frame}{Problem 105}
  \textbf{105.} If $-\frac{\pi}{2} < x < \frac{\pi}{2}$ and the sum to infinite number of terms of series $\cos x + \frac{2}{3}\cos
  x\sin^2x + \frac{4}{9}\cos x\sin^4x + \ldots$ is finite, then show that $x$ lies in the set $\left(-\frac{\pi}{2},
  \frac{\pi}{2}\right)$
\end{frame}
\begin{frame}{Solution of Problem 105}
  \textbf{Solution:} $$S = \cos x + \frac{2}{3}\cos x\sin^2x + \frac{4}{9}\cos x\sin^4x + \ldots$$
  $$= \frac{\cos x}{1 - \frac{2}{3}\sin^2x} = \frac{3\cos x}{3 - 2\sin^2x} = \frac{3\cos x}{2 + \cos 2x}$$
  The term $\frac{3\cos x}{2 + \cos 2x}$ is finite for all $x\in \left(-\frac{\pi}{2},\frac{\pi}{2}\right)$
\end{frame}
\begin{frame}{Problem 106}
  \textbf{106.} Suppose $0 < x < \pi$ and the expression $e^{{1 + |\cos x| + \cos^2x + |\cos^3x| + \ldots \infty}\log_e 4}$
  satisfies the quadratic equation $y^2 - 20y + 64 = 0,$ then find the value of $x.$
\end{frame}
\begin{frame}{Solution of Problem 106}
  \textbf{Solution:} $$e^{{1 + |\cos x| + \cos^2x + |\cos^3x| + \ldots \infty}\log_e 4} = 4^{{\frac{1}{1 - |\cos x|}}\log_e e} =
  4^{\frac{1}{1 - |\cos x|}}$$
  $$y^2 - 20y + 64 = 0 \Rightarrow (y - 16)(y - 4) = 0 \Rightarrow y = 4^1, 4^2\Rightarrow \frac{1}{1 - |\cos x|} = 1, 2$$
  $$\text{If~}\frac{1}{1 - |\cos x|} = 1 \text{~then~} |\cos x| = 0 \Rightarrow x\notin(0, \pi)$$
  $$\therefore 1 - |\cos x| = \frac{1}{2} \Rightarrow x = n\pi + \frac{\pi}{3}, n \in I$$
\end{frame}
\begin{frame}{Problem 107}
  \textbf{107.} An A.P. and a G.P. with positive terms have the same number of terms and their first terms as well as the last
  terms are equal. Show that the sum of A.P. is greater than or equal to the sum of the G.P.
\end{frame}
\begin{frame}{Solution of Problem 107}
  \textbf{Solution:} Let $a$ be the first term, $b$ be the last term and $n$ be the number of terms of A.P. and G.P.
  \linebreak\linebreak
  Then c.d. of A.P. $= \frac{b - a}{n - 1}$ and c.r. of the G.P. $= \left(\frac{b}{a}\right)^{n - 1}.$ Let $S$ be the sum of $n$
  terms of A.P. and $S'$ the sum of $n$ terms of G.P. then $S = \frac{n}{2}(a + b)$
  \linebreak\linebreak
  $$S' = a(1 + r + r^2 + \ldots + r^{n - 1})$$
  $$S' = a(r^{n - 1} + r^{n - 2} + \ldots + 1)$$
  $$\therefore S' = \frac{a}{2}[(1 + r^{n - 1}) + (r + r^{n - 2}) + (r^k + r^{n - k - 1}) + \ldots + (r^{n - 1} + 1)]$$
  Now, $$(r^k + r^{n - k - 1}) - (r^{n - 1} + 1) = (r^k - 1) + r^{n - 1}(r^{-k} - 1)$$
  $$= (r^k - 1)\left(1 - \frac{r^{n - 1}}{r^k}\right) = (r^k - 1)(1 - r^{n - k - 1})\leq 0$$
  $$\therefore S'\leq \frac{an}{2}(1 + r^{n - 1}) = \frac{an}{2}\left(1 + \frac{b}{a}\right) = \left(\frac{a + b}{2}\right)n = S$$
  $$\therefore S \geq S'$$
\end{frame}
\begin{frame}{Problem 108}
  \textbf{108.} Given a G.P. and A.P. of positive terms $a, a_1, a_2, \ldots, a_n, \ldots$ and $b, b_1, b_2, \ldots, b_n, \ldots$
  respectively, with the common ratio of the G.P. being different from $1,$ prove that there exists $x\in R, x > 0$ such that
  $\log_x a_n - b_n = \log_x a - b,~\forall n\in N.$
\end{frame}
\begin{frame}{Solution of Problem 108}
  \textbf{Solution:} Given $a, a_1, a_2, a_3, \ldots$ are in G.P. so $\log a, \log a_1, \log a_2, \ldots$ are in A.P. Let the
  common difference of this A.P. be $d_1.$ Now $\log a_n = \log a + nd_1.$ Further if $d$ be the common difference of the A.P. $b,
  b_1, b_2, \ldots$ then $b_n = b + nd$
  $$\therefore \frac{\log a_n - \log a}{b_n - b} = \frac{nd_1}{nd} = \frac{d_1}{d}$$
  Let $\log x = \frac{d_1}{d}$ for a fixed positive real number $x.$
  $$\Rightarrow \frac{\log a_n - \log a}{b_n - b} = \log x \Rightarrow b_n - b = \log_x\left(\frac{a_n}{a}\right)$$
  $$\Rightarrow \log_x a_n - \log_x a = b_n - b \Rightarrow \log_x a_n - b_n = \log_x a - b$$
\end{frame}
\begin{frame}{Problem 109}
  \textbf{109.} If the $(m + 1)$th, $(n + 1)$th and $(r + 1)$th terms of an A.P. are in G.P., and $m, n, r$ are in H.P., show that
  the ratio of the first term to the common difference of the A.P. is $-n/2.$
\end{frame}
\begin{frame}{Solution of Problem 109}
  \textbf{Solution:} Given $a + md, a + nd, a + rd$ are in G.P., where $a$ is the first term and $d$ is the c.d. of A.P.
  $$\Rightarrow (a + nd)^2 = (a + md)(a + rd)$$
  $$\Rightarrow d(n^2d + 2an) = d(am + ar + mrd)\Rightarrow (n^2 - mr)d = a(m + r - rn)$$
  $$\frac{d}{a} = \frac{m + r - 2n}{n^2 - mr}$$
  Given, $m, n, r$ are in H.P. $\therefore n = \frac{2mr}{m + r} \Rightarrow m + r = \frac{2mr}{n}$
  $$\therefore \frac{d}{a} = \frac{\frac{2mr}{n} - 2n}{n^2 - mr} = -\frac{2}{n}~\therefore~ \frac{a}{d} = -\frac{n}{2}$$
\end{frame}
\begin{frame}{Problem 110}
  \textbf{110.} If $a, b, c$ are in G.P. and $a - b, c - a, b - c$ are in H.P., then show that $a + 4b + c = 0$
\end{frame}
\begin{frame}{Solution to Problem 110}
  \textbf{Solution:} Let $r$ be the common ratio of the G.P., then $b = ar, c = ar^2.$ Given, $a - b, c - a, b - c$ are in H.P.
  $$\therefore c - a = \frac{2(a - b)(b - c)}{a - b + b - c}$$
  $$(c - a)^2 = 2(a - b)(b - c)\Rightarrow (ar^2 - a)^2 = 2(a - ar)(ar - ar^2)$$
  $$a^2(r^2 - 1)^2 = -2a^2(1 - r)r(1 - r)\Rightarrow (r + 1)^2 = -2r \Rightarrow 1 + 4r + r^2 = 0$$
  $$\Rightarrow a + 4ar + ar^2 = 0 \Rightarrow a + 4b + c = 0$$
\end{frame}
\end{document}
