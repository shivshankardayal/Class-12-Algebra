\documentclass[aspectratio=1610,8pt]{beamer}

% Standard packages

\usepackage[english]{babel}
%\usepackage[latin1]{inputenc}
%\usepackage{times}
%\usepackage[T1]{fontenc}
\usepackage{fontspec}
\usepackage[]{unicode-math}
\setmathfont{Inconsolata}
\setsansfont{Roboto}

% Setup TikZ

\usepackage{tikz}
\usetikzlibrary{arrows}
\tikzstyle{block}=[draw opacity=0.7,line width=1.4cm]


% Author, Title, etc.

\title{Miscellaneous Problems on A.P., G.P. and H.P.\\Problems 131-140}

\author[Shiv Shankar Dayal]{Shiv Shankar Dayal}

% The main document

\begin{document}
\begin{frame}
  \titlepage
\end{frame}
\begin{frame}{Problem 131}
  \textbf{131.} The second, third and sixth terms of an A.P. are consecutive terms of a geometric progression. Find the common
  ratio of the G.P.
\end{frame}
\begin{frame}{Solution of Problem 131}
  \textbf{Solution:} Let $a$ be the first term and $d$ be the common difference of A.P., then
  $$(a + 2d)^2 = (a + d)(a + 5d)\Rightarrow a^2 + 4ad + 4d^2 = a^2 + 6ad + 5d^2$$
  $$\Rightarrow d^2 + 2ad = 0 \Rightarrow d = -2a$$
  Hence, common ratio of the G.P. $= \frac{a + 2d}{a + d} = \frac{-3a}{-a} = 3$
\end{frame}
\begin{frame}{Problem 132}
  \textbf{132.} A sequence of numbers is formed by adding together corresponding terms of an A.P. and a G.P. with common ratio $2.$
  The first term of the sequence is $57,$ the second term is $94$ and the third term is $171.$ Find the fourth term. Find also an
  expression for the $n$th term of the sequence.
\end{frame}
\begin{frame}{Solution of Problem 132}
  \textbf{Solution:} Let $a$ be the first term and $d$ be the common difference of A.P. Also, let $a'$ to be the first term of
  G.P. Then, given
  $$a + a' = 57, a + d + 2a' = 94, a + 2d + 4a' = 171$$
  $$\Rightarrow 2a' + d = 94 - a \Rightarrow a + 2(94 - a) = 171 \Rightarrow a = 17$$
  $$a' = 40 \Rightarrow d = -3$$
  $$\Rightarrow t_4 = a + 3d + 8a' = 328$$
  $$\Rightarrow t_n = 17 + (n - 1).-3 + 40.2^{n - 1} = 20 - 3n + 10.2^{n + 1}$$
\end{frame}
\begin{frame}{Problem 133}
  \textbf{133.} The first, eighth and twenty second terms of an arithmetic progression are three consecutive terms of a geometric
  progression. Find the common ratio of the geometric progression. If sum of the first twenty two terms of arithmetic progression
  is $275,$ find its first term.
\end{frame}
\begin{frame}{Solution of Problem 133}
  \textbf{Solution:} Let $a$ be the first term and $d$ be the common ratio of the A.P. then given
  $$a(a + 21d) = (a + 7d)^2 \Rightarrow 49d^2 - 7ad = 0 \Rightarrow d = a/7[\because d\neq 0]$$
  $$\therefore r = \frac{a + 7d}{a} = 2$$
  Also, given $$S_{22} = \frac{22}{2}[2a + 21.d] = 275 \Rightarrow 5a = 25 \Rightarrow a = 5$$
\end{frame}
\begin{frame}{Problem 134}
  \textbf{134.} An arithmetic progression has common difference $2$ and a geometric progression has common ratio $2.$ A new
  sequence is formed by adding together the corresponding terms of these progressions. Given that the first term of this new
  sequence is $8$ and the fifth term is $91,$ find the first terms.
\end{frame}
\begin{frame}{Solution of Problem 134}
  \textbf{Solution:} Let $a$ and $a'$ be the first term of the A.P. and G.P. respectively. Then, given
  $$a + a' = 8, a + 8 + 16a' = 91$$
  $$\Rightarrow 15a' = 91 -16 = 75 \Rightarrow a' = 5 \Rightarrow a = 3$$
\end{frame}
\begin{frame}{Problem 135}
  \textbf{135.} If $a, b, c$ are in A.P. and $b, c, d$ are in H.P., prove that $ad = bc$
\end{frame}
\begin{frame}{Solution of Problem 135}
  \textbf{Solution:} $$2b = a + c, c = \frac{2bd}{b + d}$$
  $$\Rightarrow bc + cd = d(a + c)\Rightarrow bc = ad$$
\end{frame}
\begin{frame}{Problem 136}
  \textbf{136.} If $a, b, c$ are in H.P., $b, c, d$ are in G.P. and $c, d, e$ are in A.P., show that $e = \frac{ab^2}{(2a - b)^2}$
\end{frame}
\begin{frame}{Solution of Problem 136}
  \textbf{Solution:} $$b = \frac{2ac}{a + c}, c^2 = bd, 2d = c + e$$
  $$\Rightarrow ab + bc = 2ac \Rightarrow c = \frac{ab}{2a - b}$$
  $$e = 2d - c = \frac{2c^2}{b} - c = \frac{2a^2b^2}{b(2a - b)^2} - \frac{ab}{2a - b}$$
  $$= \frac{2a^2b - 2a^2b + ab^2}{(2a - b)^2} = \frac{ab^2}{(2a - b)^2}$$
\end{frame}
\begin{frame}{Problem 137}
  \textbf{137.} If an A.P. and a G.P. have the same 1st and 2nd terms then show that every other term of the A.P. will be less than
  the corresponding term of G.P. all the terms being positive.
\end{frame}
\begin{frame}{Solution of Problem 137}
  \textbf{Solution:} $$ar^n - a - nd = a\left(1 + \frac{d}{a}\right)^n - a - nd\left[\because r = \frac{a + d}{a}\right]$$
  $$= a\left[1 + {}^nC_1\left(\frac{d}{a}\right) + {}^nC_2\left(\frac{d}{a}\right)^2 + \ldots +
    {}^nC_n\left(\frac{d}{a}\right)^n\right] - a - nd$$
  $$= a\left[{}^nC_2\frac{d^2}{a^2} + {}^nC_3\frac{d^3}{a^3} + \ldots + {}^nC_n\frac{d^n}{a^n}\right] > 0 \left(\because
  \frac{d}{a}> 0\right)$$
\end{frame}
\begin{frame}{Problem 138}
  \textbf{138.} If three unequal numbers are in H.P. and their squares are in A.P. show that they are in the ratio $1 +
  \sqrt{3}:-2:1 - \sqrt{3}$ or $1 - \sqrt{3}:-2:1 + \sqrt{3}$
\end{frame}
\begin{frame}{Solution of Problem 138}
  \textbf{Solution:} Let $a, b, c$ be three numbers in H.P. Then, given that
  $$b = \frac{2ac}{a + c}, 2b^2 = a^2 + c^2$$
  Let $a = ck$
  $$\Rightarrow \frac{8a^2c^2}{(a + c)^2} = a^2 + c^2$$
  $$8a^2c^2 = (a^2 + c^2)(a + c)^2\Rightarrow (1 + k^2)(1 + k)^2 - 8k^2 = 0$$
  $$(k - 1)^2(k^2 + 4k + 1) = 0$$
  $$\because k\neq 1 k = -2\pm 3$$
  So $a:c = 1:-2\pm 3$ and now the ratio for $b$ can be found.
\end{frame}
\begin{frame}{Problem 139}
  \textbf{139.} If $A, G, H$ are the arithmetic, geometric and harmonic means of two positive real numbers $a$ and $b$, and if $A =
  kh,$ prove that $A^2 = kG^2.$ Find the ratio of $a$ to $b.$ For what value of $k$ does the ratio exist.
\end{frame}
\begin{frame}{Solution of Problem 139}
  \textbf{Solution:} $$A = \frac{a + b}{2}, H=\frac{2ab}{a + b}, G = \sqrt{ab}$$
  $$A = kH\Rightarrow (a + b)^2 = 4kab \Rightarrow A = kG^2$$
  Let $b = ma$
  $$\Rightarrow a^2(1 + m^2) = 4kma^2 \Rightarrow 1 + m^2 = 4km \Rightarrow m = \frac{4k \pm \sqrt{16k^2 - 4}}{2} = 2k\pm
  \sqrt{4k^2 - 1}$$
  Also, $(a + b)^2 = 4kab \Rightarrow (a - b)^2 = 4kab - 4ab\because (a - b)^2 \geq 0 \therefore k \geq 1$
\end{frame}
\begin{frame}{Problem 140}
  \textbf{140.} If $p$ be the $r$th term when $n$ A.M.'s are inserted between $a$ and $b$ and $q$ be the $r$th term when $n$ H.M.'s
  are inserted between $a$ and $b,$ then show that $\frac{p}{a} + \frac{b}{q}$ is independent of $n$ and $r.$
\end{frame}
\begin{frame}{Solution of Problem 140}
  \textbf{Solution:} Since $n$ means are inserted therefore total no. of terms will be $n + 2.$ Let $d$ be the c.d. of A.P. and
  $d'$ be the c.d of H.P.
  $$\Rightarrow d = \frac{b - a}{n + 1}, d' = \frac{a - b}{(n + 1)ab}$$
  $$\Rightarrow p = a + rd = \frac{(n + 1)a + r(b - a)}{n + 1}, \frac{1}{q} = \frac{1}{a} + r\frac{a - b}{(n + 1)ab} \Rightarrow q =
  \frac{(n + 1)ab}{r(a - b) + (n + 1)b}$$
  $$\frac{p}{a} + \frac{b}{q} = \frac{(n + 1)a + r(b - a)}{a(n + 1)} + \frac{r(a - b) + (n + 1)b}{(n + 1)a}$$
  $$= \frac{a + b}{a}$$
  which is independent of $n$ and $r$
\end{frame}
\end{document}
