\documentclass[aspectratio=1610,8pt]{beamer}

% Standard packages

\usepackage[english]{babel}
%\usepackage[latin1]{inputenc}
%\usepackage{times}
%\usepackage[T1]{fontenc}
\usepackage{fontspec}
\usepackage[]{unicode-math}
\setmathfont{Inconsolata}
\setsansfont{Roboto}

% Setup TikZ

\usepackage{tikz}
\usetikzlibrary{arrows}
\tikzstyle{block}=[draw opacity=0.7,line width=1.4cm]


% Author, Title, etc.

\title{Miscellaneous Problems on A.P., G.P. and H.P.\\Problems 1-10}

\author[Shiv Shankar Dayal]{Shiv Shankar Dayal}

% The main document

\begin{document}
\begin{frame}
  \titlepage
\end{frame}
\begin{frame}{Problem 1}
  \textbf{1.} If $a_1, a_2, a_3, \ldots, a_{2n}$ are in A.P., show that $a_1^2 - a_2^2 + a_3^2 - a_4^2 + \ldots + a_{2n - 1}^2 -
  a_{2n}^2 = \frac{n}{2n - 1}(a_1^2 - a_{2n}^2)$
\end{frame}
\begin{frame}{Solution of Problem 1}
  \textbf{Solution:} $$L.H.S. = (a_1^2 - a_2^2) + (a_3^2 - a_4^2) + \ldots + (a_{2n - 1}^2 - a_{2n}^2)$$
  $$= (a_1 - a_2)(a_1 + a_2) + (a_3 - a_4)(a_3 + a_4) + \ldots + (a_{2n - 1} - a_{2n})(a_{2n - 1} + a_{2n})$$
  $$= -d(a_1 + a_2 + a_3 + \ldots + a_{2n})$$
  $$= -\left(\frac{a_{2n} - a_2}{2n - 1}\right)\frac{2n}{2}(a_1 + a_{2n}) = \frac{n}{2n - 1}(a_1^2 - a_{2n}^2)$$
\end{frame}
\begin{frame}{Problem 2}
  \textbf{2.} If $\alpha_1, \alpha_2, \alpha_3, \ldots, \alpha_n$ are in A.P., whose common difference is $d$ show that $\sin
  d[\sec\alpha_1\sec\alpha_2 + \sec\alpha_2\sec\alpha_3$ $+ \ldots + \sec\alpha_{n - 1}\sec\alpha_n] = \tan\alpha_n - \tan\alpha_1$
\end{frame}
\begin{frame}{Solution of Problem 2}
  \textbf{Solution:} $t_1 = \sin d\sec\alpha_1\sec\alpha_2 = \frac{\sin(\alpha_2 - \alpha_2)}{\cos\alpha_1\cos\alpha_2}$

  $$=\frac{\sin\alpha_2\cos\alpha_1}{\cos\alpha_1\cos\alpha_2} - \frac{\cos\alpha_2\sin\alpha_1}{\cos\alpha_1\cos\alpha_2} =
  \tan\alpha_2 - \tan\alpha_1$$
  Similarly, $$t_2 = \tan\alpha_3 - \tan\alpha_2$$
  $$\ldots$$
  $$t_{n - 1} = \tan\alpha_n - \tan\alpha_{n - 1}$$

  Adding, we get $t_1 + t_2 + \ldots + t_{n - 1} = \tan\alpha_n - \tan\alpha_1$
\end{frame}
\begin{frame}{Problem 3}
  \textbf{3.} If $a_1, a_2, a_3, \ldots, a_n$ be in A.P., prove that $\frac{1}{a_aa_n} + \frac{1}{a_2a_{n - 1}} + \ldots +
  \frac{1}{a_na_1} = \frac{2}{a_1 + a_n}\left(\frac{1}{a_1} + \frac{1}{a_2} + \ldots + \frac{1}{a_n}\right)$
\end{frame}
\begin{frame}{Solution of Problem 3}
  \textbf{Solution:}$$L.H.S. = \frac{1}{a_1 + a_n}\left(\frac{a_1 + a_n}{a_1a_n} + \frac{a_1 + a_n}{a_2a_{n - 1}} + \ldots +
  \frac{a_1 + a_n}{a_na_1}\right)$$
  $$= \frac{1}{a_1 + a_n}\left(\frac{a_1 + a_n}{a_1a_n} + \frac{a_2 + a_{n - 1}}{a_2a_{n - 1}} + \ldots + \frac{a_n +
    a_1}{a_na_1}\right)$$
  $$= \frac{1}{a_1 + a_n}\left[\left(\frac{1}{a_n} + \frac{1}{a_1}\right) + \left(\frac{1}{a_2} + \frac{1}{a_{n - 1}}\right) +
    \ldots + \left(\frac{1}{a_n} + \frac{1}{a_1}\right)\right]$$
  $$= \frac{2}{a_1 + a_n}\left(\frac{1}{a_1} + \frac{1}{a_2} + \ldots + \frac{1}{a_n}\right)$$
\end{frame}
\begin{frame}{Problem 4}
  \textbf{4.} If $a_1, a_2, a_3, \ldots$ be in A.P. such that $a_i\neq 0,$ show that $S = \frac{1}{a_1a_2} + \frac{1}{a_2a_3}
  + \ldots + \frac{1}{a_na_{n + 1}} = \frac{n}{a_1a_{n + 1}}$
\end{frame}
\begin{frame}{Solution of Problem 4}
  \textbf{Solution:} $$t_1 = \frac{1}{a_1a_2} = \frac{1}{d}\left(\frac{1}{a_1} - \frac{1}{a_2}\right)$$
  $$t_2 = \frac{1}{d}\left(\frac{1}{a_2} - \frac{1}{a_3}\right)$$
  $$\ldots$$
  $$t_n = \frac{1}{d}\left(\frac{1}{a_n} - \frac{1}{a_{n + 1}}\right)$$

  Adding, we get $$S = \frac{1}{d}\left(\frac{1}{a_1} - \frac{1}{a_{n + 1}}\right) = \frac{n}{a_1a_{n + 1}}$$
\end{frame}
\begin{frame}{Problem 5}
  \textbf{5.} If $a_1, a_2, a_3, \ldots, a_n$ be in A.P. and $a_1 = 0,$ show that $\frac{a_3}{a_2} + \frac{a_4}{a_3} + \ldots +
  \frac{a_n}{a_{n - 1}} - a_2\left(\frac{1}{a_2} + \frac{1}{a_3} + \ldots + \frac{1}{a_{n - 2}}\right) = \frac{a_{n - 1}}{a_2} +
  \frac{a_2}{a_{n - 1}}$
\end{frame}
\begin{frame}{Solution of Problem 5}
  \textbf{Solution:} $\because a_1 = 0 \therefore a_2 = d, a_3 = 2d, \ldots, a_n = (n - 1)d$

  $$L.H.S. = \frac{2}{1} + \frac{3}{2} + \ldots + \frac{n - 1}{n - 2} - \left(1 + \frac{1}{2} + \frac{1}{3} + \ldots + \frac{1}{n -
    3}\right)$$
  $$= (1 + 1) + \left(1 + \frac{1}{2}\right) + \ldots + \left(1 + \frac{1}{n - 2}\right) - \left(1 + \frac{1}{2} + \frac{1}{3} +
  \ldots + \frac{1}{n - 3}\right)$$
  $$= n - 2 + \frac{1}{n - 2} = \frac{a_{n - 1}}{a_2} + \frac{a_2}{a_{n - 1}}$$
\end{frame}
\begin{frame}{Problem 6}
  \textbf{6.} If $a_1, a_2, \ldots, a_n$ are in A.P., whose common difference is $d,$ show that $\sum_{k = 1}^n\frac{a_ka_{k +
      1}a_{k + 2}}{a_k+a_{k + 2}}$ $= \frac{n}{2}\left[a_1^2 + (n + 1)a_1d + \frac{(n - 1)(2n + 5)}{6}d^2\right]$
\end{frame}
\begin{frame}{Solution of Problem 6}
  \textbf{Solution:}$$L.H.S. = \sum_{k = 1}^n\frac{a_ka_{k + 1}a_{k + 2}}{(a_{k + 1} - d) + (a_{k + 1} + d)} = \frac{1}{2\sum_{n =
      1}}^ka_ka_{k + 2}$$
  $$= \frac{1}{2}\sum_{k = 1}^n(a_{k + 1} - d)(a_{k + 1} + d) = \frac{1}{2}\sum_{k = 1}^n(a_{k + 1}^2 - d^2)$$
  $$= \frac{1}{2}\sum_{k = 1}^n[(a_1 + kd)^2 - d^2] = \frac{1}{2}\sum_{k = 1}^n[a_1^2 + 2a_1kd + (k^2 - 1)d^2]$$
  $$= \frac{n}{2}\left[a_1^2 + (n + 1)a_1d + \frac{(n - 1)(2n + 5)}{6}d^2\right]$$
\end{frame}
\begin{frame}{Problem 7}
  \textbf{7.} If $x, y$ and $z$ are positive real numbers different from $1,$ and $x^{18} = y^{21} = z^{28},$ show that $3, 3\log_y
  x, 3\log_z y, 7\log_x z$ are in A.P.
\end{frame}
\begin{frame}{Solution of Problem 7}
  \textbf{Solution:} $$x^{18} = y^{21}\Rightarrow 18\log x = 21\log y \Rightarrow \log_y x = \frac{21}{18} = \frac{7}{6}$$
  $$y^{21} = z^{28} \Rightarrow \log_z y = \frac{4}{3}$$
  $$x^{18} = z^{28} \Rightarrow \log_x z = \frac{9}{14}$$
  $$3\log_y x = \frac{7}{2}, 3\log_z y = 4, 7\log_x z = \frac{9}{2}$$

  Clearly, $3, \frac{7}{2}, 4, \frac{9}{2}$ are in A.P.
\end{frame}
\begin{frame}{Problem 8}
  \textbf{8.} If $I_n = \int_{0}^{\frac{\pi}{2}}\frac{\sin^2nx}{\sin^2x}dx,$ then $I_1, I_2, I_3, \ldots$ are in A.P.
\end{frame}
\begin{frame}{Solution of Problem 8}
  \textbf{Solution:} $$I_{n + 2}I_n - 2I_{n + 1} = \int_{0}^{\frac{\pi}{2}}\frac{\sin^2(n + 2)x + \sin^2nx - \sin^2(n +
    1)x}{\sin^2x}dx$$
  $$= \int_{0}^{\frac{\pi}{2}}\frac{1 - \cos(2n + 4)x + 1 - \cos 2nx - 2 + 2\cos(2n + 2)x}{2\sin^2x}dx$$
  $$= \int_{0}^{\frac{\pi}{2}}\frac{2\cos(2n + 2)x - 2\cos(2n + 2)x\cos 2x}{2\sin^2x}dx$$
  $$= \int_{0}^{\frac{\pi}{2}}\frac{2\cos(2n + 2)x.2\sin^2x}{2\sin^2x}dx$$
  $$= \int_{0}^{\frac{\pi}{2}}2\cos(2n + 2)x dx$$
  $$= 0$$
  Thus, $I_1, I_2, I_3, \ldots$ are in A.P.
\end{frame}
\begin{frame}{Problem 9}
  \textbf{9.} Can there be an A.P. whose terms are distinct prime numbers?
\end{frame}
\begin{frame}{Solution of Problem 9}
  \textbf{Solution:} Let $a_1, a_2, a_3, \ldots$ be an A.P., whose terms are ditinct prime numbers.
  \linebreak\linebreak
  Clearly, $a_1$ is a positive integer greater than $1.$
  \linebreak\linebreak
  Also, c.d. of A.P. i.e. $d = a2 - a1,$ then $d\geq 1$
  \linebreak\linebreak
  Now $(a_1 + 1)$th term of A.P. $= a_1 + a_1d = a_1(1 + d)$
  \linebreak\linebreak
  Since $a_1$ is a positive no. and $1 + d$ is a positive integer greater or equal than two it is a composite number. Thus, an
  A.P. of distinct prime no. is not possible.
\end{frame}
\begin{frame}{Problem 10}
  \textbf{10.} Four distinct no. are in A.P. If one of these integers is sum of the squares of remaining three, then $0$ must be
  one of the numbers in A.P.
\end{frame}
\begin{frame}{Solution of Problem 10}
  \textbf{Solution:} Let the four distinct integers in A.P. be $a, a + d, a + 2d, a + 3d$ where $d > 0$
  \linebreak\linebreak
  Let $a + 3d = a^2 + (a + d)^2 + (a + 2d)^2 = 3a^2 + 6ad + 5d^2$
  \linebreak\linebreak
  $\Rightarrow 5d^2 + 3(2a - 1)d + 3a^2 -a = 0$
  \linebreak\linebreak
  $\because d$ is real $\therefore 9(2a - 1)^2 - 20(2d^2 - a)\geq 0$
  \linebreak\linebreak
  $\Rightarrow -24a^2 - 16a + 9 \geq 0 \Leftrightarrow \frac{-4 - \sqrt{70}}{12}\leq a \leq \frac{-4 + \sqrt{70}}{12}$
  \linebreak\linebreak
  $\therefore a = -1, 0 \Rightarrow d = 1, \frac{4}{5}$
  \linebreak\linebreak
  We find that $-1, 0, 1, 2$ to be the sequence of numbers.
\end{frame}
\end{document}
