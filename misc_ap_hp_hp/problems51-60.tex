\documentclass[aspectratio=1610,8pt]{beamer}

% Standard packages

\usepackage[english]{babel}
%\usepackage[latin1]{inputenc}
%\usepackage{times}
%\usepackage[T1]{fontenc}
\usepackage{fontspec}
\usepackage[]{unicode-math}
\setmathfont{Inconsolata}
\setsansfont{Roboto}

% Setup TikZ

\usepackage{tikz}
\usetikzlibrary{arrows}
\tikzstyle{block}=[draw opacity=0.7,line width=1.4cm]


% Author, Title, etc.

\title{Miscellaneous Problems on A.P., G.P. and H.P.\\Problems 51-60}

\author[Shiv Shankar Dayal]{Shiv Shankar Dayal}

% The main document

\begin{document}
\begin{frame}
  \titlepage
\end{frame}
\begin{frame}{Problem 51}
  \textbf{51.} If $\int_0^{\frac{\pi}{2}}\cos^nx\cos(nx)dx,$ then prove that $I_1, I_2, I_3, \ldots$ are in G.P.
\end{frame}
\begin{frame}{Solution of Problem 51}
  \textbf{Solution:} $$I_n = \int_0^{\frac{\pi}{2}}\cos^nx\cos(nx)dx$$
  $$I_{n + 1} = \int_0^{\frac{\pi}{2}}\cos^{n + 1}x\cos(n + 1)xdx$$
  $$= \int_0^{\frac{\pi}{2}}\cos^{n}x[\cos x\cos (n + 1)x]dx$$
  Now, $\cos(nx) = \cos[(n + 1) - 1]x = \cos(n + 1)x\cos x + \sin(n + 1)x\sin x$
  $\Rightarrow \cos(n + 1)x\cos x = \cos nx - \sin(n + 1)x\sin x$
  $$\therefore I_{n + 1} = \int_0^{\frac{\pi}{2}}\cos^nx[\cos(nx) - \sin(n + 1)x\sin x]dx$$
  $$= I_n - \int_0^{\frac{\pi}{2}}\cos^nx \sin x\sin(n + 1)xdx$$
  Taking $u = \sin(n + 1)x$ and $v = \cos^nx\sin x,$ we get
  $$= I_n + \left[\frac{\cos^{n + 1}x\sin(n + 1)x}{n + 1}\right]_0^{\frac{\pi}{2}} - \int_0^{\frac{\pi}{2}}\cos^{n + 1}x\cos(n + 1)xdx$$
  $$= I_n + 0 - 0 - I_{n + 1} \Rightarrow 2I_{n + 1} = I_n\Rightarrow \frac{I_{n + 1}}{I_n} = \frac{1}{2}\forall n\in N$$
  Thus, $I_1, I_2, I_3, \ldots$ are in G.P.
\end{frame}
\begin{frame}{Problem 52}
  \textbf{52.} Let $I_n = \int_0^\pi \frac{\sin(2n - 1)x}{\sin x}dx.$ Show that $I_1, I_2, I_3, \ldots$ are in A.P. as well as in
  G.P.
\end{frame}
\begin{frame}{Solution of Problem 52}
  \textbf{Solution:} $$I_{n + 1} - I_{n} = \int_0^\pi \frac{\sin(2n + 1)x - \sin(2n - 1)x}{dx} = \int_0^\pi \frac{2\cos 2nx\sin
    x}{\sin x}dx$$
  $$= 2\int_0^\pi\cos 2nx dx = \frac{2}{2n}[\sin 2nx]_0^\pi = 0\Rightarrow I_{n + 1} = I_n$$
  Also, $I_1 = \int_o^\pi \frac{\sin x}{\sin x}dx = \pi$
  Thus, $I_1 = I_2 = I_3 = \ldots = \pi$
  Thus, $I_1, I_2, I_3, \ldots$ are in A.P. as well as in G.P.
\end{frame}
\begin{frame}{Problem 53}
  \textbf{53.} Prove that the three successive terms of a G.P. will form sides of a triangle if the common ratio $r$ satisfied the
  inequality $\frac{1}{2}(\sqrt{5} - 1) < r < \frac{1}{2}(\sqrt{5} + 1)$
\end{frame}
\begin{frame}{Solution of Problem 53}
  \textbf{Solution:} Let the sides of the triangle be $a, ar, ar^2.$
  \linebreak\linebreak
  When $r = 1$ the triangle formed will be equilateral triangle.
  \linebreak\linebreak
  If $r > 1$ the triangle will be formed if $a + ar > ar^2$
  $$r^2 - r - 1 < 0\Rightarrow \frac{1 - \sqrt{5}}{2} < r <\frac{1 + \sqrt{5}}{2}$$
  $$\Rightarrow r < \frac{1 + \sqrt{5}}{2}[\because r > 1]$$
  Similarly when $r < 1$ the triangle will be formed if $r > \frac{1 - \sqrt{5}}{2}$
  \linebreak\linebreak
  Thus, the range is $\frac{1}{2}(\sqrt{5} - 1) < r < \frac{1}{2}(\sqrt{5} + 1)$
\end{frame}
\begin{frame}{Problem 54}
  \textbf{54.} Find out whether $111\ldots1( 91$ digits $)$ is a prime number.
\end{frame}
\begin{frame}{Solution of Problem 54}
  \textbf{Solution:} $111\ldots 1 = 10^{90} + 10^{89} + 10^{88} + 10^2 + 10 + 1 = \frac{10^{91}}{10 - 1}$
  Since $91 = 13\times 7$ we can multiply or divide by $10^{13} - 1$ or $10^7 - 1$
  $$\frac{10^{91}}{10 - 1} = \frac{10^{91}}{10^7 - 1}\frac{10^{7}}{10 - 1}$$
  $$= (10^{84} + 10^{77} + 10^{70} + \ldots + 1)(10^6 + 10^5 + \ldots + 1)$$
  Therefore, given no. is not a prime nnumber.
\end{frame}
\begin{frame}{Problem 55}
  \textbf{55.} Find the natural number $a$ for which $\sum_{k = 1}^nf(a + k) = 16(2^n - 1),$ where the function $f$ satisfied the
  relation $f(x + y) = f(x)f(y)$ for all natural nuumbers $x, y$ and further $f(1) = 2$
\end{frame}
\begin{frame}{Solution of Problem 55}
  \textbf{Solution:} Given $f(x + y) = f(x)f(y)$ for all natural number $x$ and $y$
  \linebreak\linebreak
  $$\therefore f(a + k) = f(a)f(k)$$
  $$\sum_{k = 1}^nf(a + k) = \sum_{k = 1}^nf(a)f(k) = f(a)[f(1) + f(2) + \ldots + f(k)]$$
  $$f(2) = f(1) + f(1) \Rightarrow f(2) = [f(1)]^2 = 2^2$$
  $$f(3) = f(1)f(2)= 2^3$$
  Given $$\sum_{k = 1}^nf(a + k) = 16(2^n - 1)$$
  $$2^a[2 + 2^2 + \ldots + 2^n] = 16(2^n - 1)$$
  $$2^a.2.(2^n - 1) = 16.(2^n - 1)\Rightarrow a = 3$$
\end{frame}
\begin{frame}{Problem 56}
  \textbf{56.} In a certain test, there are $n$ questions. In this test $2^{n - i}$ students give wrong answers to at least $i$
  questions ($1\leq i \leq n.$) If total no. of wrong answers given is $2047,$ find the value of $n.$
\end{frame}
\begin{frame}{Solution of Problem 56}
  \textbf{Solution:} Number of students giving wrong answers to at least $i$ questions $= 2^{n - i}$
  \linebreak\linebreak
  Number of students gicing wrong answers to at least $i + 1$ questions $= 2^{n - i - 1}$
  \linebreak\linebreak
  $\therefore$ Number of students giving wrong answers to exactly $i$ questions $= 2^{n - i} - 2^{n - i - 1}.$ Also, number of
  students giving wrong answers to exactly $n$ questions $= 2^{n - n} = 1$
  \linebreak\linebreak
  $\therefore$ Total no. of wrong answers $1(2^{n - 1} - 2^{n - 2}) + 2(2^{n - 2} - 2^{n - 3}) + \ldots + (n - 1)(2^1 - 2^0) +
  n(2^0)$
  \linebreak\linebreak
  $= 2^{n - 1} + (-2^{n - 2} + 2.2^{n - 2}) + (-2.2^{n - 3} + 3.2^{n - 3}) + \ldots + \left[-(n - 1)2^0\right] + n.2^0$
  \linebreak\linebreak
  $= 2^{n - 1} + 2^{n - 2} + 2^{n - 3} + \ldots + 2^0 = 2^n - 1$
  \linebreak\linebreak
  Given, $2^n - 1 = 2047 \Rightarrow n = 11$
\end{frame}
\begin{frame}{Problem 57}
  \textbf{57.} If $S_1, S_2, S_3, \ldots, S_2n$ are the sums of infinite geometric series whose first terms are respectively $1, 2,
  3, \ldots, 2n$ and common ratio are respectively $\frac{1}{2}, \frac{1}{3}, \ldots, \frac{1}{2n + 1},$ find the value of $S_1^2 +
  S_2^2 + \ldots + S_{2n - 1}^2$
\end{frame}
\begin{frame}{Solution of Problem 57}
  \textbf{Solution:} $S_1 = \frac{1}{1 - \frac{1}{2}} = 2, S_2 = \frac{2}{1 - \frac{1}{3}} = 3, S_3 = \frac{3}{1 - \frac{1}{4}} =
  4$ and so on.
  $$S_1^2 + S_2^2 + \ldots + S_{2n - 1}^2 = 2^2 + 3^2 + \ldots + 2n^2 = 1^2 + 2^2 + 3^2 + \ldots + 2n^2 - 1^2$$
  $$= \frac{2n(2n + 1)(4n + 1)}{6} - 1 = \frac{n(2n + 1)(4n + 1)}{3} - 1$$
\end{frame}
\begin{frame}{Problem 58}
  \textbf{58.} A sqaure is given, a second square is made by joining the middle points of the first square and then a third square
  is made by joining the middle points of the sides of second square and so on till infinity. Show that the area of first square is
  equal to sum of the areas of all the succeeding squares.
\end{frame}
\begin{frame}{Solution of Problem 58}
  \textbf{Solution:}
  \begin{center}
    \begin{tikzpicture}
      \draw (0, 0) -- (2, 0) -- (2, 2) -- (0, 2) -- cycle;
      \draw (1, 0) -- (2, 1) -- (1, 2) -- (0, 1) -- cycle;
      \draw (0, 0) -- (2, 2);
      \draw (0, 0) node[anchor=north east] {$A$};
      \draw (2, 0) node[anchor=north west] {$B$};
      \draw (2, 2) node[anchor=south west] {$C$};
      \draw (0, 2) node[anchor=south east] {$D$};
      \draw (1, 0) node[anchor=north] {$P$};
      \draw (2, 1) node[anchor=west] {$Q$};
      \draw (1, 2) node[anchor=south] {$R$};
      \draw (0, 1) node[anchor=east] {$S$};
    \end{tikzpicture}
  \end{center}
  Let $ABCD$ be the first sqaure. Let $AB = a \Rightarrow AC = \sqrt{2}a~\therefore PQ = \frac{AC}{2} = \frac{a}{\sqrt{2}}$
  \linebreak\linebreak
  $\therefore $ Area of first square $= a^2$
  \linebreak\linebreak
  Area of second sqaure $= \frac{a^2}{2}$
  \linebreak\linebreak
  Area of third square $= \frac{a^2}{4}$
  \linebreak\linebreak
  Sum of areas of all squares except first $= \frac{a^2}{2} + \frac{a^2}{4} + \ldots = \frac{\frac{a^2}{2}}{1 - \frac{1}{2}} = a^2$
\end{frame}
\begin{frame}{Problem 59}
  \textbf{59.} If $a$ is the value of $x$ for which the function $7 + 2x\log 25 - 5^{x - 1} - 5^{2 - x}$ has the greatest value and
  $r = lim_{x\to 0}\int_{0}^x\frac{t^2}{x^2\tan(\pi + x)}dt,$ find $\lim_{n \to \infty}\sum_{n = 1}^nar^{n - 1}$
\end{frame}
\begin{frame}{Solution of Problem 59}
  \textbf{Solution:} Let $f(x) = 7 + 2x\log 25 - 5^{x - 1} -5^{2 - x}$
  $$f'(x) = 4\log 5 - 5^{x - 1}\log 5 + 5^{2 - x}\log 5 = \frac{\log 5}{5^{x + 1}}(5^x - 25)(5^x + 5)$$
  $\because \frac{\log 5}{5^{x + 1}}(5^x + 5) > 0$ for all real $x > 0$
  \linebreak\linebreak
  $f'(x) < 0$ or $> 0$ as $x > 2$ or $x < 2$
  \linebreak\linebreak
  So $f(x)$ has local max. at $x = 2$ and has no local min. Hence, $f(x)$ has greatest value at $x = 2$ i.e. $a = 2.$ Now
  $$r = lim_{x\to 0}\int_{0}^x\frac{t^2}{x^2\tan(\pi + x)}dt = lim_{x\to 0}\frac{\int_{0}^xt^2}{x^2\tan(\pi + x)}dt$$
  $$= \lim_{x\to 0}\frac{x^3}{3x^2\tan x} = \frac{1}{3}$$
  $$\Rightarrow \lim_{n \to \infty}\sum_{n = 1}^nar^{n - 1} = a + ar + ar^2 + \ldots~\text{to}~\infty = \frac{a}{1 - r}$$
  $$= 3$$
\end{frame}
\begin{frame}{Problem 60}
  \textbf{60.} If $p$th, $q$th, $r$th terms of a G.P. are positive numbers $a, b, c$ respectively, show that the vectors $(\log
  a).\vec{i} + (\log b)\vec{j} + (\log c)\vec{k}$ and $(q - r)\vec{i} + (r - p)\vec{j} + (p - q)\vec{k}$ are perpendicular.
\end{frame}
\begin{frame}{Solution of Problem 60}
  \textbf{Solution:} Let $x$ be the first term and $y$ be the common ratio of the G.P. Then
  $$a = xy^{p - 1}, b = xy^{q - 1}, c = xy^{r - 1}$$
  $$\Rightarrow \log a = \log x + (p - 1)\log y, \log b = \log x + (q - 1)\log y, \log c = \log x + (r - 1)\log y$$
  If the vectors are perpendicular the dot product will be zero.
  $$\therefore (q - r)\log a + (r - p)\log b + (p - q)\log c = 0$$
  $$(q - r + r - p + p - q)\log x + [(q - r)(p - 1) + (r - p)(q - 1) + (p - q)(r - 1)]\log y = 0$$
\end{frame}
\end{document}
