\documentclass[aspectratio=1610,8pt]{beamer}

% Standard packages

\usepackage[english]{babel}
%\usepackage[latin1]{inputenc}
%\usepackage{times}
%\usepackage[T1]{fontenc}
\usepackage{fontspec}
\usepackage[]{unicode-math}
\setmathfont{Inconsolata}
\setsansfont{Roboto}

% Setup TikZ

\usepackage{tikz}
\usetikzlibrary{arrows}
\tikzstyle{block}=[draw opacity=0.7,line width=1.4cm]


% Author, Title, etc.

\title{Miscellaneous Problems on A.P., G.P. and H.P.\\Problems 21-30}

\author[Shiv Shankar Dayal]{Shiv Shankar Dayal}

% The main document

\begin{document}
\begin{frame}
  \titlepage
\end{frame}
\begin{frame}{Problem 21}
  \textbf{21.} Prove that $\tan 70^\circ, \tan 50^\circ + \tan 20^\circ, \tan 20^\circ$ are in A.P.
\end{frame}
\begin{frame}{Solution of Problem 21}
  \textbf{Solution:} $$\tan70^\circ = \tan(50^\circ + 20^\circ) = \frac{\tan50^\circ + \tan 20^\circ}{1 - \tan50^\circ\tan20^\circ}$$
  $$\tan70^\circ - \tan70^\circ\tan50^\circ\tan20^\circ = \tan50^\circ + \tan 20^\circ$$
  $$\because \tan70^\circ.\tan20^\circ = \tan70^\circ\cot70^\circ = 1$$
  $$\Rightarrow \tan70^\circ = \tan20^\circ + 2\tan50^\circ$$
  Thus, $\tan 70^\circ, \tan 50^\circ + \tan 20^\circ, \tan 20^\circ$ are in A.P. with common difference of $\tan50^\circ.$
\end{frame}
\begin{frame}{Problem 22}
  \textbf{22.} If $\log_l x, \log_m x, \log_n x$ are in A.P. and $x \neq 1,$ prove that $n^2 = (nl)^{\log_l m}$
\end{frame}
\begin{frame}{Solution of Problem 22}
  \textbf{Solution:} $\because \log_l x, \log_m x, \log_n x$ are in A.P.
  $$2\log_m x = \log_l x + \log_n x \Rightarrow 2\frac{\log x}{\log m} = \frac{\log x}{\log l} + \frac{\log x}{\log n}$$
  $$\Rightarrow \frac{2}{\log m} = \frac{\log nl}{\log l\log n}\Rightarrow 2\log n = \log nl\log_l m$$
  $$\Rightarrow \log n^2 = \log (nl)^{\log_l m}$$
  Taking anti-log $n^2 = (nl)^{\log_l m}$
\end{frame}
\begin{frame}{Problem 23}
  \textbf{23.} The length of sides of a right angled triangle are in A.P., show that their ratio is $3:4:5$
\end{frame}
\begin{frame}{Solution of Problem 23}
  \textbf{Solution:} Let the sides are $a, a + d, a + 2d$ of the right angled triangle where $d > 0$. Clearly, $a + 2d$ is largest and hence
  hypotenuse. Thus, we can write

  $$(a + 2d)^2 = a^2 + (a + d^2) \Rightarrow a^2 - 2ad - 3d^2 = 0 \Rightarrow a = -d, a = 3d$$

  However, a side cannot be -ve so $a = 3d$ and hence ratio of sides are $3:4:5$
\end{frame}
\begin{frame}{Problem 24}
  \textbf{24.} If $\log_3 2, \log_3(2^x - 5), \log_3\left(2^x - \frac{7}{2}\right)$ are in A.P., determine the value of $x.$
\end{frame}
\begin{frame}{Solution of Problem 24}
  \textbf{Solution:} $\because \log_3 2, \log_3(2^x - 5), \log_3\left(2^x - \frac{7}{2}\right)$ are in A.P.

  $$\log_3(2^x - 5) - \log_32 = \log_3\left(2^x - \frac{7}{2}\right) - \log_3(2^x - 5)$$
  $$(2^x - 5)^2 = 2\left(2^x - \frac{7}{2}\right) \Rightarrow 2^{2x} - 12.2^x + 32 = 0$$
  $$2^x = 8, 4\Rightarrow x = 3, 2$$

  However, if $d = 2, 2^x - 5 < 0$ and logarithm of that wont be real. Thus, $x = 3$
\end{frame}
\begin{frame}{Problem 25}
  \textbf{25.} Find the values of $a$ for which $5^{1 + x} + 5^{1 - x}, \frac{a}{2}, 25^x + 25^{-x}$ are in A.P.
\end{frame}
\begin{frame}{Solution of Problem 25}
  \textbf{Solution:} $$a = 5^{1 + x} + 5^{1 - x} + 25^x + 25^{-x} = 5(5^x + 5^{-x}) + 25^x + 25^{-x} \geq 5.2 + 2 = 12$$

  This solution used the fact that for real no. $x + \frac{1}{x}\geq 2$ which can be proved trivially.
\end{frame}
\begin{frame}{Problem 26}
  \textbf{26.} If $\log 2, \log(2^x - 1)$ and $\log(2^x + 3)$ are in A.P., then find $x.$
\end{frame}
\begin{frame}{Solution of Problem 26}
  \textbf{Solution:} $$\log(2^x - 1) - \log 2 = \log(2^x + 3) - \log(2^x - 1) \Rightarrow (2^x - 1)^2 = 2(2^x + 3)$$
  $$2^{2x} - 4.2^x -5 = 0 \Rightarrow 2^x = -1, 5$$
  However, for real $x, 2^x$ cannot be $-1.$ Thus, $x = \log_25$ is the solution.
\end{frame}
\begin{frame}{Problem 27}
  \textbf{27.} If $1, \log_y x, \log_zy, -15\log_xz$ are in A.P., then prove that $x = z^3$ and $y = z^{-3}$
\end{frame}
\begin{frame}{Solution of Problem 27}
  \textbf{Solution:} Let $d$ be the common difference of the A.P. Then,
  $$\log_yx = 1 + d\Rightarrow x = y^{1 + d}$$
  $$\log_zy = 1 + 2d\Rightarrow y = z^{1 + 2d} \Rightarrow x = y^{(1 + d)(1 + 2d)}$$
  $$-15\log_xz = 1 + 3d \Rightarrow z = x^{-\frac{1 + 3d}{15}}$$
  $$\therefore x = x^{-\frac{(1 + d)(1 + 2d)(1 + 3d)}{15}}$$
  $$(1 + d)(1 + 2d)(1 + 3d) = -15$$
  $$(d + 2)(6d^2 - d + 8 = 0)$$
  $\therefore d = -2[\because 6d^2-d + 8 = 0$ has complex roots.$]$
  \linebreak\linebreak
  $\therefore x = z^3, y = z^{-3}$
\end{frame}
\begin{frame}{Problem 28}
  \textbf{28.} Show that $\sqrt{2}, \sqrt{3}, \sqrt{5}$ cannot be terms of a single A.P.
\end{frame}
\begin{frame}{Solution of Problem 28}
  \textbf{Solution:} Let $\sqrt{2}, \sqrt{3}, \sqrt{5}$ be $p$th, $q$th and $r$th term of an A.P. Then,
  $$\frac{\sqrt{5} - \sqrt{3}}{\sqrt{3} - \sqrt{2}} = \frac{r - q}{q - p} = k$$
  Since $p, q, r$ are integers therefore $k$ will be a rational number.
  $$\sqrt{5}- \sqrt{3} = k(\sqrt{3}- \sqrt{2})$$
  Squaring, we get
  $$5 + 3 - 2\sqrt{15} = k(3 + 2 - 2\sqrt{6})$$
  $$k = \frac{8 - 2\sqrt{15}}{5 - 2\sqrt{6}}$$
  Multiplying both numberator and denominator with $5 + 2\sqrt{6},$ we obtain that $k$ is not a rational number. Thus, we have
  proved the required condition.
\end{frame}
\begin{frame}{Problem 29}
  \textbf{29.} A circle of one centimeter radius is drawn on a piece of paper and with the same center $3n - 1$ other circles are
  drawn of radii $2$ cm, $3$ cm, $4$ cm and so on. The inner circle is painted blue, the ring between that and next circle is
  painted red, the next ring yellow then other rings blue, red, yellow and so on in this order. Show that the successive aread of
  each color are in A.P.
\end{frame}
\begin{frame}{Solution of Problem 29}
  \textbf{Solution:} The blue color circles will have radius of $1, 4, 7, 10, \ldots, 3n - 2$ cm each.
  \linebreak\linebreak
  So the areas of blue rings are $\pi.1^2, \pi(4^2 - 3^2), \pi(7^2 - 6^2), \ldots, \pi[(3n -2)^2 - (3n - 3)^2]$
  \linebreak\linebreak
  $\pi[(3n -2)^2 - (3n - 3)^2] = \pi(6n - 5)$ which is $t_n$ and forms an A.P. with common ratio $6$
  \linebreak\linebreak
  Similarly, it can be shown that red and yellow rings form an A.P.
\end{frame}
\begin{frame}{Problem 30}
  \textbf{30.} If $x, y, z(x, y, z\neq 0)$ are in A.P. and $\tan^{-1}x, \tan^{-1}y, \tan^{-1}z$ are also in A.P., then prove that
  $x = y = z$
\end{frame}
\begin{frame}{Solution of Problem 30}
  \textbf{Solution:} Since $x, y, z$ are in A.P. $2y = x + z$
  Similarly, $$2\tan^{-1}y = \tan^{-1}x + \tan^{-1}z$$
  $$2\tan^{-1}y = \tan^{-1}\frac{x + z}{1 - xz}$$
  $$\tan(2\tan^{-1}y) = \frac{x + z}{1 - xz}$$
  $$\frac{2y}{1 - y^2} = \frac{x + z}{1 - xz}$$
  $$(x + z)\left[\frac{1}{1- y^2} - \frac{1}{1 - xz} = 0\right]$$
  $$\Rightarrow y^2 = xz$$
  So $x, y, z$ are in G.P. as well. Thus,
  $$4y^2 = x^2 + z^2 + 2xz = x^2 + z^2 + 2y^2 \Rightarrow x^2 + z^2 - 2xz = 0 \Rightarrow x = z \Rightarrow x = y = z$$
\end{frame}
\end{document}
