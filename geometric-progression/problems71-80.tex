\documentclass[aspectratio=1610,8pt]{beamer}

% Standard packages

\usepackage[english]{babel}
%\usepackage[latin1]{inputenc}
%\usepackage{times}
%\usepackage[T1]{fontenc}


% Setup TikZ

\usepackage{tikz}
\usetikzlibrary{arrows}
\tikzstyle{block}=[draw opacity=0.7,line width=1.4cm]


% Author, Title, etc.

\title{Geometric Progression\\Problems 71-80}

\author[Shiv Shankar Dayal]{Shiv Shankar Dayal}

% The main document

\begin{document}
\begin{frame}
  \titlepage
\end{frame}
\begin{frame}{Problem 71}
  \textbf{71.} After striking the floor a certain ball rebound to $\frac{4}{5}$th of the height from which it has fallen. Find the
  total distance it travels before coming to rest if it is gently dropped from a height of $120$ meters.
\end{frame}
\begin{frame}{Solution of Problem 71}
  \textbf{Solution:} Distance covered before first bounce $= 120$ meters.

  After striking the floor the ball will go up for $120\frac{4}{5}$ meters and then fall the same distance so distance covered $=
  2.120.\frac{4}{5}$ meteres.

  For the next bounce distance covered would be $2.120.\frac{4^2}{5^2}$ meters.

  This will keep happening till ball comes to rest.

  Thus, total distance covered would be $= 120 + 240.\frac{4}{5} + 240.\frac{4^2}{5^2} + \ldots~\text{to}~\infty$

  $$= 120 + 240.\frac{4}{5}\left[1 + \frac{4}{5} + \frac{4^2}{5^2} + \ldots~\text{to}~\infty\right]$$

  $$= 120 + 240.\frac{4}{5}.5 = 1080$$ meters.
\end{frame}
\begin{frame}{Problem 72}
  \textbf{72.} If $a$ be the first term and $b$ be the $n$th term and $p$ be the product of $n$ terms of a G.P., show that $p^2 =
  (ab)^n$
\end{frame}
\begin{frame}{Solution of Problem 72}
  \textbf{Solution:} Let $r$ be the common ratio. $b = ar^{n - 1} \Rightarrow ab = a^2r^{n - 1}\Rightarrow (ab)^n = a^{2n}r^{n(n - 1)}$

  $$p = a.ar.ar^2.\ldots.ar^{n - 1} = a^nr^{\frac{n(n - 1)}{2}} \Rightarrow p^2 = a^{2n}r^{n(n - 1)}$$

  Thus, $p^2 = (ab)^n$
\end{frame}
\begin{frame}{Problem 73}
  \textbf{73.} Show that the ratio of sum of $n$ terms of two G.P.'s having the same common ratio is equal to the ratio of their
  $n$th terms.
\end{frame}
\begin{frame}{Solution of Problem 73}
  \textbf{Solution:} Let $a$ and $b$ be first terms and $r$ be the common ratio of two G.P.

  Ratio of sums $= \frac{\frac{a(r^n - 1)}{r - 1}}{\frac{b(r^n - 1)}{r - 1}} = \frac{a/b}{}$

  Ratio of $n$th terms $= \frac{ar^{n - 1}}{br^{n - 1}} = \frac{a}{b}$

  Hence, proved.
\end{frame}
\begin{frame}{Problem 74}
  \textbf{74.} If $S_1, S_2, S_3$ be the sum of $m, 2n, 3n$ terms respectively of a G.P. show that $(S_2 - S_1)^2 = S_1(S_3 - S_2)$
\end{frame}
\begin{frame}{Solution of Problem 75}
  \textbf{Solution:} Let $a$ be the first term and $r$ be the common ratio of the G.P. Then,

  $$S_1 = \frac{a(r^n - 1)}{r - 1}, S_2 = \frac{a(r^{2n} - 1)}{r - 1}, S_3 = \frac{a(r^{3n} - 1)}{r - 1}$$
  $$S_2 - S_1 = \frac{ar^n(r^n - 1)}{r - 1}$$
  $$S_3 - S_2 = \frac{ar^{2n}(r^n - 1)}{r - 1}$$
  $$(S_2 - S_1)^2 = \frac{a^2r^{2n}(r^n - 1)^2}{(r - 1)^2}$$
  $$S_1(S_3 - S_2) = \frac{a(r^n - 1)}{r - 1}\left(\frac{a(r^{3n} - 1)}{r - 1}- \frac{a(r^{2n} - 1)}{r - 1}\right)$$
  $$= \frac{a(r^n - 1)}{(r - 1)^2}[a(r^{3n} - r^{2n})] = \frac{a^2r^{2n}(r^n - 1)^2}{(r - 1)^2}$$

  Hence, proved.
\end{frame}
\begin{frame}{Problem 75}
  \textbf{75.} If $S_n$ denotes the sum of $n$ terms of a G.P.,whose first term is $a$ and common ratio is $r,$ find $S_1 + S_2 +
  \ldots + S_{2n - 1}$
\end{frame}
\begin{frame}{Solution of Problem 76}
  \textbf{Solution:} $$S_1 = a = \frac{a(r - 1)}{r - 1}$$
  $$S_2 = a + ar = \frac{a(r^2 - 1)}{r - 1}$$
  $$\ldots$$
  $$S_{2n - 1} = a + ar + \ldots+ ar^{2n- 2} = \frac{a(r^{2n- 1} - 1)}{r - 1}$$
  $$S_1 + S_2 + \ldots + S_{2n - 1} = \frac{a}{1- r}\left[(r - 1) + (r^2 - 1) + \ldots + (r^{2n - 1} - 1)\right]$$
  $$ = \frac{a}{r - 1}\left[\frac{r(r^{2n - 1 - 1})}{r - 1} - 2n + 1\right]$$
\end{frame}
\begin{frame}{Problem 77}
  \textbf{77.} The sum of $n$ terms of a series is $a.2^n - b,$ find its $n$th term. Are the terms of this series in G.P.
\end{frame}
\begin{frame}{Solution of Problem 77}
  \textbf{Solution:} Given $S_n = a.2^n - b \Rightarrow S_{n - 1} = a.2^{n - 1} - b\Rightarrow t_n= a2^{n - 1}$ Since the ratio of
  terms will be $2$ as evident from $t_n$ the series is in G. P.
\end{frame}
\begin{frame}{Problem 78}
  \textbf{78.} 
\end{document}
