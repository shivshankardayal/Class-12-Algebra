\documentclass[aspectratio=169,8pt]{beamer}

% Standard packages

\usepackage[english]{babel}
\usepackage{siunitx}
\usepackage{textcomp}
%\usepackage[latin1]{inputenc}
%\usepackage{times}
%\usepackage[T1]{fontenc}


% Setup TikZ

\usepackage{tikz}
\usetikzlibrary{arrows}
\tikzstyle{block}=[draw opacity=0.7,line width=1.4cm]


% Author, Title, etc.

\title{Geometric Progression}

\author[Shiv Shankar Dayal]{Shiv Shankar Dayal}

% The main document

\begin{document}
\begin{frame}
       \titlepage
\end{frame}
\begin{frame}{Geometric Progression}
  \textbf{Definition:} A succession of numbers is said to be in G.P. if the
  ratio of any term and the preceding term is constant throughout. The constant
  term is known as \textit{common ratio} of the G.P.\\
  \textbf{$n$th term of a G.P.:} Let $a$ be the first term and $r$ be the
  common ratio of the G.P.\\
  \vspace{.5cm}
  Now, first term of G.P., $t_1 = a$\\
  second term of G.P., $t_2 = ar$\\
  third term of G.P., $t_3 = ar^2$\\
  \ldots
  $n$th term of G.P., $t_n = ar^{n - 1}$
\end{frame}
\begin{frame}{Properties of G.P.}
  \begin{enumerate}
    \item If each term of a G.P. is multiplied with a non-zero number then the
      sequence thus obtained is also in G.P.\\
      \vspace{.3cm}
      Let $a, ar, ar^2, ar^3, \ldots$ be a sequence in G.P. where $a$ is the
      first term and $r$ is the common ratio.\\
      \vspace{.3cm}
      Upon multiplying the terms of this sequence with a non-zero number, say
      $k$, it becomes $ak, ark, ar^2k, ar^3k, \ldots$\\
      \vspace{.3cm}
      Thus, we see that the resulting sequence is still G.P. with first term
      as $ak$ and common ratio $r$\\
    \item If each term of a G.P. is divided with a non-zero number then the
      sequence thus obtained is also in G.P.\\
      \vspace{.3cm}
      Following as above we will have our sequence as $\frac{a}{k},
      \frac{ar}{k}, \frac{ar^2}{k}, \frac{ar^3}{k},\dots$\\
      \vspace{.3cm}
      We see that this sequence is also in G.P.
    \item The reciprocals  of the terms of of a G.P. are also in G.P.\\
      \vspace{.3cm}
      The reciprocals of the terms of a G.P. $a, ar, ar^2, ar^3, \ldots$ is
      $\frac{1}{a}, \frac{1}{ar}, \frac{1}{ar^2}, \frac{1}{ar^3},\ldots$ which
      we see is a G.P. with first term as $\frac{1}{a}$ and common ratio
      $\frac{1}{r}$
  \end{enumerate}
\end{frame}
\begin{frame}{Sum of first $n$ terms of a G.P.}
  Let $a$ be the first term and $r$ be the common ratio of a G.P. and $S_n$ be
  the sum of first $n$ terms.\\
  \vspace{0.3cm}
  \textbf{Case I:} When $r \ne 1$\\
  $$S_n = a + ar + ar^2 + ar^2 + \ldots + ar^{n - 1}$$
  $$rS_n = ar + ar^2 + ar^3 + \ldots + ar^{n - 1} + ar^n$$
  Upon subtratcion,
  $$(1 - r)S_n = a - ar^n$$
  $$S_n = \frac{a(1 - r^n)}{1 - r} = \frac{a(r^n - 1)}{r - 1}$$
  \textbf{Case II:} When $r = 1$\\
  $$S_n = a + a + a + \ldots~\text{up to}~n~\text{terms} = na$$
\end{frame}
\begin{frame}{Sum of a G.P. when $|r| < 1$}
  Let $a$ be the first term, $r$ be the common ratio and $S_n$ be the sum of
  $n$ terms of the G.P. in question.\\
  \vspace{.3cm}
  Now, we have already found that $S_n = \frac{a(1 - r^n)}{1 - r}$. However,
  when $n = \infty, r^n = 0$ if $|r| < 1$\\
  \vspace{.3cm}
  $$\therefore S_{\infty} = \frac{a}{1 - r}$$
\end{frame}
\begin{frame}{Recurring Decimal}
  Recurring decimal is a very good example of an infinite G.P. and its value
  can be obtained from the formula for sum to infinity of a G.P. For example,
  let us find the value of $.\dot{3}$

  Now, $$.\dot{3} = .33333 \ldots~\text{to infinity}$$
  $$= .3 + .03 + .003 + .0003 + \ldots~\text{to infinity}$$
  $$= \frac{3}{10} + \frac{3}{100} + \frac{3}{1000} + \ldots~\text{to infinity}$$
  $$= \frac{3}{10}\left[1 + \frac{1}{10} + \frac{1}{100} + \ldots~\text{to
      infinity}\right]$$
  $$= \frac{3}{10}.\frac{1}{1 - \frac{1}{10}}$$
  $$= \frac{3}{10}.\frac{10}{9}$$
  $$= \frac{1}{3}$$
\end{frame}
\begin{frame}{Arithmetico Geometric Series}
  If the terms of an A.P. is multiplied by the corresponding terms of a G.P.,
  then the new series obtained is called an Arithmetico Geometric series.

  \textbf{Example: } If the terms of the arithmetic series $2 + 5 + 8 + 11 +
  \ldots$ is myltiplied by the corresponsing terms of the geometric series
  $x + x^2 + x^3 + x^4 + \ldots,$ then the following arithmetic geometric
  series is obtained.
  $$2x + 5x^2 + 8x^3 + 11x^4 + \ldots$$

  \textbf{Sum of an Arithmetico Geometric Series}\\
  Let $S$ be the sum of the arithmetic geometric series. Then each terms of the
  series is multiplied by $r$(the common ratio of G.P.) and are written
  shifting each term one step rightward and then we can subtract $rS$ from $S$
  to get $(1 - r)S$. Then the sum can be obtained.

  \textbf{Example:}$$S = 2x + 5x^2 + 8x^3 + 11x^4 + \ldots$$
  $$xS = ~~~~2x^2 + 5x^3 + 8x^4 + \ldots$$
  $$(1 - x)S = 2x + 3x^2 + 3x^3 + 3x^4 + \ldots$$
\end{frame}
\end{document}
