\documentclass[aspectratio=1610,8pt]{beamer}

% Standard packages

\usepackage[english]{babel}
%\usepackage[latin1]{inputenc}
%\usepackage{times}
%\usepackage[T1]{fontenc}


% Setup TikZ

\usepackage{tikz}
\usetikzlibrary{arrows}
\tikzstyle{block}=[draw opacity=0.7,line width=1.4cm]


% Author, Title, etc.

\title{Geometric Progression\\Problems 21-30}

\author[Shiv Shankar Dayal]{Shiv Shankar Dayal}

% The main document

\begin{document}
\begin{frame}
       \titlepage
\end{frame}
\begin{frame}{Choosing numbers in G.P. when product is given}
  \begin{enumerate}
    \item If three numbers in G.P. are given with a product then you should
      take the numbers as $\frac{a}{r}, a, ar$. If the product is not given
      then you should take them as $a, ar, ar^2$
    \item If four nummbers in G.P. are given with a product then you should
      take the numbers as $\frac{a}{r^3}, \frac{a}{r}, ar, ar^3$. If the
      product of numbers is not given then take them as $a, ar, ar^2, ar^3$.
  \end{enumerate}
\end{frame}
\begin{frame}{Problem 21}
  \textbf{21.} If the product of three numbers in G.P. be $216$ and their sum
  is $19$, find the numbers.
\end{frame}
\begin{frame}{Solution of problem 21}
  \textbf{Solution:} Let the three numbers be $\frac{a}{r}, a, ar$\\
  Given that product it $216$.
  $$\therefore \frac{a}{r}.a.ar = 216 \Rightarrow a^3 = 216 \Rightarrow a = 6$$
  Also, given that sum of these numbers is $19$
  $$\therefore \frac{a}{r} + a + ar = 19 \Rightarrow \frac{6}{r} + 6 + 6r =
  19$$
  $$6r^2 - 13r + 6 = 0 \Rightarrow 6r^2 -9r - 4r + 6 = 0$$
  $$3r(2r - 3) - 2(2r -3) = 0 \Rightarrow (2r - 3)(3r -2) = 0$$
  $$\r = \frac{2}{3}, \frac{3}{2}$$
  When $r = \frac{2}{3}$, numbers are $9, 6, 4$ and when $r = \frac{3}{2}$
  numbers are $4, 6, 9$
\end{frame}
\begin{frame}{Problem 22}
  \textbf{22.} A number consists of three digits in G.P. The sum of the right
  hand and left hand digits exceed twice the middle digit by $1$ and the sum of
  left hand and middle digit is two-third of the sum of the middle and right
  hand digits. Find the number.
\end{frame}
\begin{frame}{Solution of problem 22}
  \textbf{Solution:} Let the three digits be $a, ar, ar^2$\\
  Given, $$a + ar^2 = 2ar + 1 \Rightarrow a(r - 1)^2 = 1$$
  Also given that, $$a + ar = \frac{2}{3}(ar + ar^2) \Rightarrow 3a(1 + r) =
  2ar(1 + r)$$
  $$(1 + r)(3 - 2r) = 0 \therefore r = \frac{3}{2}, -1$$
  $$r = \frac{3}{2} \Rightarrow a = \frac{1}{(r - 1)^2} = \frac{1}{(\frac{3}{2} -
    1)^2} = 4$$
  $r$ cannot be $-1$ as that will make $a = \frac{1}{4}$ which is not possible
  as digits of a number are integers.\\
  Hence, $a = 4, ar = 4\frac{3}{2} = 6, ar^2 = 4\left(\frac{3}{2}\right)^2 = 9$\\
  Thus, our number is $469$
\end{frame}
\begin{frame}{Problem 23}
  \textbf{23.} In a set of four numbers, the first three are in G.P. and the
  last three are in A.P. with a common difference of $6$. If the first number
  is same as fourth, find the four numbers.
\end{frame}
\begin{frame}{Solution of problem 23}
  \textbf{Solution:} Let the last three numbers in A.P. be $b, b+6, b+12$ and
  the first number be $a$.\\
  Thus, $$a = b + 12, b^2 = a(b + 6) \Rightarrow b^2 = (b + 12)(b + 6)$$
  $$18b + 72 = 0 \Rightarrow b = -4 \Rightarrow a = -4 + 12 = 8$$
  Thus, numbers are $8, -4, 2, 8$
\end{frame}
\begin{frame}{Problem 24}
  \textbf{24.} The sum of three numbers in G.P. is $21$ and the sum of their
  squares is $189$. Find the numbers.
\end{frame}
\begin{frame}{Solution of problem 24}
  \textbf{Solution:} Let the three numbers be $a, ar, ar^2$ where $a$ is the
  first term and $r$ is the common ratio. Thus, we have
  $$a + ar + ar^2 = 21 \Rightarrow a(1 + r + r^2) = 21$$
  $$a^2 + a^2r^2 + a^2r^4 = 189 \Rightarrow a^2(1 + r^2 + r^4) = 189$$
  Squaring first equation and dividing it by second, we get
  $$\frac{a^2(1 + r + r^2)^2}{a^2(1 + r^2 + r^4)} = \frac{441}{189} =
  \frac{7}{3}$$
  $$\frac{(1 + r + r^2)^2}{1 + 2r^2 + r^4 - r^2} = \frac{7}{3} \Rightarrow
  \frac{(1 + r + r^2)^2}{(1 + r^2)^2 - r^2} = \frac{7}{3}$$
  $$\frac{1 + r + r^2}{1 -r + r^2} = \frac{7}{3} \Rightarrow 2r^2 - 5r + 2 =
  0$$
  $$\therefore r = 2, \frac{1}{2}$$
  When $r = 2, a = 3$, when $r = \frac{1}{2}, a = 12$\\
  Thus, numbers are either $3, 6, 12$ or $12, 6, 3$  
\end{frame}
\begin{frame}{Problem 25}
  \textbf{25.} The prodduct of three consecutive terms of a G.P. is $-64$ and
  the first term is four times the third. Find the terms.
\end{frame}
\begin{frame}{Solution of problem 25}
  \textbf{Solution:} Let the numbers are $\frac{a}{r}, a, ar$, where $a$ be the
  first term and $r$ be the common ratio. Given,
  $$\frac{a}{r}.a.ar = -64 \Rightarrow a^3 = -64 \Rightarrow a = -4$$
  $$\frac{a}{r} = 4ar \Rightarrow \frac{1}{r} = 4r, \Rightarrow r^2 =
  \frac{1}{4} \Rightarrow r = \pm \frac{1}{2}$$
  When $r = \frac{1}{2}$ numbers are $-8, -4, -2$ and when $r = -\frac{1}{2}$
  numbers are $8, -4, 2$
\end{frame}
\begin{frame}{Problem 26}
  \textbf{26.} Three numbers whose sum is $15$ are in A.P. If $1, 4, 19$ be
  added to them respectively the resulting numbers are in G.P. Find the
  numbers.
\end{frame}
\begin{frame}{Solution of problem 26}
  \textbf{Solution:} Let the three numbers in A.P. are $a - d, a, a + d$ where
  $a$ is the first term and $d$ is the common difference. Given,
  $$a - d + a + a + d = 15 \Rightarrow 3a = 15 \Rightarrow a = 5$$
  After adding $1, 4, 19$ the numbers are in G.P.,
  $$\therefore (a + 4)^2 = (a - d + 1)(a + d + 19)$$
  $$\Rightarrow a^2 + 8a + 16 = a^2 - d^2 + 19 + ad - ad + 19a + a - 19d + d$$
  $$\Rightarrow 8a + 16 = 20a - d^2 - 18d + 19 \Rightarrow d^2 + 18d + 12a + 3
  = 0$$
  $$d^2 + 18d +63 = 0 \Rightarrow (d + 21)(d - 3) = 0$$
  $$\therefore d = -21, 3$$
  When $d = 3$, numbers are $2, 5, 8$ and when $d = -21$, numbers are $26, 5,
  -16$
\end{frame}
\begin{frame}{Problem 27}
  \textbf{27.} From three numbers in G.P. other three numbers in G.P. are
  subtracted. Resulting numbers are found to be in G.P. again. Prove that the
  three sequences have the same common ratio.
\end{frame}
\begin{frame}{Solution of problem 27}
  \textbf{Solution:} Let $a, x$ be the first terms and $r, y$ be the common
  ratios of the first two G.P. in question. Then we are given that,
  $$(ar - xy)^2 = (a - x)(ar^2 - xy^2) \Rightarrow a^2r^2 + x^2y^2 - 2arxy =
  a^2r^2 + x^2y^2 - axr^2 - axy^2$$
  $$\Rightarrow axr^2 + axy^2 - 2arxy = 0 \Rightarrow ax(r^2 + y^2 - 2ry) = 0$$
  $$\because a, x \neq = 0, (r - y)^2 = 0 \Rightarrow r = y$$
  Common ratio of the third G. P. $\frac{ar - xy}{a - x} = \frac{r(a - x)}{a -
    x} = r$
\end{frame}
\begin{frame}{Problem 28}
  \textbf{28.} If $a, b, c, d$ are in G.P., show that $(b - c)^2 + (c - a)^2 +
  (d - b)^2 = (a - d)^2$
\end{frame}
\begin{frame}{Solution of problem 28}
  \textbf{Solution:} Let $r$ be the common ratio of the G.P., then $b = ar, c =
  ar^2, d= ar^3$
  $$L.H.S. = (ar - ar^2)^2 + (ar^2 - a)^2 + (ar^3 -ar)^2 = a^2[(r - r^2)^2 +
  (r^2 - 1)^2 + (r^3 -r)^2]$$
  $$= a^2[r^2 + r^4 - 2r^3 + r^4 + 1 - 2r^2 + r^4 + r^2 - 2r^4]$$
  $$= a^2[r^6 -2r^3 + 1] = (ar^3 - a)^2 = (d - a)^2 = R.H.S$$
\end{frame}
\end{document}
