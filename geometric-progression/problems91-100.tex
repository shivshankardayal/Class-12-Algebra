\documentclass[aspectratio=1610,8pt]{beamer}

% Standard packages

\usepackage[english]{babel}
%\usepackage[latin1]{inputenc}
%\usepackage{times}
%\usepackage[T1]{fontenc}
\usepackage{fontspec}
\usepackage[]{unicode-math}
\setmathfont{Inconsolata}
\setsansfont{Roboto}

% Setup TikZ

\usepackage{tikz}
\usetikzlibrary{arrows}
\tikzstyle{block}=[draw opacity=0.7,line width=1.4cm]


% Author, Title, etc.

\title{Geometric Progression\\Problems 91-100}

\author[Shiv Shankar Dayal]{Shiv Shankar Dayal}

% The main document

\begin{document}
\begin{frame}
  \titlepage
\end{frame}
\begin{frame}{Problem 91}
  \textbf{91.} If the sum of the series $\sum_{n = 0}^\infty r^n, |r| < 1$ is $s,$ then find the sum of the series $\sum_{n =
    0}^\infty r^{2n}$
\end{frame}
\begin{frame}{Solution of Problem 91}
  \textbf{Solution:} $$\sum_{n = 0}^\infty r^n = 1 + r + r^2 + \ldots = \frac{1}{1 - r} = s \Rightarrow r = 1 - frac{1}{s}$$
  $$\sum_{n = 0}^\infty r^{2n} = 1 + r^2 + r^4 + \ldots = \frac{1}{1 - r^2} = \frac{1}{1 - \left(1 - frac{1}{s}\right)^2}$$
\end{frame}
\begin{frame}{Problem 92}
  \textbf{92.} If for a G.P. $t_m = \frac{1}{n^2}$ and $t_n = \frac{1}{m^2}$ then find the term $t_{\frac{m + n}{2}}$
\end{frame}
\begin{frame}{Solution of Problem 92}
  \textbf{Solution:} $$t_m = ar^{m - 1} = \frac{1}{n^2}$$
  $$t_n = ar^{n - 1} = \frac{1}{m^2}$$
  Dividing, we get
  $$r^{m - n} = \frac{n^2}{m^2}\Rightarrow r = \left(\frac{n^2}{m^2}\right)^{\frac{1}{m - n}}$$

  Now, $a$ can be found and $t_{\frac{m + n}{2}}$ can be found.
\end{frame}
\begin{frame}{Problem 93}
  \textbf{93.} If $a, b, c$ be three successive terms of a G.P. with common ratio $r$ and $a < 0$ satisfying the condition $c > 4b
  - 3a,$ then prove that $r > 3$ or $r < 1.$
\end{frame}
\begin{frame}{Solution of Problem 93}
  \textbf{Solution:} $c = ar^2, b = ar.$ We have $c > 4b - 3a \Rightarrow ar^2 > 4ar - 3a$
  $$r^2 > 4r - 3 \Rightarrow (r - 1)(r - 3)> 0$$
  $$\Rightarrow r > 3~\text{or}~r < 1$$
\end{frame}
\begin{frame}{Problem 94}
  \textbf{94.} If $(1 - k)(1 + 2x + 4x^2 + 8x^3 + 16x^4 + 32x^5) = 1 - k^6,$ where $k \neq 1,$ then find $\frac{k}{x}$
\end{frame}
\begin{frame}{Solution of Problem 94}
  \textbf{Solution:} $$1 + 2x + 4x^2 + 8x^3 + 16x^4 + 32x^5 = \frac{1 - k^6}{1 - k}$$
  $$\frac{1 - (2x)^6}{1 - 2x} = \frac{1 - k^6}{1 - k}$$

  Thus, $k = 2x \Rightarrow \frac{k}{x} = 2$
\end{frame}
\begin{frame}{Problem 95}
  \textbf{95.} If $(a^2 + b^2 + c^2)(b^2 + c^2 + d^2) \leq (ab + bc + cd)^2,$ where $a, b, c, d$ are non-zero real numbers, then
  show that they are in G.P.
\end{frame}
\begin{frame}{Solution of Problem 95}
  \textbf{Solution:} Rewriting the given inequality we have
  $$(b^4 - 2b^2ac + a^2c^2) + (c^4 - 2c^2bd + b^2d^2) + (a^2d^2 - abcd + b^2c^2)\leq 0$$
  $$\Rightarrow (b^2 - ac)^2 + (c^2 - ad)^2 + (ad - bc)^2\leq 0$$

  This is possible if and only if $b^2 = ac, c^2 = bd, ac = bd \Rightarrow \frac{b}{a} = \frac{c}{b} = \frac{d}{c}$ i.e. these
  numbers are in G.P.
\end{frame}
\begin{frame}{Problem 96}
  \textbf{96.} If $a_1, a_2, \ldots, a_n$ are $n$ non-zero numbers such that $(a_1^2 + a_2^2 + \ldots + a_{n - 1}^2)(a_2^2 + a_3^2
  + \ldots + a_n^2) \leq (a_1a_2 + a_2a_3 + \ldots + a_{n - 1}a_n)^2,$ then show that $a_1, a_2, \ldots, a_n$ are in G.P.
\end{frame}
\begin{frame}{Solution of Problem 96}
  \textbf{Solution:} Considering terms involving $a_1, a_2, a_3,$ we get
  $$(a_1^2a_3^2 + a_2^4 - 2a_2^2a_1a_3) \leq 0 \Rightarrow (a_1a_3 - a_2^2)\leq 0$$

  This cannot be -ve but only zero. Thus, $a_1, a_2, a_3$ are in G.P. Similarly, we can prove for rest of the terms.
\end{frame}
\begin{frame}{Problem 97}
  \textbf{97.} $\alpha, \beta$ be the roots of $x^2 - 3x + a = 0$ and $\gamma, \delta$ be the roots of $x^2 - 12x + b = 0$ and the
  numbers $\alpha, \beta, \gamma, \delta$ form an increasing G.P., then find the values of $a$ and $b.$
\end{frame}
\begin{frame}{Solution of Problem 97}
  \textbf{Solution:} $\alpha, \beta, \gamma, \delta$ being in an increasing G.P., they may be taken as $k, kr, kr^2, kr^3,$ where
  $r > 1$

  Sum of the roots of the given equation, $S_1 = k(1 + r) = 3, S_2 = kr^2(1 + r) = 12$

  Putting $S_1$ in $S_2.$ we have $3r^2 = 12 \Rightarrow r = 2\Rightarrow k = 1$

  Product of roots, $P_1 = k^2r = a, P_2 = k^2r^5 = b$

  Thus, $a = 2, b = 32$
\end{frame}
\begin{frame}{Problem 98}
  \textbf{98.} There are $4n + 1$ terms in a certain sequence of which the first $2n + 1$ terms are in A.P. of common difference
  $2$ and the last $2n + 1$ terms are in G.P. of common ratio $\frac{1}{2}.$ If the middle terms of both the A.P. and G.P. are same
  then find the mid term of the sequence.
\end{frame}
\begin{frame}{Solution of Problem 98}
  \textbf{Solution:} Given $d = 2, r = \frac{1}{2}$

  Middle term of sequence will be $\frac{1}{2}(4n + 1 + 1)$ because no. of terms is odd. Thus, $T_{2n + 1}$ is the middle term of
  sequence, last term of A.P. and first term of G.P.

  Thus, $a + 2nd = a + 4n$

  Let $T_{n + 1}$ and $t_{n + 1}$ are mid terms of A.P. and G.P.

  $$T_{n + 1} = a + nd = a + 2n$$
  $$t_{n + 1} = T_{2n + 1}r^n = (a + 4n)\left(\frac{1}{2}\right)^n$$

  Given, $T_{n + 1} = t_{n + 1}$

  $$\therefore a + 2n = (a + 4n)\frac{1}{2^n} \Rightarrow a = \frac{4n - n.2^{n + 1}}{2^n - 1}$$

  Now mid term can be computed.
\end{frame}
\begin{frame}{Problem 99}
  \textbf{99.} If $f(x) = 2x + 1$ and three unequal numbers $f(x), f(2x), f(4x)$ are in G.P, then find the number of values for
  $x.$
\end{frame}
\begin{frame}{Solution of Problem 99}
  \textbf{Solution:} Given, $$(4x + 1)^2 = (2x + 1)(8x + 1)$$
  $$2x = 0 \Rightarrow x = 0$$
\end{frame}
\begin{frame}{Problem 100}
  \textbf{100:} Three distinct real numbers, $a, b, c$ are in G.P. such that $a + b + c = xb,$ then show that $x < -1$ or $x >
  3$
\end{frame}
\begin{frame}{Solution of Problem 100}
  \textbf{Solution:} Let $r$ be the common ratio, then, we have

  $$a + ar + ar^2 = x.ar \Rightarrow r^2 + r(1 - x) + 1 = 0$$

  Since $r$ is real, discriminant of the quadratic equation will be greater than $0.$

  $$D > 0 \Rightarrow (1 - x)^2 - 4> 0 \Rightarrow x < -1, x > 3$$
\end{frame}
\end{document}
