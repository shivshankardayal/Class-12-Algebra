\documentclass[aspectratio=1610,8pt]{beamer}

% Standard packages

\usepackage[english]{babel}
%\usepackage[latin1]{inputenc}
%\usepackage{times}
%\usepackage[T1]{fontenc}


% Setup TikZ

\usepackage{tikz}
\usetikzlibrary{arrows}
\tikzstyle{block}=[draw opacity=0.7,line width=1.4cm]


% Author, Title, etc.

\title{Geometric Progression\\Problems 1-10}

\author[Shiv Shankar Dayal]{Shiv Shankar Dayal}

% The main document

\begin{document}
\begin{frame}
       \titlepage
\end{frame}
\begin{frame}{Problem 1}
  \textbf{1.} How many terms are in the G.P. $5, 20, 80, ..., 5120$?
\end{frame}
\begin{frame}{Solution of problem 1}
  \textbf{Solution:} Given $a = 5$ and $r = \frac{20}{5} = 4$\\
  Let there are $n$ terms in G.P.\\
  The formula for $t_n$ is $t_n = ar^{n - 1}$
  $$t_n = ar^{n - 1} = 5120 \Rightarrow 5.4^{n - 1} = 5120$$
  $$\Rightarrow 4^{n - 1} = 1024 = 4^{5}\Rightarrow n - 1 = 5$$
  $$\therefore n = 6$$
\end{frame}
\begin{frame}{Problem 2}
  \textbf{2.} How many terms are in the G.P. $0.03, 0.06, 0.12, \ldots, 3.84$?
\end{frame}
\begin{frame}{Solution of problem 2}
  \textbf{Solution:} Given $a = 0.03$ and $r = \frac{0.06}{0.03} = 2$\\
  Let there are $n$ terms in G.P.\\
  The formula for $t_n$ is $t_n = ar^{n - 1}$
  $$t_n = ar^{n - 1} = 3.84$$
  $$\Rightarrow 0.03.2^{n - 1} = 3.84$$
  $$\Rightarrow 2^{n - 1} = 128 = 2^7$$
  $$\Rightarrow n -1 = 7 \Rightarrow n = 8$$
\end{frame}
\begin{frame}{Problem 3}
  \textbf{3.} A boy agrees to work at the rate of one rupee the first day, two
  rupee the second day, four rupees the third day, eight rupees the fourth day
  and so on. How much would he get on $20th$ day?
\end{frame}
\begin{frame}{Solution of problem 3}
  \textbf{Solution:} Clearly, the money gained by the boy is in G.P. with $a =
  1$ and $r = 2$\\
  Thus, the money made by boy on $20$th day $$t_{20} = 1.2^{20 -1} = 2^{19}$$
  $$\therefore t_{20} = 524288$$
\end{frame}
\begin{frame}{Problem 4}
  \textbf{4.} The population of a city in January $1987$ was $20,000$. It
  increased at the rate of $2\%$ per annum. Find the population of the city in
  January $1997$.
\end{frame}
\begin{frame}{Solution of problem 4}
  \textbf{Solution:} Population in $1988 = 20000 + \frac{2}{100}\times 20000 =
  20000 * 1.02$\\
  Population in $1989 = 20000*1.02 + \frac{2}{100}\times 20000 * 1.02 =
  20000*(1.02)^2$
  
  Thus, we see that it is a geometric progression with $a = 20000$ and $r =
  1.02$

  Thus, after $10$ years population in $1997 = 20000 * (1.02)^{10} = 24379$
\end{frame}
\begin{frame}{Problem 5}
  \textbf{5.} The sum of $n$ terms of a sequence is $2^n - 1,$ find its $n$th
  term. Is the sequence in G.P.?
\end{frame}
\begin{frame}{Solution of problem 5}
  \textbf{Solution:}Given, $S_n = 2^n - 1 \therefore S_{n - 1} = 2^{n - 1} - 1$

  $$t_n = S_n - S_{n - 1} = 2^n - 1 - 2^{n - 1} + 1 = 2^n - 2^{n - 1} = 2^{n -
    1}(2 -1) = 2^{n - 1}$$
  $$\therefore \frac{t_n}{t_{n - 1}} = \frac{2^{n - 1}}{2^{n - 2}} = 2$$

  Since the ratio of consecutive terms is a constant and independent of $n$ the
  sequence is in G.P.
\end{frame}
\begin{frame}{Problem  6}
  \textbf{6.} If the fifth term of a G.P. is $81$ and second term is $24$. Find
  the G.P.
\end{frame}
\begin{frame}{Solution of problem 6}
  \textbf{Solution:} Let $a$ be the first term and $r$ be the common ratio of
  the G.P.

  $$t_2 = ar = 24~\text{and}~t_5 = ar^4 = 81$$
  Dividing we get, $$r^3 = \frac{81}{24} = \frac{27}{8}$$
  $$r = \frac{3}{2}$$
  Substituring the value of $r$ for $t_2$
  $$t_2 = ar = 24 \Rightarrow a = \frac{24}{r} = \frac{24.2}{3} = 16$$
  Therefore, the G.P. is $16, 24, 36, 54, 81, \ldots$
\end{frame}
\begin{frame}{Problem 7}
 \textbf{7.} The seventh term of a G.P. is $8$ times the fourth term. Find the
 G.P. when its $5$th term is 48.
\end{frame}
\begin{frame}{Solution of problem 7}
  \textbf{Solution:} Let $a$ be the first term and $r$ be the common ratio of
  the G.P.
  $$t_7 = ar^6~\text{and}~t_4 = ar^3$$
  Given, $t_7 = 8. t_4$
  $$ar^6 = 8.ar^3\Rightarrow r^3 = 8 \Rightarrow r = 2$$
  Also, given $t_5 = 48 \Rightarrow ar^4 = 48 \Rightarrow a = \frac{48}{2^4} =
  3$
  
  Thus, G.P. is 3, 6, 12, 24, ...
\end{frame}
\begin{frame}{Problem 8}
  \textbf{8.} If the 5th and 8th terms of a G.P. be 48 and 384 respectively,
  find the G.P.
\end{frame}
\begin{frame}{Solution of problem 8}
  \textbf{Solution:} Let $a$ be the first term and $r$ be the common ratio of
  G.P.

  Given, $t_5 = ar^4 = 48$ and $t_8 = ar^7 = 384$

  Dividing we get, $$\frac{t_8}{t_5} = \frac{ar^7}{ar^4} = r^3 =
  \frac{384}{48} = 8$$

  $$\Rightarrow r = 2$$

  Substituting the value of $r$ in $t_5, a.2^4 = 48 \Rightarrow a = 3$

  Thus, G.P. is 3, 6, 12, 24, ...
\end{frame}
\begin{frame}{Problem 9}
  \textbf{9.} If the 6th and 10th terms of a G.P. are $\frac{1}{16}$ and
  $\frac{1}{256}$ respectively, find the G.P.
\end{frame}
\begin{frame}{Solution of problem 9}
  \textbf{Solution:} Let $a$ be the first term and $r$ be the common ratio of
  G.P.

  Given, $t_6 = ar^5 = \frac{1}{16}$ and $t_{10} = ar^9 = \frac{1}{256}$

  Dviding we get, $\frac{t_{10}}{t_6} = r^4 = \frac{1}{256}16 = \frac{1}{16}$

  $$\Rightarrow r = \pm\frac{1}{2}$$

  Substituting the value of $r$ in $t_6$, we get

  $$t_6 = a\pm\frac{1}{32} = \frac{1}{16}\Rightarrow a = \pm 2$$

  Thus, the G.P. is either $2, 1, \frac{1}{2}, \frac{1}{4}, \ldots$ or $-2, 1,
  -\frac{1}{2}, \frac{1}{4}, \ldots$.
\end{frame}
\begin{frame}{Problem 10}
  \textbf{10.} If the $p$th, $q$th and $r$th terms of a G.P. be $a, b, c (a, b,
  c >0)$, then prove that $(q - r)\log a + (r - p)\log b + (p - q)\log c = 0$
\end{frame}
\begin{frame}{Solution of problem 10}
  \textbf{Solution:} Let $x$ be the first term and $y$ be the common
  ratio. Then,
  $$a = xy^{p - 1} \Rightarrow \log a = \log x + (p - 1)\log y$$
  $$b = xy^{q - 1} \Rightarrow \log b = \log x + (q - 1)\log y$$
  $$c = xy^{r - 1} \Rightarrow \log c = \log x + (r - 1)\log y$$

  Now,$$(q - r)\log a + (r - p)\log b + (p - q)\log c = (q - r)[\log x + (p -
    1)\log y + (r - p)[\log x + (q - 1)\log y] + (p - q)[\log x + (r - 1)\log
      y]$$
    $$= \log x(p - q + r - p + p - q) + \log y[(q - r)(p - 1) + (r - p)(q - 1) +
      (p - q)(r - 1)]$$
    $$= \log y(pq - q - pr + r + qr -r - pq + p + pr - p - qr + q)$$
    $$= 0$$
\end{frame}
\end{document}
