\documentclass[aspectratio=169,8pt]{beamer}

% Standard packages

\usepackage[english]{babel}
%\usepackage[latin1]{inputenc}
%\usepackage{times}
%\usepackage[T1]{fontenc}
\usepackage{fontspec}
\usepackage[]{unicode-math}
\setmathfont{Consolas}
\setsansfont{Arial}
% Setup TikZ

\usepackage{tikz}
\usetikzlibrary{arrows}
\tikzstyle{block}=[draw opacity=0.7,line width=1.4cm]

\newcounter{counter}

% Author, Title, etc.
\title{Logarithm Problem 81-90}

\author[Shiv Shankar Dayal]{Shiv Shankar Dayal}

% The main document

\begin{document}
\begin{frame}
  \titlepage
\end{frame}
\begin{frame}{Problem 81}
  \textbf{81.} Solve $(5 + 2\sqrt{6}){x^2 - 3} + (5 - 2\sqrt{6})^{x^2 - 3} = 10$
\end{frame}
\begin{frame}{Solution of Problem 81}
  \textbf{Solution:} $$\text{Given,~}(5 + 2\sqrt{6}){x^2 - 3} + (5 - 2\sqrt{6})^{x^2 - 3} = 10$$
  $$\Rightarrow (5 + 2\sqrt{6})^{x^2 - 3} + (5 + 2\sqrt{6})^{-(x^2 - 3)} = 10$$
  $$\text{Let~}z = (5 + 2\sqrt{6})^{x^2 - 3}\text{,~then we can rewrite above as}$$
  $$z + \frac{1}{z} = 10$$
  $$z = 5 \pm 2\sqrt{6}$$
  $$\therefore x = \pm2, \pm\sqrt{2}$$
\end{frame}
\begin{frame}{Problem 82}
  \textbf{82.} For $x > 1,$ show that $2\log_{10}x - \log_x.01\geq 4$
\end{frame}
\begin{frame}{Solution of Problem 82}
  \textbf{Solution:} $$2\log_{10}x - \log_x.01 = 2\log_{10}x - \log_x10^{-2}$$
  $$= 2\log_{10}x + 2\log_x10 = 2\log_{10}x + 2\frac{1}{\log_{10}x}$$
  $$= 2\left(\log_{10}x + \frac{1}{\log_{10}x}\right)$$
  $$= 2\left[\left(\sqrt{\log_{10}x} - \frac{1}{\sqrt{\log_{10}x}}\right)^2 + 2\right] \geq 4$$
\end{frame}
\begin{frame}{Problem 83}
  \textbf{83.} Show that $|\log_ba + \log_ab| > 2$
\end{frame}
\begin{frame}{Solution of Problem 83}
  \textbf{Solution:} Let $E = |\log_ba + \log_ab|$
  \linebreak\linebreak
  Also, let $z = \log_ba,$ then we can rewrite above as $E = \left|z + \frac{1}{z}\right|$
  \linebreak\linebreak
  Clearly, $z\neq 0,$ or the problem will be undefined. When $z > 0, E = \left(\sqrt{z} - \frac{1}{\sqrt{z}}\right)^2 + 2 > 2$
  \linebreak\linebreak
  When $z < 0,$ let $z = -y,$ then $E = \left|z + \frac{1}{z}\right| = \left|-y - \frac{1}{y}\right| = y + \frac{1}{y} > 2$
\end{frame}
\begin{frame}{Problem 84}
  \textbf{84.} Solve $\log_{0.3}(x^2 + 8) > \log_{0.3}9x$
\end{frame}
\begin{frame}{Solution of Problem 84}
  \textbf{Solution:} $$\text{Given,~}\log_{0.3}(x^2 + 8) > \log_{0.3}9x$$
  $$\Rightarrow x^2 + 8 < 9x$$
  $$\Rightarrow (x - 1)(x - 8) < 0$$
  $$\Rightarrow 1 < x < 8$$
\end{frame}
\begin{frame}{Problem 85}
  \textbf{85.} Solve $\log_{x - 2}(2x - 3) >\log_{x - 2}(24 - 6x)$
\end{frame}
\begin{frame}{Solution of Problem 85}
  \textbf{Solution:} $$\text{Given,~}\log_{x - 2}(2x - 3) > \log_{x - 2}(24 - 6x)$$
  $$\text{Case I: When~} 0 < x - 2 < 1 \Rightarrow 2< x < 3$$
  $$\text{Given inequality becomes~} 2x - 3 < 24 - 6x \Rightarrow x < \frac{27}{8}$$
  $$\text{But~}x < 3\text{~so~}3\text{~is still limiting value of~}x$$
  $$\text{Case II: When~}x - 2 > 1 \Rightarrow x > 3$$
  $$2x - 3 > 24 - 6x\Rightarrow x > \frac{27}{8}$$
  However, for logarithm to be defined $2x - 3 > 0$ and $24 - 6x > 0$ and also $x - 2 > 0.$ Combining all these we get $2 < x < 3$
\end{frame}
\begin{frame}{Problem 86}
  \textbf{86.} Find the interval in which $x$ will lie if $\log_{0.3}(x - 1) < \log_{0.09}(x - 1)$
\end{frame}
\begin{frame}{Solution of Problem 86}
  \textbf{Solution:} $$\text{Given,~}\log_{0.3}(x - 1) < \log_{0.09}(x - 1)$$
  $$\Rightarrow \log_{0.3}(x - 1) < \log_{0.3^2}(x - 1)$$
  $$(x - 1)^2 > (x - 1)$$
  $$\Rightarrow x^2 - 3x + 2 > 0$$
  $$\Rightarrow x < 1, x > 2$$
  For logarithm to be defined $x - 1 > 0$ i.e. $x > 1,$ thus the interval for $x$ would be $(2, \infty]$
\end{frame}
\begin{frame}{Problem 87}
  \textbf{87.} Solve $\log_{\frac{1}{2}}x \geq \log_{\frac{1}{3}}x$
\end{frame}
\begin{frame}{Solution of Problem 87}
  \textbf{Solution:} $$\text{Given,~}\log_{\frac{1}{2}}x \geq \log_{\frac{1}{3}}x$$
  $$\Rightarrow \log_{\frac{1}{2}}x \geq \log_{\frac{1}{2}}x\log_{\frac{1}{3}}\frac{1}{2}$$
  $$\Rightarrow \log_{\frac{1}{2}}x\left[1 - \log_{\frac{1}{3}}\frac{1}{2}\right]\geq 0$$
  $$\Rightarrow \log_{\frac{1}{2}}x\left[1 - \log_{3^{-1}}2^{-1}\right]\geq 0$$
  $$\Rightarrow \log_{\frac{1}{2}}x\left[1 - \log_32\right]\geq 0$$
  $$\Rightarrow \log_{\frac{1}{2}}x \geq 0$$
  $$\Rightarrow x \leq 1$$
  For logarithm to be defined $x > 0,$ thus range of $x$ would be $(0, 1]$
\end{frame}
\begin{frame}{Problem 88}
  \textbf{88.} Solve $\log_{\frac{1}{2}}\log_4(x^2 - 5) > 0$
\end{frame}
\begin{frame}{Solution of Problem 88}
  \textbf{Solution:} $$\text{Given,~}\log_{\frac{1}{2}}\log_4(x^2 - 5) > 0$$
  $$\Rightarrow \log_4(x^2 - 5) < 1$$
  $$\Rightarrow x^2 - 5 < 4$$
  $$\Rightarrow x^2 < 9 \Rightarrow -3 < x < 3$$
  For logarithm to be defined $x^2 - 5 > 0$ and $\log_4(x^2 - 5)>0 \Rightarrow x^2 - 5 > 1 \Rightarrow x < -\sqrt{6}, x > \sqrt{6}$.

  Thus, the two ranges for $x$ are $(-3, -\sqrt{6})$ and $(\sqrt{6}, 3)$
\end{frame}
\begin{frame}{Problem 89}
  \textbf{89.} Solve $\log(x^2 - 2x - 2)\leq 0$
\end{frame}
\begin{frame}{Solution of Problem 89}
  \textbf{Solution:} $$\text{Given,~}\log(x^2 - 2x - 2)\leq 0$$
  $$\Rightarrow x^2 - 2x - 2 \leq 1$$
  $$\Rightarrow (x - 3)(x + 1)\leq 0$$
  $$-1 \leq x \leq 3$$
  For logarithm to be defined $x^2 - 2x + 2 > 0 \Rightarrow x < 1 - \sqrt{3}, x > 1 + \sqrt{3}$

  Thus, the ranges are $[-1, 1 - \sqrt{3}), (1 + \sqrt{3}, 3]$
\end{frame}
\begin{frame}{Problem 90}
  \textbf{90.} Solve $\log_{2^2}(x - 1)^2 - \log_{0.5}(x - 1)> 5$
\end{frame}
\begin{frame}{Solution of Problem 90}
  \textbf{Solution:} $$\text{Given,~}\log_{2^2}(x - 1)^2 - \log_{0.5}(x - 1)> 5$$
  $$\Rightarrow (2\log_2|x - 1|)^2 - \log_{0.5}(x - 1) > 5$$
  $$\Rightarrow 4[\log_2(x - 1)]^2 + \log_2(x - 1) > 5$$
  $$[\because \text{~for~}\log_{0.5}(x - 1) \text{~to be defined~} x- 1> 0 \therefore |x - 1| = x -1]$$
  $$\log_2(x - 1)< \frac{-5}{4}, \log_2(x - 1) > 1$$
  $$\text{When~}\log_2(x - 1)<\frac{-5}{4} \Rightarrow x < 1 + \frac{1}{2\sqrt[4]{2}}$$
  For logarithm to be defined $x - 1>0 \Rightarrow 1< x < 1 + \frac{1}{2\sqrt[4]{2}}$
  $$\text{When~}\log_2(x - 1)> 2 \Rightarrow x > 3$$
  Thus, ranges are $\left(1, 1 + \frac{1}{2\sqrt[4]{2}}\right), (3, \infty]$
\end{frame}
\end{document}
