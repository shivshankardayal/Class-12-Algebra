\documentclass[aspectratio=1610,8pt]{beamer}

% Standard packages

\usepackage[english]{babel}
%\usepackage[latin1]{inputenc}
%\usepackage{times}
%\usepackage[T1]{fontenc}
\usepackage{fontspec}
\usepackage[]{unicode-math}
\setmathfont{Inconsolata}
\setsansfont{Roboto}
% Setup TikZ

\usepackage{tikz}
\usetikzlibrary{arrows}
\tikzstyle{block}=[draw opacity=0.7,line width=1.4cm]

\newcounter{counter}

% Author, Title, etc.
\title{Logarithm Problem 41-50}

\author[Shiv Shankar Dayal]{Shiv Shankar Dayal}

% The main document

\begin{document}
\begin{frame}
  \titlepage
\end{frame}
\begin{frame}{Problem 41}
  \textbf{41.} If $\frac{\log x}{q - r} = \frac{\log y}{r - p} = \frac{\log z}{p - q},$ prove that $x^{q+r}y^{p+r}z^{p+q} = x^py^qz^r$
\end{frame}
\begin{frame}{Solution of Problem 41}
  \textbf{Solution:} $$\frac{\log x}{q - r} = \frac{\log y}{r - p} = \frac{\log z}{p - q} = k(\text{let})$$
  $$\Rightarrow \log x = k(q - r), \log y = k(r - p), \log z = k(p - q)$$
  We have to prove that $x^{q+r}y^{p+r}z^{p+q} = x^py^qz^r.$ Taking $\log$ of both sides
  $$(q + r)\log x + (p + r)\log y + (p + q)\log z = p\log x + q\log y + r\log z$$
  $$k(q^2 - r^2) + k(r^2 - p^2) + k(p^2 - q^2) = k(pq - pr + qr - pq + pr - qr)$$
  $$0 = 0$$
\end{frame}
\begin{frame}{Problem 42}
  \textbf{42.} If $y = a^{\frac{1}{1 - \log_ax}}$ and $z = a^{\frac{1}{1 -\log_ay}},$ prove that $x = a^{\frac{1}{1 - \log_az}}$
\end{frame}
\begin{frame}{Solution of Problem 42}
  \textbf{Solution:} Given $y = a^{\frac{1}{1 - \log_ax}}$ and $z = a^{\frac{1}{1 -\log_ay}}$
  $$\Rightarrow z = a^{\frac{1}{1 - \log_a{a^{\frac{1}{1 - \log_ax}}}}}$$
  $$z = a^{\frac{1}{1 - {\frac{1}{1 - \log_ax}}}}$$
  Taking $\log$ of both sides with base $a,$ we get
  $$\log_az = \frac{1}{1 - \frac{1}{1-\log_ax}}$$
  $$= \frac{1 - \log_ax}{-\log_ax} = 1 - \frac{1}{\log_ax}$$
  $$x = a^{\frac{1}{1 - \log_az}}$$
\end{frame}
\begin{frame}{Problem 43}
  \textbf{43.} Let $f(x) = \frac{1}{1 - \log_e x}.$ If $f(y) = e^{f(z)}$ and $z = e^{f(x)},$ prove that $x = e^{f(y)}$
\end{frame}
\begin{frame}{Solution of Problem 43}
  \textbf{Solution:} $$f(y) = e^{\frac{1}{1 - \log_az}}, z = e^{\frac{1}{1- \log_ex}}$$
  $$\Rightarrow f(y) = e^{\frac{1}{1 - \log_ee^{\frac{1}{1 - \log_ex}}}}$$
  $$f(y) = e^{\frac{1}{1 - {\frac{1}{1 - \log_ex}}}}$$
  Following like previous problem
  $$x = e^{f(y)}$$
\end{frame}
\begin{frame}{Problem 44}
  \textbf{44.} Show that $\frac{1}{\log_2n} + \frac{1}{\log_3n} + \frac{1}{\log_4n} + \ldots + \frac{1}{\log_{43}n} = \frac{1}{\log_{43!}n}$
\end{frame}
\begin{frame}{Solution of Problem 44}
  \textbf{Solution:} $$L.H.S. = \frac{1}{\log_2n} + \frac{1}{\log_3n} + \frac{1}{\log_4n} + \ldots + \frac{1}{\log_{43}n}$$
  $$= \log_n2 + \log_n3 + \log_n4 + \ldots + \log_n43$$
  $$= \log_n(2.3.4.\ldots 43) = \log_n43! = \frac{1}{\log_{43!}n}$$
\end{frame}
\begin{frame}{Problem 45}
  \textbf{45.} Show that $2(\log a + \log a^2 + \log a^3 + \ldots + \log a^n) = n(n + 1)\log a$
\end{frame}
\begin{frame}{Solution of Problem 45}
  \textbf{Solution:} $$L.H.S. = 2(\log a + \log a^2 + \log a^3 + \ldots + \log a^n)$$
  $$= 2\log a(1 + 2 + 3 + \ldots + n) = 2\log a\frac{n(n + 1)}{2}$$
  $$= n(n +1)\log a$$
\end{frame}
\begin{frame}{Problem 46}
  \textbf{46.} Find the number of digits in $12^{12},$ without actual computation. Given $\log 2 = 0.301$ and $\log 3 = 0.477$
\end{frame}
\begin{frame}{Solution of Problem 46}
  \textbf{Solution:} We will make use of the fact that positive characteristics of $n$ of a logarithm there are $n + 1$ digits in the number.
  \linebreak\linebreak
  Let $y = 12^{12} \Rightarrow \log y = 12\log 12 = 12\log (2.2.3) = 12[2*0.301 + 0.477]$
  \linebreak\linebreak
  $= 12.96$
  \linebreak\linebreak
  Thus, number of digits is $13.$
\end{frame}
\begin{frame}{Problem 47}
  \textbf{47.} How many positive integers have characteristics $2$ when base is $3?$
\end{frame}
\begin{frame}{Solution of Problem 47}
  \textbf{Solution:} Number of positive integers having base $b$ and characteristics $n$ is $b^{n + 1} - b^n$
  \linebreak\linebreak
  Thus, number of integers with base $3$ and characteristics $2$ is $3^3 - 3^2 = 18.$
\end{frame}
\begin{frame}{Problem 48}
  \textbf{48.} How many zeros are there between the decimal point and the first significant digit in $0.0504^{10}.$ Given, $\log 2 = 0.301, \log 3 = 0.477, \log 7 = 0.845$
\end{frame}
\begin{frame}{Solution of Problem 48}
  \textbf{Solution:} Let $y = 0.0504^{10}$
  \linebreak\linebreak
  $\log_{10}y = 10\log_{10}0.0504 = 10\log_{10}(504*10^{-4})$
  \linebreak\linebreak
  $= 10\log_{10}[-4 + \log(2^3.3^2.7)]$
  \linebreak\linebreak
  $= -12.98$
  \linebreak\linebreak
  Thus, characteristics is $-13,$ therefore number of zeros after decimal and first significant digit $= 12$
\end{frame}
\begin{frame}{Problem 49}
  \textbf{49.} Find the number of digits in $72^{15}$ without actual computation. Given $\log 2 = 0.301, \log 3 = 0.477.$
\end{frame}
\begin{frame}{Solution of Problem 49}
  \textbf{Solution:} Let $x = 72^{15}~\therefore~ \log_{10}x = 15\log_{10}72$
  \linebreak\linebreak
  $= 15\log_{10}(2^3*3^2) = 15\log_{10}[3\log_{10}2 + 2\log_{10}3]$
  \linebreak\linebreak
  $= 15[3*0.301 + 2*0.477] = 27.855$
  \linebreak\linebreak
  So characteristics is $27$, therefore, the number of digits will be $28.$
\end{frame}
\begin{frame}{Problem 50}
  \textbf{50.} How many positive integers have characteritics $2$ when base is $5?$
\end{frame}
\begin{frame}{Solution of Problem 50}
  \textbf{Solution:} Number of integers with base $b$ and characteristics $n$ is $b^{n + 1} - b^n.$
  \linebreak\linebreak
  $\therefore ~$ number of integers is $5^3 - 5^2 = 100$
\end{frame}
\end{document}
